%% $Id: pst-platon-doc.tex 208 2009-12-25 08:54:53Z herbert $
\documentclass[11pt,english,BCOR10mm,DIV12,bibliography=totoc,parskip=false,smallheadings
    headexclude,footexclude,oneside]{pst-doc}
\usepackage[utf8]{inputenc}
\usepackage{pst-platon}

\let\pstFV\fileversion
\lstset{language=PSTricks,basicstyle=\footnotesize\ttfamily}
%
\def\bgImage{\psset{psscale=2}\psIcosahedron%[Frame=false,Viewpoint=-1 0.5 1.2]}
}
\title{\texttt{pst-platon}}
\subtitle{A PSTricks package for drawing platonic solids; v.\pstFV}
\author{Manuel Luque \\ Herbert Vo\ss}
\docauthor{}
\date{\today}

\begin{document}
\maketitle

\begin{abstract}
A platonic solid is a convex polyhedron that is a regular polygon.
The faces of a platonic solid are congruent regular polygons, 
with the same number of faces meeting at each vertex. 
All edges are congruent, as are its vertices and angles.
There exists five platonic solids.
\end{abstract}

\vfill
\clearpage


\tableofcontents

\newpage

\section{The optional Arguments}
\subsection{\nxLkeyword{PstPicture}}
With \Lkeyset{PstPicture=true} (default) the image is set into a \Lenv{pspicture} environment, which 
reserves some space. The correct bounding box depends to the viewpoint. With setting of \Lkeyset{PstPicture=false}
you can set the image inside your own \Lenv{pspicture} environment with other coordinates. All
solids areplaced relative to the origin of the coordinate system. Use \Lcs{rput} to place the
platonic solid elsewhere.

\begin{LTXexample}[pos=t,rframe=]
\begin{pspicture}[showgrid=true](-1,-2)(10,5)
\psTetrahedron[PstPicture=false]
\rput(2,2){\psTetrahedron[PstPicture=false,Viewpoint=1 1.2 0.5]}
\psset{unit=1.3}
\rput(5,3){\psTetrahedron[PstPicture=false,Frame=false,Viewpoint=-1 0.5 2]}
\end{pspicture}
\end{LTXexample}

\subsection{\nxLkeyword{Frame}}

With \Lkeyset{Frame=true} (default) the unique cube with a=1 is printed with
dotted lines.

\begin{LTXexample}[width=7cm,rframe=]
\psTetrahedron 
\psTetrahedron[Frame=false] 
\end{LTXexample}

\clearpage

\subsection{\nxLkeyword{Viewpoint}}

With \Lkeyword{Viewpoint} the three dimensional view point from which the 
solid is seen can be set. The default is \verb=1 1 1=.

\begin{LTXexample}[width=10.5cm,rframe=,wide]
\psTetrahedron
\psTetrahedron[Viewpoint=-1 1 .5]
\psTetrahedron[Viewpoint=0.4 -1 .5] 
\end{LTXexample}

\subsection{\nxLkeyword{faceName}}

With \Lkeyword{faceName} the name of the faces can be set with setting
it to one of the macros \Lcs{Alph} (default), \Lcs{alph}, \Lcs{arabic},
\Lcs{Roman}, and \Lcs{roman}.

\begin{LTXexample}[width=10.5cm,rframe=,wide]
\psHexahedron%
\psHexahedron[faceName=\alph]% 
\psHexahedron[faceName=\Roman] 
\end{LTXexample}


\subsection{\nxLkeyword{faceNameFont}}

With \Lkeyword{faceNameFont} the font for the face name can be set.
Any valid \LaTeX\ command is possible.

\begin{LTXexample}[width=10.5cm,rframe=,wide]
\psHexahedron%
\psHexahedron[faceNameFont=\Huge]% 
\psHexahedron[faceNameFont=\Huge\sffamily] 
\end{LTXexample}

\subsection{\nxLkeyword{psscale}}

The solids can be magnified by the keyword \Lkeyword{psscale}
which is preset to 1.

\begin{LTXexample}[width=8cm,rframe=]
\psOctahedron[Frame=false]
\psOctahedron[Frame=false,psscale=2]
\end{LTXexample}

\subsection{Colors}
The faces are defined by the colors of type A or B with 
\begin{verbatim}
\newcommand\colorTypeA{%
\definecolor{ColorA}{cmyk}{0.1,0.1,0.05,0}
\definecolor{ColorB}{cmyk}{0.15,0.15,0.05,0}
...
}
\newcommand\colorTypeB{%
\definecolor{ColorA}{cmyk}{0.1,0.2,0.1,0}
\definecolor{ColorB}{cmyk}{0.15,0.2,0.15,0}
...
}
\end{verbatim}

New types can be definied in the same way and then set by the keyword \Lkeyword{colorType}=\Larga{type}.

\begin{LTXexample}[width=5cm,rframe=]
\newcommand\colorTypeC{%
  \colorlet{ColorA}{red}
  \colorlet{ColorB}{green}
  \colorlet{ColorC}{blue}
  \definecolor{ColorD}{rgb}{0.55,0.2,0.15}
}
\psTetrahedron[colorType=C]
\end{LTXexample}


\section{The Platonic Solids}
There are the five platonic solids with the macronames 
\Lcs{psTetrahedron}, \Lcs{psHexahedron}, \Lcs{psOctahedron}, \Lcs{psDodecahedron}, 
and \Lcs{psIcosahedron}.


\subsection{Tetrahedron}

\begin{LTXexample}[width=5cm,rframe=]
\psTetrahedron
\end{LTXexample}

\begin{LTXexample}[width=5cm,rframe=]
\psTetrahedron[Viewpoint=1 1.2 0.5]
\end{LTXexample}

\begin{LTXexample}[width=5cm,rframe=]
\psTetrahedron[Frame=false,Viewpoint=0.7 -0.5 -0.8]
\end{LTXexample}

\begin{LTXexample}[pos=t,rframe=]
\psTetrahedron[Frame=false,Viewpoint=1 1.2 0.7] 
\psTetrahedron[Frame=false,Viewpoint=-1 0.5 2] 
\psTetrahedron[Frame=false,Viewpoint=0.7 -0.5 -0.8]
\end{LTXexample}

\subsection{Hexahedron}
\begin{LTXexample}[width=5cm,rframe=]
\psHexahedron
\end{LTXexample}

\begin{LTXexample}[width=5cm,rframe=]
\psHexahedron[Viewpoint=1 1.2 0.5]
\end{LTXexample}

\begin{LTXexample}[width=5cm,rframe=]
\psHexahedron[Frame=false,Viewpoint=0.7 -0.5 -0.8]
\end{LTXexample}

\begin{LTXexample}[pos=t,rframe=]
\psHexahedron[Frame=false,Viewpoint=1 1.2 0.7] 
\psHexahedron[Frame=false,Viewpoint=-1 0.5 2] 
\psHexahedron[Frame=false,Viewpoint=0.7 -0.5 -0.8]
\end{LTXexample}


\subsection{Octahedron}
\begin{LTXexample}[width=5cm,rframe=]
\psOctahedron
\end{LTXexample}

\begin{LTXexample}[width=5cm,rframe=]
\psOctahedron[Viewpoint=1 1.2 0.5]
\end{LTXexample}

\begin{LTXexample}[width=5cm,rframe=]
\psOctahedron[Frame=false,Viewpoint=0.7 -0.5 -0.8]
\end{LTXexample}

\begin{LTXexample}[pos=t,rframe=]
\psset{psscale=2}
\psOctahedron[Frame=false,Viewpoint=1 1.2 0.7] 
\psOctahedron[Frame=false,Viewpoint=-1 0.5 2] 
\psOctahedron[Frame=false,Viewpoint=0.7 -0.5 -0.8]
\end{LTXexample}

\clearpage
\subsection{Dodecahedron}
\begin{LTXexample}[width=5cm,rframe=]
\psDodecahedron
\end{LTXexample}

\begin{LTXexample}[pos=t,rframe=]
\psDodecahedron[Viewpoint=-0.5 0.9 0.9]
\psDodecahedron[Viewpoint=-0.5 0.7 -1.2]
\psDodecahedron[Viewpoint=0.5 -0.7 -0.5]
\end{LTXexample}

\begin{LTXexample}[pos=t,rframe=]
\psDodecahedron[Frame=false,Viewpoint=-0.2 0.2 0.2]
\psDodecahedron[Frame=false,Viewpoint=-0.707 -0.707 -1]
\psDodecahedron[Frame=false,Viewpoint=0.6 -0.7 -0.5]
\end{LTXexample}

\subsection{Isocahedron}

\begin{LTXexample}[width=5cm,rframe=]
\psIcosahedron
\end{LTXexample}

\begin{LTXexample}[pos=t,rframe=]
\psIcosahedron[Viewpoint=1 1.2 0.5]
\psIcosahedron[Viewpoint=-1 1.2 0.5]
\psIcosahedron[Viewpoint=-1 -1.2 0.5]
\psIcosahedron[Viewpoint=1 -1.2 0.5]
\end{LTXexample}

\begin{LTXexample}[pos=t,rframe=]
\psIcosahedron[Frame=false,Viewpoint=0.5 -1 1]
\psIcosahedron[Frame=false,Viewpoint=-1 0.5 1.2]
\psIcosahedron[Frame=false,Viewpoint=0.7 -0.5 -0.8]
\psIcosahedron[Frame=false,Viewpoint=-0.7 -0.7 -0.2]
\end{LTXexample}



\section{List of all optional arguments for \texttt{pst-platon}}

\xkvview{family=pst-platon,columns={key,type,default}}





\nocite{*}
\bgroup
\RaggedRight
\bibliographystyle{plain}
\bibliography{pst-platon-doc}
\egroup

\printindex



\end{document}

