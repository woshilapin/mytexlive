%% $Id: pst-news09.tex 62 2008-11-25 19:52:07Z herbert $
\documentclass[11pt,english,BCOR10mm,DIV12,bibliography=totoc,parskip=false,smallheadings
    headexclude,footexclude,oneside]{pst-doc}
\listfiles
\let\Lfile\LFile
\usepackage[utf8]{inputenc}
\usepackage{pst-node,pst-plot}
\usepackage{xkvview}
\let\pstnodeFV\fileversion
\let\pstnodeFD\filedate
\renewcommand\bgImage{\psscalebox{15}{\color{blue!20}2012}}
\def\textat{\char064}
\lstset{explpreset={pos=l,width=-99pt,overhang=0pt,hsep=\columnsep,vsep=\bigskipamount,rframe={}},
    escapechar=?}
\begin{document}

%\psset{PstDebug=1}
\title{\texttt{News -- 2012}\\ \Large new macros and bugfixes for the
basic package \nxLFile{pstricks}}
\author{Herbert Voß}
\date{\today}

\maketitle

\clearpage
\tableofcontents

\clearpage
\part{\texttt{pstricks} -- package}

\section{General}
There exists a new document class \LClass{pst-doc} for writing PSTricks documentations,
like this news document. It depends on the KOMA-Script document class \LClass{scrartcl}.
\LClass{pst-doc} defines a lot of special macros to create a good index. Take one of
the already existing package documentation and look into the source file. Then it will be
easy to understand, how all these macros have to be used.

When running \Lprog{pdflatex} the title page is created with boxes and inserted 
with the macro \Lcs{AddToShipoutPicture} from the package \LPack{eso-pic}. It
inserts the background title page image \Lfile{pst-doc-pdf} to use directly
\Lprog{pdflatex}.
When running \Lprog{latex} the title page
 is created with \PST\ macros.This allows to use the Perl script \Lprog{pst2pdf} or
the package \LPack{pst-pdf} or \LPack{auto-pst-pdf} or any other program/package which
supports \PS\ code in the document.


%--------------------------------------------------------------------------------------
\section{\texttt{pstricks.sty}}
%--------------------------------------------------------------------------------------
\subsection{New optional argument}


%--------------------------------------------------------------------------------------
\section{\texttt{pstricks.tex} (\pstricksFV -- \pstricksFD)}
%--------------------------------------------------------------------------------------

The optional argument \Lkeyword{symbol} for a linestyle can now be
set with a negative \Lkeyword{symbolstep} for a computed width of the steps:
\begin{LTXexample}[pos=t]
\psset{unit=0.75cm}
\begin{pspicture}(8,8)
\psaxes[labels=none,ticks=none]{->}(0,0)(8,8)[$x$,0][$y$,0]
\pscurve(1,1)(3,4)(6,6)(8,4)
\pscurve[linestyle=symbol,symbolStep=-20,symbol=l](1,1)(3,4)(6,6)(8,4)
\end{pspicture}
\end{LTXexample}


Instead of setting a symbol it is possible to set ticks on the curve with \Lkeyword{curveticks}.  They can be set by a fixed width or a computed
width if the \Lkeyword{symbolStep} is set by a negative number without a unit.

\begin{LTXexample}[pos=t]
\psset{unit=0.75cm}
\begin{pspicture}(8,8)
\psaxes[labels=none,ticks=none]{->}(0,0)(8,8)[$x$,0][$y$,0]
\pscurve(1,1)(3,4)(6,6)(8,4)
\pscurve[linestyle=symbol,symbolStep=12.25pt,
  curveticks,startAngle=60](1,1)(3,4)(6,6)(8,4)
\end{pspicture}
\end{LTXexample}


\begin{LTXexample}[pos=t]
\psset{unit=0.75cm}
\begin{pspicture}(8,8)
\psaxes[labels=none,ticks=none]{->}(0,0)(8,8)[$x$,0][$y$,0]
\pscurve(1,1)(3,4)(6,6)(8,4)
\pscurve[linestyle=symbol,symbolStep=-12,
  curveticks,startAngle=60](1,1)(3,4)(6,6)(8,4)
\end{pspicture}
\end{LTXexample}




\section{The PostScript header files}
\subsection{\nxLFile{pstricks.pro}}
%--------------------------------------------------------------------------------------

The file \Lfile{pst-tools.pro} contains additional helper functions:

\begin{lstlisting}
/factorial { % n on stack, returns n! 
/MoverN { % m n on stack, returns the binomial coefficient m over n
/cxadd {		% [a1 b1] [a2 b2] = [a1+a2 b1+b2]
/cxneg {		% [a b]
/cxsub { cxneg cxadd } def  % same as negative addition
/cxmul {		% [a1 b1] [a2 b2]
/cxsqr {		% [a b]   square root
/cxsqrt {		% [a b]
/cxarg { 		% [a b] 
/cxlog {		% [a b]
/cxnorm2 {		% [a b]
/cxnorm {		% [a b]
/cxconj {		% conjugent complex
/cxre { 0 get } def	% real value
/cxim { 1 get } def	% imag value
/cxrecip {		% [a b]
/cxmake1 { 0 2 array astore } def % make a complex number, real given
/cxmake2 { 2 array astore } def	  % dito, both given
/cxdiv { cxrecip cxmul } def
/cxrmul {		% [a b] r
/cxrdiv {		% [a b] r
/cxconv {		% theta
/bubblesort { % on stack must be an array [ ... ]
/concatstringarray{  %  [(a) (b) ... (z)] --> (ab...z)  20100422
/dot2comma {% on stack a string (...) 
\end{lstlisting}


%--------------------------------------------------------------------------------------
%\subsection{\nxLFile{pst-algparser.pro}}
%--------------------------------------------------------------------------------------

%\section{\nxLcs{psforeach} and \nxLcs{psForeach}}
%--------------------------------------------------------------------------------------

\section{List of all optional arguments for \texttt{pstricks}}

\makeatletter
\xkvview{family=pstricks,columns={key,type,default}}
\makeatother


\clearpage
\part{Other packages}
\section{\texttt{pst-node} -- version \pstnodeFV\ | \pstnodeFD}
The command \Lcs{pnode} now knows an optional argument for an offset, which
expects two values (x,y) separated by a comma:

\begin{BDef}
\Lcs{pnode}\OptArg*{\Largs{\Larga{offset}}}\Largr{\CAny}\Largb{\Larga{node name}}
\end{BDef}


\begin{LTXexample}[width=6cm]
\begin{pspicture}[showgrid](0,-2)(6,2)
\pnode{A}\psdot(A)\uput[90](A){A} 
\pnode[0,-2]{B}\psdot(B)\uput[90](B){B} 
\pnode(2,0){C}\psdot(C)\uput[90](C){C} 
\pnode[1,-2](2,0){D}\psdot(D)\uput[90](D){D} 
\pnode[2,2](3.5,0){E}\psdot(E)\uput[90](E){E} 
\end{pspicture}
\end{LTXexample}


\clearpage
\nocite{*}
\bibliographystyle{plain}
\bibliography{PSTricks}

\printindex


\end{document}
