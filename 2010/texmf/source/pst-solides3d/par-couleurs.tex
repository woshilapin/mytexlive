\section {Les couleurs et les d�grad�s de couleur}

L'argument \texttt{[fillcolor=}\textsl{name}\texttt{]} permet de sp�cifier la couleur souhait�e
pour les faces externes d'un solide.
L'argument \texttt{[incolor=}\textsl{name}\texttt{]} permet de sp�cifier la couleur souhait�e
pour les faces internes d'un solide.

Les valeurs possibles pour \textsl{name} sont toutes celles reconnues
par PSTricks (et en particulier son package \texttt{xcolor}).

On peut �galement utiliser des d�grad�s de couleur dans les espaces
HSB, RGB ou CMYK. On utilise pour cela les options \verb+[hue]+,
\verb+[inhue]+ ou \verb+[inouthue]+ selon que l'on consid�re
respectivement les faces externes, les faces internes, ou l'ensemble
des faces.
Le nombre d'arguments de \verb+hue+ d�termine le cas de figure

\subsection {Couleurs pr\'{e}d\'{e}finies par l'option [\texttt{dvipsnames}]}

Il y a $68$~couleurs pr�d�finies, qui sont identifi�es dans le fichier
\textsl {solides.pro}~: \textsl {Black}, \textsl {White}, et les
$66$~couleurs ci-dessous.

\bgroup
\newcommand{\colorcube}[1]{%
\begin{pspicture}(-1.2,-1)(1.2,1)
\psframe(-1.2,-1)(1.2,1)
\psSolid[object=cube,
    linewidth=0.07\pslinewidth,
%    linecolor=#1!50,
    fillcolor=#1,
    ngrid=4,
    a=0.25,
    action=draw**](0,0,0.1)
\rput(0,-0.75){\footnotesize \texttt{#1}}
\end{pspicture}
}

\parindent0pt
%\parskip-8pt
\colorcube{GreenYellow}
\colorcube{Yellow}
\colorcube{Goldenrod}
\colorcube{Dandelion}
\colorcube{Apricot}
\colorcube{Peach}

\colorcube{Melon}
\colorcube{YellowOrange}
\colorcube{Orange}
\colorcube{BurntOrange}
\colorcube{Bittersweet}
\colorcube{RedOrange}

\colorcube{Mahogany}
\colorcube{Maroon}
\colorcube{BrickRed}
\colorcube{Red}
\colorcube{OrangeRed}
\colorcube{RubineRed}

\colorcube{WildStrawberry}
\colorcube{Salmon}
\colorcube{CarnationPink}
\colorcube{Magenta}
\colorcube{VioletRed}
\colorcube{Rhodamine}

\colorcube{Mulberry}
\colorcube{RedViolet}
\colorcube{Fuchsia}
\colorcube{Lavender}
\colorcube{Thistle}
\colorcube{Orchid}

\colorcube{DarkOrchid}
\colorcube{Purple}
\colorcube{Plum}
\colorcube{Violet}
\colorcube{RoyalPurple}
\colorcube{BlueViolet}

\colorcube{Periwinkle}
\colorcube{CadetBlue}
\colorcube{CornflowerBlue}
\colorcube{MidnightBlue}
\colorcube{NavyBlue}
\colorcube{RoyalBlue}

\colorcube{Blue}
\colorcube{Cerulean}
\colorcube{Cyan}
\colorcube{ProcessBlue}
\colorcube{SkyBlue}
\colorcube{Turquoise}

\colorcube{TealBlue}
\colorcube{Aquamarine}
\colorcube{BlueGreen}
\colorcube{Emerald}
\colorcube{JungleGreen}
\colorcube{SeaGreen}

\colorcube{Green}
\colorcube{ForestGreen}
\colorcube{PineGreen}
\colorcube{LimeGreen}
\colorcube{YellowGreen}
\colorcube{SpringGreen}

\colorcube{OliveGreen}
\colorcube{RawSienna}
\colorcube{Sepia}
\colorcube{Brown}
\colorcube{Tan}
\colorcube{Gray}

\egroup

\subsection {Couleurs pr\'{e}d\'{e}finies par l'option [\texttt{svgnames}]}

Les couleurs suivantes sont reconnues par pstricks si l'on utilise
l'option [\texttt{svgnames}].
Par contre, elles ne sont pas identifi�es dans le fichier
\textsl {solides.pro}~: on ne peut les utiliser directement dans
l'option [\texttt{fcol}].

\bgroup
\newcommand{\colorcone}[1]{%
\begin{pspicture}(-1.2,-1)(1.2,1)
\psframe(-1.2,-1)(1.2,1)
\psSolid[object=cone,
    linewidth=0.07\pslinewidth,
%    linecolor=#1!50,
    fillcolor=#1,
    ngrid=4 12,
    r=0.2,h=0.37,
    action=draw**](0,0,-0.05)
\rput(0,-0.75){\footnotesize \texttt{#1}}
\end{pspicture}
}


\parindent0pt
%\parskip-8pt

Ces couleurs sont propos\'{e}es par le package \texttt{xcolor}.
\bigskip

\colorcone{AliceBlue}
\colorcone{AntiqueWhite}
\colorcone{Aqua}
\colorcone{Aquamarine}
\colorcone{Azure}
\colorcone{Beige}

\colorcone{Bisque}
\colorcone{Black}
\colorcone{BlanchedAlmond}
\colorcone{Blue}
\colorcone{BlueViolet}
\colorcone{Brown}

\colorcone{BurlyWood}
\colorcone{CadetBlue}
\colorcone{Chartreuse}
\colorcone{Chocolate}
\colorcone{Coral}
\colorcone{CornflowerBlue}

\colorcone{Cornsilk}
\colorcone{Crimson}
\colorcone{Cyan}
\colorcone{DarkBlue}
\colorcone{DarkCyan}
\colorcone{DarkGoldenrod}

\colorcone{DarkGray}
\colorcone{DarkGreen}
\colorcone{DarkGrey}
\colorcone{DarkKhaki}
\colorcone{DarkMagenta}
\colorcone{DarkOliveGreen}

\colorcone{DarkOrange}
\colorcone{DarkOrchid}
\colorcone{DarkRed}
\colorcone{DarkSalmon}
\colorcone{DarkSeaGreen}
\colorcone{DarkSlateBlue}

\colorcone{DarkSlateGray}
\colorcone{DarkSlateGrey}
\colorcone{DarkTurquoise}
\colorcone{DarkViolet}
\colorcone{DeepPink}
\colorcone{DeepSkyBlue}

\colorcone{DimGray}
\colorcone{DimGrey}
\colorcone{DodgerBlue}
\colorcone{FireBrick}
\colorcone{FloralWhite}
\colorcone{ForestGreen}

\colorcone{Fuchsia}
\colorcone{Gainsboro}
\colorcone{GhostWhite}
\colorcone{Gold}
\colorcone{Goldenrod}
\colorcone{Gray}

\colorcone{Grey}
\colorcone{Green}
\colorcone{GreenYellow}
\colorcone{Honeydew}
\colorcone{HotPink}
\colorcone{IndianRed}

\colorcone{Indigo}
\colorcone{Ivory}
\colorcone{Khaki}
\colorcone{Lavender}
\colorcone{LavenderBlush}
\colorcone{LawnGreen}

\colorcone{LemonChiffon}
\colorcone{LightBlue}
\colorcone{LightCoral}
\colorcone{LightCyan}
\colorcone{LightGoldenrodYellow}
\colorcone{LightGray}

\colorcone{LightGreen}
\colorcone{LightGrey}
\colorcone{LightPink}
\colorcone{LightSalmon}
\colorcone{LightSeaGreen}
\colorcone{LightSkyBlue}

\colorcone{LightSlateGray}
\colorcone{LightSlateGrey}
\colorcone{LightSteelBlue}
\colorcone{LightYellow}
\colorcone{Lime}
\colorcone{LimeGreen}

\colorcone{Linen}
\colorcone{Magenta}
\colorcone{Maroon}
\colorcone{MediumAquamarine}
\colorcone{MediumBlue}
\colorcone{MediumOrchid}

\colorcone{MediumPurple}
\colorcone{MediumSeaGreen}
\colorcone{MediumSlateBlue}
\colorcone{MediumSpringGreen}
\colorcone{MediumTurquoise}
\colorcone{MediumVioletRed}

\colorcone{MidnightBlue}
\colorcone{MintCream}
\colorcone{MistyRose}
\colorcone{Moccasin}
\colorcone{NavajoWhite}
\colorcone{Navy}

\colorcone{OldLace}
\colorcone{Olive}
\colorcone{OliveDrab}
\colorcone{Orange}
\colorcone{OrangeRed}
\colorcone{Orchid}

\colorcone{PaleGoldenrod}
\colorcone{PaleGreen}
\colorcone{PaleTurquoise}
\colorcone{PaleVioletRed}
\colorcone{PapayaWhip}
\colorcone{PeachPuff}

\colorcone{Peru}
\colorcone{Pink}
\colorcone{Plum}
\colorcone{PowderBlue}
\colorcone{Purple}
\colorcone{Red}

\colorcone{RosyBrown}
\colorcone{RoyalBlue}
\colorcone{SaddleBrown}
\colorcone{Salmon}
\colorcone{SandyBrown}
\colorcone{SeaGreen}

\colorcone{Seashell}
\colorcone{Sienna}
\colorcone{Silver}
\colorcone{SkyBlue}
\colorcone{SlateBlue}
\colorcone{SlateGray}

\colorcone{SlateGrey}
\colorcone{Snow}
\colorcone{SpringGreen}
\colorcone{SteelBlue}
\colorcone{Tan}
\colorcone{Teal}

\colorcone{Thistle}
\colorcone{Tomato}
\colorcone{Turquoise}
\colorcone{Violet}
\colorcone{Wheat}
\colorcone{White}

\colorcone{WhiteSmoke}
\colorcone{Yellow}
\colorcone{YellowGreen}
\hfill
\egroup

\subsection {D�grad� dans l'espace HSB, saturation et brillance maximales}

Il y a 2 arguments~: \verb+[hue=+$h_0$ $h_1$\verb+]+ o�
les nombres $h_0$ et $h_1$ v�rifiant $0\leq h_0 < h_1 \leq 1$
indiquent les bornes du premier param�tre dans l'espace HSB.

\begin{multicols}{2}
\psset{unit=1}
\psset{SphericalCoor,viewpoint=50 50 20,Decran=30}
\begin{pspicture}(-4,-1.5)(3,1)
\psframe(-4,-1.5)(3,1)
\psSolid[object=grille,
   base=-3 5 -3 3,
   linecolor=gray,
   hue=0 1](0,0,0)
\end{pspicture}

\columnbreak

\begin{verbatim}
\psSolid[object=grille,
   base=-3 5 -3 3,
   linecolor=gray,
   hue=0 1](0,0,0)
\end{verbatim}
\end{multicols}

\begin{multicols}{2}
\psset{unit=1}
\psset{SphericalCoor,viewpoint=50 50 20,Decran=30}
\begin{pspicture}(-4,-1.5)(3,1)
\psframe(-4,-1.5)(3,1)
\psSolid[object=grille,
   base=-3 5 -3 3,
   linecolor=gray,
   hue=0 .3](0,0,0)
\end{pspicture}

\columnbreak

\begin{verbatim}
\psSolid[object=grille,
   base=-3 5 -3 3,
   linecolor=gray,
   hue=0 .3](0,0,0)
\end{verbatim}
\end{multicols}

\begin{multicols}{2}
\psset{unit=1}
\psset{SphericalCoor,viewpoint=50 50 20,Decran=30}
\begin{pspicture}(-4,-1.5)(3,1)
\psframe(-4,-1.5)(3,1)
\psSolid[object=grille,
   base=-3 5 -3 3,
   linecolor=gray,
   hue=.5 .6](0,0,0)
\end{pspicture}

\columnbreak

\begin{verbatim}
\psSolid[object=grille,
   base=-3 5 -3 3,
   linecolor=gray,
   hue=.5 .6](0,0,0)
\end{verbatim}
\end{multicols}

\subsection {D�grad� dans l'espace HSB, saturation et brillance fixes}

Il y a 4 arguments~: \verb+[hue=+$h_0$ $h_1$ $s$ $b$\verb+]+ o�
les nombres $h_0$ et $h_1$ v�rifiant $0\leq h_0 < h_1 \leq 1$
indiquent les bornes du premier param�tre dans l'espace HSB et o� $s$
et $b$ sont les param�tres respectifs \textsl {saturastion} et \textsl
{brillance}. 

\begin{multicols}{2}
\psset{unit=1}
\psset{SphericalCoor,viewpoint=50 50 20,Decran=30}
\begin{pspicture}(-4,-1.5)(3,1)
\psframe(-4,-1.5)(3,1)
\psSolid[object=grille,
   base=-3 5 -3 3,
   linecolor=gray,
   hue=0 1 .8 .7](0,0,0)
\end{pspicture}

\columnbreak

\begin{verbatim}
\psSolid[object=grille,
   base=-3 5 -3 3,
   linecolor=gray,
   hue=0 1 .8 .7](0,0,0)
\end{verbatim}
\end{multicols}


\begin{multicols}{2}
\psset{unit=1}
\psset{SphericalCoor,viewpoint=50 50 20,Decran=30}
\begin{pspicture}(-4,-1.5)(3,1)
\psframe(-4,-1.5)(3,1)
\psSolid[object=grille,
   base=-3 5 -3 3,
   linecolor=gray,
   hue=0 1 .5 1](0,0,0)
\end{pspicture}

\columnbreak

\begin{verbatim}
\psSolid[object=grille,
   base=-3 5 -3 3,
   linecolor=gray,
   hue=0 1 .5 1](0,0,0)
\end{verbatim}
\end{multicols}

\subsection {D�grad� dans l'espace HSB, cas g�n�ral}

Il y a 7 arguments~: \verb+[hue=+$h_0$ $s_0$ $b_0$ $h_1$ $s_1$
$b_1$\verb+ (hsb)]+ o� les nombres $h_i$, $s_i$ et $b_i$ indiquent les
bornes des param�tre HSB. 

\begin{multicols}{2}
\psset{unit=1}
\psset{SphericalCoor,viewpoint=50 50 20,Decran=30}
\begin{pspicture}(-4,-1.5)(3,1)
\psframe(-4,-1.5)(3,1)
\psSolid[object=grille,
   base=-3 5 -3 3,
   linecolor=gray,
   hue=0 .8 1 1 1 .7 (hsb)](0,0,0)
\end{pspicture}

\columnbreak

\begin{verbatim}
\psSolid[object=grille,
   base=-3 5 -3 3,
   linecolor=gray,
   hue=0 .8 1 1 1 .7 (hsb)](0,0,0)
\end{verbatim}
\end{multicols}

\subsection {D�grad� dans l'espace RGB}

Il y a 6 arguments~: \verb+[hue=+$r_0$ $g_0$ $b_0$ $r_1$ $g_1$
$b_1$\verb+]+ o� les nombres $r_i$, $g_i$ et $b_i$ indiquent les
bornes respectives des $3$ param�tres RGB.

\begin{multicols}{2}
\psset{unit=1}
\psset{SphericalCoor,viewpoint=50 50 20,Decran=30}
\begin{pspicture}(-4,-1.5)(3,1)
\psframe(-4,-1.5)(3,1)
\psSolid[object=grille,
   base=-3 5 -3 3,
   linecolor=gray,
   hue=1 0 0 0 0 1](0,0,0)
\end{pspicture}

\columnbreak

\begin{verbatim}
\psSolid[object=grille,
   base=-3 5 -3 3,
   linecolor=gray,
   hue=1 0 0 0 0 1](0,0,0)
\end{verbatim}
\end{multicols}


\subsection {D�grad� dans l'espace CMYK}

Il y a 8 arguments~: \verb+[hue=+$c_0$ $m_0$ $y_0$ $k_0$ $c_1$ $m_1$ 
$y_1$ $k_1$\verb+]+ o� les nombres $c_i$, $m_i$, $y_i$ et $k_i$ indiquent les
bornes respectives des $4$ param�tres CMYK.

\begin{multicols}{2}
\psset{unit=1}
\psset{SphericalCoor,viewpoint=50 50 20,Decran=30}
\begin{pspicture}(-4,-1.5)(3,1)
\psframe(-4,-1.5)(3,1)
\psSolid[object=grille,
   base=-3 5 -3 3,
   linecolor=gray,
   hue=1 0 0 0 0 0 1 0](0,0,0)
\end{pspicture}

\columnbreak

\begin{verbatim}
\psSolid[object=grille,
   base=-3 5 -3 3,
   linecolor=gray,
   hue=1 0 0 0 0 0 1 0](0,0,0)
\end{verbatim}
\end{multicols}

\subsection {D�grad� entre 2 couleurs nomm�es}

Il y a deux param�tres
\verb+[hue=(+\textsl{couleur1}\verb+) (+\textsl{couleur2}\verb+)]+ o�
\textsl{couleur1} et \textsl{couleur2} sont des noms de couleurs
connues dans \verb+solides.pro+.

\begin{multicols}{2}
\psset{unit=1}
\psset{SphericalCoor,viewpoint=50 50 20,Decran=30}
\begin{pspicture}(-4,-1.5)(3,1)
\psframe(-4,-1.5)(3,1)
\psSolid[object=grille,
   base=-3 5 -3 3,
   linecolor=gray,
   hue=(jaune) (CadetBlue)](0,0,0)
\end{pspicture}

\columnbreak

\begin{verbatim}
\psSolid[object=grille,
   base=-3 5 -3 3,
   linecolor=gray,
   hue=(jaune) (CadetBlue)](0,0,0)
\end{verbatim}
\end{multicols}

Si on veut utiliser des couleurs d�finies par \texttt{xcolor}, on
utilise les param�tres \texttt{color1}, \texttt{color2}, etc... de
\verb+\psSolid+.

\begin{multicols}{2}
\psset{unit=1}
\psset{SphericalCoor,viewpoint=50 50 20,Decran=30}
\begin{pspicture}(-4,-1.5)(3,1)
\psframe(-4,-1.5)(3,1)
\psSolid[object=grille,
   base=-3 5 -3 3,
   linecolor=gray,
   color1=red!50,
   color2=green!20,
   hue=(color1) (color2)](0,0,0)
\end{pspicture}

\columnbreak

\begin{verbatim}
\psSolid[object=grille,
   base=-3 5 -3 3,
   linecolor=gray,
   color1=red!50,
   color2=green!20,
   hue=(color1) (color2)](0,0,0)
\end{verbatim}
\end{multicols}

