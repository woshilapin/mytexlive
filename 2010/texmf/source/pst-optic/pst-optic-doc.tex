%% $Id: pst-func-doc.tex 64 2008-12-08 22:14:27Z herbert $
\documentclass[11pt,english,BCOR10mm,DIV13,bibliography=totoc,parskip=false,smallheadings
    headexclude,footexclude,oneside]{pst-doc}
\usepackage[utf8]{inputenc}
\usepackage{pst-text}
\usepackage{pst-optic}
\let\pstOpticFV\fileversion
\lstset{pos=t,wide=true,language=PSTricks,basicstyle=\footnotesize\ttfamily}
\let\belowcaptionskip\abovecaptionskip
%
\begin{document}

\title{\texttt{pst-optic}}
\subtitle{Lenses and Mirrors; v.\pstOpticFV}

\author{Manuel Luque \\Herbert Vo\ss}
\docauthor{Herbert Vo\ss}
\date{\today}
\maketitle

\tableofcontents

\clearpage

\begin{abstract}
\noindent
\LPack{pst-optic} loads by default the following packages: 
\LPack{pstricks}, \LPack{pst-node}, \LPack{pst-plot}, \LPack{pst-3d},
\LPack{pst-grad}, \LPack{pst-math}, \LPack{multido}, and \LPack{pst-xke}.
All should be already part of your local \TeX\ installation. If not, or in case
of having older versions, go to \url{http://www.CTAN.org/} and load the newest version.

\vfill\noindent
Thanks to: \\
Jean-C\^ome Charpentier, 
Arnaud Schmittbuhl.
\end{abstract}

\section{General Options}

All options are by default document wide valid but not supported by all macros. 
Table~\ref{tab:generalOptions} shows the general ones. Others are shown in 
Table~\ref{tab:defaultLens} and \ref{tab:defaultSpiegel}.

\begin{table}[!htb]
\centering
\caption{General options and the defaults}\label{tab:generalOptions}
\begin{tabular}{@{}l l l @{}}
	\emph{Option} & \emph{Name} & \emph{Default}\\\hline
	Left value of the picture in cm	 & \Lkeyword{xLeft}	& -7.5\\
	Right value of the picture in cm	& \Lkeyword{xRight}	& 7.5\\
	Lowest value of the picture in cm	& \Lkeyword{xBottom}	& -3\\
	Highest value of the picture in cm	& \Lkeyword{xTop}	& 3\\
	x-Offset 				& \Lkeyword{XO}	& 0\\
	y-Offset 				& \Lkeyword{YO}	& 0\\
	Node A as string		& \Lkeyword{nameA}	& A\\
	Angle A in degrees		& \Lkeyword{spotA}	& 270\\
	Node B as string		& \Lkeyword{nameB}	& B\\
	Angle B in degrees		& \Lkeyword{spotB}	& 270\\
	Node F as string		& \Lkeyword{nameF}	& F\\
	Angle F in degrees		& \Lkeyword{spotF}	& 270\\
	Node O as string		& \Lkeyword{nameO}	& O\\
	Angle O in degrees		& \Lkeyword{spotO}	& 225\\
	Node A' as string		& \Lkeyword{nameAi}	& A'\\
	Angle A' in degrees		& \Lkeyword{spotAi}	& 90\\
	Node B' as string		& \Lkeyword{nameBi}	& B'\\
	Angle B' in degrees		& \Lkeyword{spotBi}	& 270\\
	Node F' as string		& \Lkeyword{nameFi}	& B'\\
	Angle F' in degrees		& \Lkeyword{spotFi}	& 270\\
	Ray color			& \Lkeyword{rayColor} & black\\\hline
\end{tabular}
\end{table}

\Lcs{pst-optic} puts the lens and mirror macros in an own \Lenv{pspicture} environment. 
The star version enables the clipping option of \LPack{pstricks}:

\begin{lstlisting}[style=syntax]
\begin{pspicture}*(xLeft,yBottom)(xRight,yTop)
  \lens[focus=2,OA=-3,AB=1,XO=0,YO=0,xLeft=-7.5,xRight=7.5,yBottom=-3,yTop=3]
\end{pspicture}
\end{lstlisting}



If you need other values for the \Lenv{pspicture} environment, then use the \Lcs{rput} 
command to place the macro at any position.

\begin{LTXexample}
\begin{pspicture}[showgrid=true](-5,-2.2)(7,4)
\rput(1.5,1.5){%
  \lens[lensType=DVG,lensGlass=true,lensWidth=0.5,rayColor=red,
      focus=-2,AB=2,spotAi=270,spotBi=90]}
\end{pspicture}
\end{LTXexample}

\subsection{\nxLcs{resetOpticOptions}}
The Macro \Lcs{resetOpticOptions} resets all \LPack{pst-optic} options to the default value.

\subsection{Optical axis line style}
\LPack{pst-optic} definies a line style \Lnotation{opticalAxis} with the predefines values
of:
\begin{lstlisting}[style=syntax]
\newpsstyle{opticalAxis}{linewidth=0.5pt,linecolor=black,linestyle=solid}
\end{lstlisting}

It can be overwritten in the same way with \Lcs{newpsstyle}.

\begin{LTXexample}
\newpsstyle{opticalAxis}{linewidth=0.5pt,linecolor=blue,linestyle=dashed}
\lens
\end{LTXexample}

\newpsstyle{opticalAxis}{linewidth=0.5pt,linecolor=black,linestyle=solid}% reset the changes

\clearpage
\section{Lenses}

There are macros for the convergent and divergent lens 
\begin{BDef}
\Lcs{lens}\OptArgs\\
\Lcs{lensCVG}\OptArgs\% \textbf{C}on\textbf{v}er\textbf{g}ent (Collecting lens ) -- default\\
\Lcs{lensDVG}\OptArgs\% \textbf{D}i\textbf{v}er\textbf{g}ent (Scatter lens )
\end{BDef}

\subsection{The Coordinates of the predefined Nodes}

The following figure shows the coordinates of the predefined nodes (see Table~\ref{tab:generalOptions}).

\resizebox{\textwidth}{!}{%
\begin{pspicture*}(-8,-3.25)(8,3.25)
\rput(0,0){%
	\lens[drawing=false]
	\psline[linewidth=1pt](xLeft)(xRight)
	\qdisk(A){1.5pt}
	\qdisk(B){1.5pt}
	\qdisk(A'){1.5pt}\qdisk(B'){1.5pt}
	\qdisk(F){1.5pt}\qdisk(F'){1.5pt}
	\qdisk(O){1.5pt}\qdisk(I){1.5pt}
	\qdisk(I'){1.5pt}\qdisk(I1){1.5pt}
	\qdisk(I2){1.5pt}
	\uput[270](A){A}\uput[90](B){B}\uput[270](F){F}\uput[0](I){I}
	\uput[0](I'){$\mathrm{I'}$}\uput[270](F'){$\mathrm{F'}$}
	\uput[270](B'){$\mathrm{B'}$}\uput[90](A'){$\mathrm{A'}$}
	\uput[180](I1){I1}\uput[0](I2){I2}}
	\psgrid[subgriddiv=0,griddots=4,gridlabels=7pt]
\end{pspicture*}}
\captionof{figure}{Coodinates of the predefined Nodes}\label{fig:nodes}

\subsection{The Lens Type}

\begin{LTXexample}[wide=false]
\lens[lensType=CVG]
\end{LTXexample}
\vspace*{-2ex}
\captionof{figure}{Collecting lens}\label{fig:defaultCVG}

\begin{LTXexample}[wide=false]
\lens[lensType=PCVG]
\end{LTXexample}
\vspace*{-2ex}
\captionof{figure}{Plan Collecting lens}

\enlargethispage{2ex}

\begin{LTXexample}[wide=false]
\psscalebox{0.75}{\lens[lensType=DVG,focus=-2,spotAi=270,spotBi=90]}
\end{LTXexample}
\vspace*{-2ex}
\captionof{figure}{Scatter lens}\label{fig:defaultDVG}

\begin{LTXexample}[wide=false]
\lens[lensType=PDVG,focus=-2,spotAi=270,spotBi=90]
\end{LTXexample}
\vspace*{-2ex}
\captionof{figure}{Plan Scatter lens}

\clearpage

Using \Lcs{lens}\OptArg{\Lkeyword{lensType}=\ldots} gives the in figures \ref{fig:defaultCVG} and 
\ref{fig:defaultDVG} shown lenses with the default values from Table~\ref{tab:defaultLens}.




\begin{table}[!htb]
\centering
\begin{minipage}{\columnwidth}
\centering
\caption{Available options for lenses with the defaults}\label{tab:defaultLens}
\begin{tabular}{@{} l l l @{}}
	\emph{Option} & \emph{Name} & \emph{Default}\\\hline
	Lense type (\Lkeyval{CVG}|\Lkeyval{DVG}|\Lkeyval{PCVG}|\Lkeyval{PDVG}) & \Lkeyword{lensType}	& \Lkeyval{CVG}\\
	Lense height in cm		& \Lkeyword{lensHeight} & 5cm\\
	Lense width in cm 		& \Lkeyword{lensWidth}  & 0.5cm\footnotemark\\
	vertical scale (obsolet)	& \Lkeyword{lensScale}  & 1\\
	View the lens 			& \Lkeyword{lensGlass}  & \false\\
	Second lens			& \Lkeyword{lensTwo}	& \false\\
	Focus  in cm			& \Lkeyword{focus}	& 2\\
	Distance $\mathrm{\overline{OA}}$& \Lkeyword{OA}	& -4\\
	Distance $\mathrm{\overline{AB}}$& \Lkeyword{AB}	& 1.5\\
	Lens color			& \Lkeyword{lenscolor}  & \black\\
	Arrow length in cm		& \Lkeyword{lensarrowsize}  & 0.2\\
	Arrow inset in cm		& \Lkeyword{lensarrowinset} & 0.5\\
\end{tabular}
\footnotetext{\textsuperscript{1} only for \Lkeyset{lensGlass=true}, otherwise set to \texttt{2\textbackslash pslinewidth}}
\end{minipage}
\end{table}

The origin of the coordinate system is by default vertically and horinzontally symmetric. 
If you want to place the lens at another coordinates then define your own 
\Lenv{pspicture}-environment and use the \Lcs{rput}-command:

\begin{lstlisting}[style=syntax]
\begin{pspicture}(-7.5,-3)(7.5,3)
   \rput(x,y){\lens[...]}
\begin{pspicture}
\begin{pspicture*}(-7.5,-3)(7.5,3)
   \rput(x,y){\lens[...]}
\begin{pspicture*}
\end{lstlisting}

The star version enables the clipping option.



\subsection{\nxLcs{Transform}}

The \Lcs{Transform}-macro renames all existing nodes in names with an additional 
,,1``. Table~\ref{tab:Transform} shows a list of all nodes. 
\Lcs{Transform} also defines a new node \verb|factice|\index{factice@\texttt{factice}} with the coordinates  
\verb|(XO1,YO1)|. The renaming of all nodes makes it easier to handle objects with more than one lens. 
With the option \Lkeyset{lensTwo=true} it is possible to chain the different rays of the lenses 
(Figure~\ref{fig:lensTwoC}).

\begin{table}[!htb]
\addtolength{\tabcolsep}{-2pt}
\caption{Renaming of the nodes after calling the macro \nxLcs{Transform}}\label{tab:Transform}
\centering
{\ttfamily\begin{tabular}{@{}l|lllllllllllll @{}}
\emph{old} & A & B & A' & B' & O & F & F' & I & I' & XO & YO & OA' & A'B'\\\hline
\emph{new} &A1& B1& A'1& B'1& O1& F1& F'1& I1& I'1& XO1& YO1& O1A1'& A'1B'1
\end{tabular}}
\end{table} 


\begin{LTXexample}[wide=false]
\begin{pspicture*}(-7.5,-2.75)(7.5,3)
\rput(0,0){\lens[lensScale=0.6,XO=-4,focus=1,OA=-2,lensGlass=true, lensWidth=0.5]}
\rput(0,0){\lens[lensScale=1.2,XO=2,focus=2,lensGlass=true,lensWidth=0.5]}
\end{pspicture*}
\end{LTXexample}
\vspace*{-2ex}
\captionof{figure}{Definition of two unchained lenses}

\begin{LTXexample}[wide=false]
\begin{pspicture*}(-7.5,-2.75)(7.5,3)
\rput(0,0){\lens[lensScale=0.6,XO=-4,focus=1,OA=-2,lensGlass=, lensWidth=0.5]}
\Transform
\rput(0,0){\lens[lensScale=1.2,XO=2,focus=2,lensTwo=true,lensGlass=true,lensWidth=0.5]}
\end{pspicture*}
\end{LTXexample}
\vspace*{-2ex}
\captionof{figure}{Definition of two chained lenses}

\begin{LTXexample}[wide=false]
\begin{pspicture*}(-7.5,-2.75)(7.5,3)
\rput(0,0){\lens[lensScale=0.6,XO=-4,nameF=F_1,nameA=A_1,nameB=B_1,
  nameFi=F'_1,nameAi={ },nameBi={},nameO=O_1,focus=1,OA=-2,lensGlass=true, lensWidth=0.5]}
\pspolygon[style=rayuresJaunes,linestyle=none](B)(I)(B')(I')(B)
\Transform
\rput(0,0){\lens[lensScale=1.2,XO=2,focus=2,nameA=A'_1,spotA=90,nameB=B'_1,spotB=270,
  nameO=O_2,nameAi=A'_2,spotAi=270,nameBi=B'_2,spotBi=90,nameF=F_2,nameFi=F'_2,
  lensTwo=true,lensGlass=true,lensWidth=0.5]}
\pspolygon[style=rayuresJaunes,linestyle=none](B)(I)(B')(I')(B)
\end{pspicture*}
\end{LTXexample}
\vspace*{-2ex}
\captionof{figure}{Definition of two chained lenses and an additional modification of the node labels.}\label{fig:lensTwoC}

\clearpage

\subsection{\nxLcs{rayInterLens}}

This macro is only useful for a two-lens-system. Figure~\ref{fig:RayInter} shows such a system. 
The nodes \Lnotation{B1}, \Lnotation{I11}, \Lnotation{F'1}, and \Lnotation{B'1} are predefined 
by the \Lcs{lens}-macro. To draw the two rays 
from the left lense via the node \Lnotation{B'1} to the second lens, we need the coordinates of these 
points. \Lcs{rayInterLense} defines such nodes. The Syntax: 

\begin{BDef}
\Lcs{rayInterLense}\Largr{StartNode}\Largr{IntermediatNode}\Largr{LensDistance}\Largb{LensNode}
\end{BDef}

Two parallel lines are drawn with the \Lcs{Parallel}-Macro. 

\begin{LTXexample}
\begin{pspicture*}(-7.5,-4)(7.5,4)
\rput(0,0){\lens[focus=1.5,OA=-2,AB=0.6,XO=-5,lensGlass,yBottom=-4,yTop=4,drawing=false,
  lensWidth=0.5,lensScale=0.5,nameF=F_1,nameFi=F'_1]}
\psline[linewidth=1pt](xLeft)(xRight)
\Transform
\rput(0,0){\lens[focus=2,XO=4,lensGlass,yBottom=-4,yTop=4,drawing=false,lensWidth=0.5,
  lensHeight=7,nameF=F_2,nameFi=F'_2,spotF=90,spotFi=90]}
\psline{->}(A1)(B1)\psline{->}(A'1)(B'1)
\psset{linecolor=red}
\uput[45](B1){B1} \uput[90](O){O} \uput[225](O1){O1} \uput[45](I11){I11} \uput[45](B'1){$B'_1$}
\rayInterLens(I11)(B'1){4}{Inter1L2} \rayInterLens(O1)(B'1){4}{Inter2L2}
\uput[350](Inter1L2){Inter1L2}
\psline(B1)(I11)(B'1)(Inter1L2) \psline(B1)(O1)(B'1)(Inter2L2)
\Parallel(B'1)(O)(Inter2L2){B2inftyRigth} \Parallel(B'1)(O)(Inter1L2){B3inftyRigth}
\psset{length=-2,linestyle=dashed}
\Parallel(B'1)(O)(Inter2L2){B2inftyLeft} \Parallel(B'1)(O)(Inter1L2){B3inftyLeft}
\psline[linestyle=dotted, linewidth=2pt,linecolor=black]{->}(0,-4)(0,+4)
\end{pspicture*}
\end{LTXexample}
\captionof{figure}{Demonstration of \nxLcs{rayInterLens}\label{fig:RayInter}}





\subsection{\nxLcs{telescope}}

Figure~\ref{fig:defaultTele} shows the configuration of a telescope and Table~\ref{tab:defaultSpiegel} 
the special options for the \Lcs{telescop}-Macro.


\begin{LTXexample}[wide=false]
\telescope
\end{LTXexample}
\captionof{figure}{\nxLcs{telescope}-Macro}\label{fig:defaultTele}



\section{Mirrors}

\subsection{options}
Figure~\ref{fig:Spiegel} shows the available mirrors and Table~\ref{tab:defaultSpiegel} the possible options.




\begin{table}[!htb]
\centering
\caption{List of options for mirrors with the predefines values}\label{tab:defaultSpiegel}
\begin{tabularx}{\linewidth}{@{} >{\RaggedRight}X l l @{}}
\emph{Option} & \emph{Name} & \emph{Default}\\\hline
Left value of the picture in cm	& \Lkeyword{xLeft}	& -0.5\\
Right value of the picture in cm	& \Lkeyword{xRight}	& 11\\
Lowest value of the picture in cm	& \Lkeyword{xBottom}	& -6\\
Highest value of the picture in cm	& \Lkeyword{xTop}	& 2.5\\
Mirror height in cm	& \Lkeyword{mirrorHeight}	& 5\\
Mirror depth in cm	& \Lkeyword{mirrorDepth}	& 1\\
Mirror width in cm	& \Lkeyword{mirrorWidth}	& 0.25\\
Mirror color		& \Lkeyword{mirrorColor}	& lightgray\\
Ray color		& \Lkeyword{rayColor}		& black\\
Focus in cm 
(only together with the option \Lkeyword{posMirrorTwo} senseful)	& \Lkeyword{mirrorFocus}	& 8\\
Position of the 2. mirror in cm	& \Lkeyword{posMirrorTwo}	& 8\\
Inclination of the 2. mirror in degrees & \Lkeyword{mirrorTwoAngle}	& 45\\
Draw lines & \Lkeyword{drawing}	& true\\
\end{tabularx}
\end{table}


\begin{LTXexample}[width=4cm,pos=l]
\begin{pspicture*}[showgrid=true](-1,-3)(3,3)
\rput(0,0){\mirrorCVG[mirrorColor=gray,drawing=false]}
\psaxes[linestyle=dashed,linecolor=red,linewidth=1pt,arrows=->](0,0)(-1,-3)(3,3)
\qdisk(Focus){2pt}  \rput(Focus){\rput(0,0.25){Focus}}
\pcline[arrows=|-|](-0.75,-2.5)(-0.75,2.5)\ncput*[nrot=:U]{mirrorHeight}
\pcline[arrows=|-|](0,2.75)(1,2.75)       \rput[l](1.1,2.75){mirrorDepth}
\pcline[arrows=|-|](1,-2.75)(0.75,-2.75)  \rput[l](1.1,-2.75){mirrorWidth}
\rput[l](1,-1){mirrorColor}
\psline{<-}(0.2,-1)(0.8,-1)
\end{pspicture*}
\end{LTXexample}

\begin{LTXexample}[width=5cm,pos=l]
\begin{pspicture*}[showgrid=true](-4,-3)(1,3)
\rput(0,0){\mirrorDVG[mirrorColor=gray,drawing=false]}
\psaxes[linestyle=dashed,linecolor=red,linewidth=1pt,arrows=->](0,0)(-4,-3)(1,3)
\qdisk(Focus){2pt}                         \rput(Focus){\rput(0,0.25){Focus}}
\pcline[arrows=|-|](.5,-2.5)(.5,2.5)       \ncput*[nrot=:U]{mirrorHeight}
\pcline[arrows=|-|](-1.25,2.75)(-.25,2.75) \rput[r](-1.3,2.75){mirrorDepth}
\pcline[arrows=|-|](-1.25,-2.75)(-1,-2.75) \rput[r](-1.3,-2.75){mirrorWidth}
\rput[r](-2,-2){mirrorColor}               \psline{->}(-2,-2)(-0.9,-2)
\end{pspicture*}
\end{LTXexample}

\begin{LTXexample}[width=4cm,pos=l]
\begin{pspicture}[showgrid=true](-2,-2)(2,2)
\pnode(-1.5,-1.5){M1} \pnode(1,1){M2}
\uput[-90](M1){$\mathrm{M_1}$}\uput[90](M2){$\mathrm{M_2}$}
\pnode(-1.5,1.5){A}
\planMirrorRay(A)(M1)(M2){A'}
\psline[linewidth=5pt](M1)(M2)\pscircle*(A){2pt}
\uput[0](A){A}      \uput[0](A'){$\mathrm{A'}$}
\pscircle*(A'){2pt} \psline[linestyle=dashed](A)(A')
\end{pspicture}
\end{LTXexample}
\captionof{figure}{The different mirror macros: a) \nxLcs{mirrorCVG} b) \nxLcs{mirrorDVG} c) 
\nxLcs{planMirrorRay}}\label{fig:Spiegel}\label{fig:planMirror}

\subsection{\nxLcs{mirrorCVG}}

Figure~\ref{fig:mirrorCVG} shows the default for the \Lcs{mirrorCVG}-macro with the 
predefined nodes anf three default rays.


\begin{LTXexample}[width=7cm,pos=l]
\begin{pspicture*}[showgrid=true](-1,-3)(6,3)
  \rput(0,0){\mirrorCVG[rayColor=red]}
\end{pspicture*}
\end{LTXexample}
\captionof{figure}{Parabolic Mirror \nxLcs{mirrorCVG}}\label{fig:mirrorCVG}

\begin{LTXexample}[width=8.5cm,pos=l]
\begin{pspicture*}(-0.5,-4)(8,4)
  \rput(0,0){\mirrorCVG[mirrorHeight=8,mirrorDepth=4,drawing=false]}
  \multido{\rY=-4.00+0.25}{33}{%
    \mirrorCVGRay[linewidth=0.5pt,mirrorHeight=8,
      mirrorDepth=4](10,\rY)(1,\rY){Dummy}}
\end{pspicture*}
\end{LTXexample}
\captionof{figure}{Example}\label{fig:CVGAnwendung}


\clearpage

\section{\nxLcs{mirrorDVG}}

%Figure~\ref{fig:mirrorDVG} shows the defaults for the macro \Lcs{mirrorDVG}-Makros.


\begin{LTXexample}[width=7cm,pos=l]
\begin{pspicture*}[showgrid=true](-2,-2.6)(5,3)
  \rput(0,0){\mirrorDVG[rayColor=red]}
\end{pspicture*}
\end{LTXexample}
\vspace{-10mm}
\captionof{figure}{\nxLcs{mirrorDVG}}\label{fig:mirrorDVG}


\enlargethispage{13mm}

\begin{LTXexample}[wide]
\begin{pspicture*}[showgrid=true](-4,-4)(6,4)
\rput(0,0){\mirrorCVG[drawing=false,mirrorDepth=0.75,mirrorHeight=7]}
\qdisk(Focus){2pt}  \rput(6,2.5){\eye}
\pnode(1.75,-1.5){A}\pnode(1.75,1){B}\psline[arrows=->,linewidth=4pt](A)(B)
\uput{0.2}[0](2,0){\Huge G} \psset{linecolor=red}
\mirrorCVGRay[rayColor=red,mirrorHeight=7,mirrorDepth=0.75](A)(0,-0.9){P1}
\psOutLine[length=3](P1)(P1'){PEnd}\psBeforeLine[length=5,linestyle=dashed](P1)(P1'){PBegin}
\mirrorCVGRay[rayColor=red,mirrorHeight=7,mirrorDepth=0.75](A)(0,-0.2){P2}
\psOutLine[length=3](P2)(P2''){PEnd}\psBeforeLine[length=5,linestyle=dashed](P2)(P2'){PBegin}
%
\mirrorCVGRay[rayColor=red,mirrorHeight=7,mirrorDepth=0.75](B)(0,2.75){P3}
\psOutLine[length=3](P3)(P3'){PEnd}\psBeforeLine[length=5,linestyle=dashed](P3)(P3'){PBegin}
\mirrorCVGRay[rayColor=red,mirrorHeight=7,mirrorDepth=0.75](B)(0,1.8){P4}
\psOutLine[length=3](P4)(P4'){PEnd}\psBeforeLine[length=5,linestyle=dashed](P4)(P4'){PBegin}
\ABinterCD(P3)(P3')(P4)(P4'){A'}\ABinterCD(P1)(P1')(P2)(P2'){B'}
\psline[arrows=->,linewidth=4pt,linestyle=dashed](B')(A')
\nodeBetween(A')(B'){G''}\uput{0}[0](G''){\Huge G'}
\end{pspicture*}
\end{LTXexample}
\captionof{figure}{Example as a magnifier}\label{fig:DVGAnwendung}

\clearpage

\subsection{Drawing Rays in the Mirror Macros}

There are two different macros for drawing rays:




\begin{BDef}
\Lcs{mirrorCVGRay}\OptArgs\Largr{Node1}\Largr{Node2}\Largb{MirrorNode}\\
\Lcs{mirrorDVGRay}\OptArgs\Largr{Node1}\Largr{Node2}\Largr{MirrorNode}
\end{BDef}

The \Lnotation{MirrorNode} maybe :

\begin{tabularx}{\textwidth}{l|X}
\Lnotation{MirrorNode}   & first point on the mirror\\
\Lnotation{MirrorNode'}  & end node or second point on the mirror if one more reflection happens\\
\Lnotation{MirrorNode''} & end node for a second reflection
\end{tabularx}

If there are only one reflection, then \Lnotation{MirrorNode'} and \Lnotation{MirrorNode''} are the same.

\subsection{\nxLcs{planMirrorRay}}

The \Lcs{planMirrorRay}-Macro caculates the coordinates of a mirrored point. In 
Figure~\ref{fig:planMirror} is a given node \verb|A|, whereas \verb|A'| is calculated by the macro. The syntax is:

\begin{BDef}
\Lcs{planMirrorRay}\Largr{Mirrorbegin}\Largr{Mirrorend}\Largr{Originalpoint}\Largb{New point}
\end{BDef}

The macro doesn't draw any lines, only the coordinates of the new point are saved by the new node name.

\subsection{\nxLcs{symPlan}}

\verb|\symPlan| allows to mirroring complete plain graphical objects along a virtual center line. Figure \ref{fig:symPlan} shows that this mirroring is a mathematical one and not a physical one. For more examples look at  \cite{pstoptic:2001}. The syntax is:

\begin{BDef}
\Lcs{symPlan}\Largr{node1}\Largr{node2}\Largb{graphic object}
\end{BDef}

The two nodes define the mirror axis and the graphics object is in most cases a user defined macro, f.ex:
This example needs the package \LPack{pst-text} for the \Lcs{pstextpath} macro.

\begin{center}
\begin{LTXexample}[wide=false]
\newcommand{\dtk}{\pstextpath(0,0){%
  \psplot[linestyle=none]{0}{8}{x sqrt sqrt 2 mul}}%
    {\Large Die \TeX{}nische Kom\"odie von DANTE}}
\begin{pspicture}(-4.5,-2.5)(2.5,5)
\pnode(-4,-2){M1}  \uput[-90](M1){M1}
\pnode(4,4){M2}\uput[90](M2){M2}
\psline[linewidth=5\pslinewidth,linecolor=lightgray](M1)(M2)
\rput(-3.5,-1.75){\dtk}
\symPlan(M1)(M2){\rput(-3.5,-1.75){\dtk}}
\end{pspicture}
\end{LTXexample}
\captionof{figure}{Demonstration of the \nxLcs{symPlan}-Macro} \label{fig:symPlan}
\end{center}

\clearpage

\subsection{Beam Light}

This macro is useful for the demonstration of high and low beam light. The syntax for this macro is:

\begin{BDef}
\Lcs{beamLight}\OptArgs
\end{BDef}

The predefined options especially for the \Lenv{pspicture}-coordinates are
\begin{lstlisting}[style=syntax]
\psset[pst-optic]{xLeft=-5,xRight=5,yBottom=-5,yTop=5,drawing=false}% the default
\end{lstlisting}

You can place this macro with the \Lcs{rput}-command at any place in your own \Lenv{pspicture}-environment.

\begin{minipage}[b]{0.35\textwidth}
\begin{LTXexample}
\begin{pspicture}(-1,-3)(3,3)
  \rput(0,0){\beamLight}
\end{pspicture}
\end{LTXexample}
\captionof{figure}{\nxLcs{beamLight} without any Options}
\end{minipage}\hfill%
\begin{minipage}[b]{0.5\textwidth}
\centering
\begin{LTXexample}
\begin{pspicture}(-1,-5.5)(5,5.5)
\rput(0,0){\beamLight[mirrorDepth=4.75,
  mirrorWidth=0.1,mirrorHeight=10,
  linecolor=lightgray]}
\end{pspicture}
\end{LTXexample}
\captionof{figure}{\nxLcs{beamLight} with Options}
\end{minipage}


\clearpage

\section{Refraction}

\section{\nxLcs{refractionRay}}
The syntax is

\begin{BDef}
\Lcs{refractionRay}\Largr{A}\Largr{B}\Largr{C}\Largr{D}\Largb{n1}\Largb{n2}\Largb{EndNode}
\end{BDef}


The macro uses the law of Snell

\begin{align}
\frac{n_1}{n_2}&=\frac{\sin \beta}{\sin \alpha}\label{snell}
\end{align}

where the $n_1$ and $n_2$ are the refraction numbers with the predefined values 

\begin{align}
	n_1 &= 1 \\
	n_2 &= 1.41 
\end{align}

and $\alpha$ the incoming abd $\beta$ the outgoing angle of the ray.

The refractionnumbers have the internal names \Lnotation{refractA} and \Lnotation{refractB}.

A total reflection instead of a refraction is possible, 
when the ray starts in a medium with a higher refrectionnumber. This happens when 
$\sin\beta>1$ in Equ.~\ref{snell}. In this case we have $\alpha =\beta$, a 
total reflection.


\begin{LTXexample}[width=5cm,pos=l,wide=false]
\begin{pspicture}[showgrid=true](5,5)
\pnode(0.5,1.5){A}\qdisk(A){2pt}\uput[90](A){A}
\pnode(2,3){B}\qdisk(B){2pt}\uput[90](B){B}
\psline[linewidth=1.5pt,linecolor=red]{->}(A)(B)
\pnode(3,5){C}\qdisk(C){2pt}\uput[225](C){C}
\pnode(4,0){D}\qdisk(D){2pt}\uput[135](D){D}
\pspolygon[fillstyle=solid,fillcolor=lightgray,
  linecolor=lightgray](C)(D)(5,0)(5,5)(C)
\psline[linewidth=1.5pt,linecolor=red](C)(D)
\rput(2,2){\Large$n_1$} \rput(4.5,2){\Large$n_2$}
\refractionRay(A)(B)(C)(D){1}{4}{E}
\psline[linestyle=dashed,linecolor=red](B)(E)
\psline[linestyle=dashed,linecolor=red](E)(E')
\qdisk(E){2pt}\uput[-20](E){E}
\qdisk(E'){2pt}\uput[-90](E'){E'}
\end{pspicture}
\end{LTXexample}

The macro needs the values for the four nodes, the two refractionnumbers and the name for 
the end node. As you can see in the figure the end node of the ray is the intermediate 
point between the linear ray and the linear medium. The end node of the refracted ray has 
the same name with an additional single quotation mark. In the figure the macro was called as 

\begin{BDef}
\Lcs{refractionRay}\Largr{A}\Largr{B}\Largr{C}\Largr{D}\Largb{1}\Largb{4}\Largb{E}
\end{BDef}

\begin{align}n_1&<n_2\end{align}


It is no problem to draw a ray which is going straight through another medium. 
It can be done by using the macro twice as shown in the following examples.


\subsection{Total Reflection}


\begin{LTXexample}[width=5cm,pos=l,wide=false]
\begin{pspicture}[showgrid=true](5,5)
\pnode(0.5,1.5){A}\pnode(2,3){B}
\pnode(3,5){C}    \pnode(4,0){D}
\pspolygon[fillstyle=solid,fillcolor=lightgray,
  linecolor=lightgray](C)(D)(0,0)(0,5)(C)
\qdisk(A){2pt}\uput[90](A){A}
\qdisk(B){2pt}\uput[90](B){B}
\qdisk(C){2pt}\uput[225](C){C}
\qdisk(D){2pt}\uput[135](D){D}
\psline[linewidth=1.5pt,linecolor=red]{->}(A)(B)
\psline[linewidth=1.5pt,linecolor=red](C)(D)
\rput(2,2){\Large$n_1$}\rput(4.5,2){\Large$n_2$}
\refractionRay(A)(B)(C)(D){4}{1}{E}
\psline[linestyle=dashed,linecolor=red](B)(E)
\psline[linestyle=dashed,linecolor=red](E)(E')
\qdisk(E){2pt}\uput[-20](E){E}
\qdisk(E'){2pt}\uput[-90](E'){E'}
\end{pspicture}
\end{LTXexample}


In the figure the macro was called as 

\begin{lstlisting}[style=syntax]
\refractionRay(A)(B)(C)(D){4}{1}{E}
\end{lstlisting}

\begin{align}n_1&>n_2\end{align}



\section{Prism}
This command allows to simulate the deviation of a mono-chromatic
light ray with a prism. There are only few parameters. The indicated
values are the default ones.

\smallskip\noindent
\begin{tabularx}{\linewidth}{@{} l X l @{}}
\emph{name} & \emph{meaning} & \emph{default} \\\hline
\Lkeyword{AnglePrism} & The angle to the top of prism. & 60\\
\Lkeyword{AnglePlan1}  & The angle between the plane (1), where the tranmitter takes
   place, and the vertical. Negative values are allowed. & 25 \\
\Lkeyword{AnglePlan2} & The angle between the plane (2) (the screen), and the vertical. 
  Negative values are allowed. & 55\\
\Lkeyword{k} & Position of transmitter relative to choosen origin $C_1$ on the
   plane : $\overrightarrow{C_1E_1} = k\overrightarrow{u_1}$. & 1 \\
\Lkeyword{lambda} & The wavelength , in nm. & 632.8 \\
\Lkeyword{notations} & The plane where transmitting source takes place, with all
   indications, origin, angle, etc., as well as the screen are
   displayed by default. This can be useful in order to finalize a
   figure, but it is possible to deactivate this feature with the
   option. & \false\\\hline
\end{tabularx}

\bigskip
With \Lkeyword{AnglePlan1} the incident ray direction can be changed. The incidence spot changes according to \Lkeyword{k}.

The outline of processing we have adopted is the Gernot Hoffmann
one. For more details look into the document:
\url{http://www.fho-emden.de/~hoffmann/prism16072005.pdf}

\subsection{Figure with default values and construction indications}


\begin{LTXexample}[wide=false]
\begin{pspicture*}(-8,0)(8,8)
\psprism
\end{pspicture*}
\end{LTXexample}


\subsection{Figure with default values, without construction indications}

\begin{LTXexample}[wide=false]
\begin{pspicture*}(-8,0)(8,6)
\psprism[notations=false]
\end{pspicture*}
\end{LTXexample}

\subsection{Color matches wavelength}

\begin{LTXexample}[wide=false]
\begin{pspicture*}(-8,0)(8,8)
\psprism[lambda=530]%
\end{pspicture*}
\end{LTXexample}



Note: we have not planned physical impossibilities. When $r_2$ is greater
than the limit angle, there is no transmission in air, and it's
impossible to calculate $i_2$. Then, we have a PostScript message:




\begin{lstlisting}
Displaying page 1
Displaying page 2
Displaying page 3
Displaying page 4
Error: /rangecheck in --sqrt--
Operand stack:
   alpha2   -1.02701   -0.0547467
\end{lstlisting}
We remind you that \Lkeyword{alpha2} is $i_2$.

For instance, \Lkeyword{AnglePrism}=65, other default parameters remains unchanged.

\begin{LTXexample}[wide=false]
\begin{pspicture*}(-7,-0.2)(7,8)
\psprism[AnglePrism=65]
\end{pspicture*}
\end{LTXexample}

It will become right when we change the incident ray slope:

\begin{LTXexample}[wide=false]
\begin{pspicture*}(-8,-0.2)(8,8)
\psprism[AnglePrism=65,AnglePlan1=51,k=-1.5]
\end{pspicture*}
\end{LTXexample}

We choose \Lkeyword{k}=-1.5 in order to have a incident ray
which strikes (?) the input side roudly in its center. But, in these
particular cases, the physicist know-how is important (\emph{bis repetita}). Isn't it?



\section{Spherical Optic}


\subsection{\nxLcs{lensSPH}}
The syntax is

\begin{BDef}
\Lcs{lensSPH}\OptArgs
\end{BDef}


It changes some default values for the options to:


\begin{center}
\begin{tabular}{@{} lll @{}}
\textbf{meaning} & \textbf{name} & \textbf{default}\\\hline
Object Distance in cm   & \Lkeyword{OA} & -7 \\
Lens Height in cm       & \Lkeyword{lensHeight} & 6  \\
Lens Width in cm        & \Lkeyword{lensWidth}  & 1.5\\
Refraction Number $n_2$ & \Lkeyword{refractB}   & 2
\end{tabular} 
\end{center}



\subsubsection{Convergent Lens}
Without any option it draws a spherical convergent lens. \Lcs{lensSPH} is equivilant to 
\Lcs{lensSPH}\Largs{\Lkeyset{lensType=CVG}}.

\begin{LTXexample}[wide=false]
\lensSPH
\end{LTXexample}

\subsubsection{Divergent Lens}

The syntax is

\begin{BDef}
\Lcs{lensSPH}\OptArg{\Lkeyset{lensType=DVG},\ldots}
\end{BDef}

It draws a spherical divergent lens:

\begin{LTXexample}[wide=false]
\lensSPH[lensType=DVG,lensWidth=0.5]
\end{LTXexample}

It changes some default values for the options in the same way as for the convergent lens.


\subsection{Options}

The macro uses the law of Snell

\begin{align}
\frac{n_1}{n_2}&=\frac{\sin \beta}{\sin \alpha}
\end{align}

where the $n_1$ and $n_2$ are the refraction numbers with the predefined values 

\begin{align}
	n_1 &= 1 \\
	n_2 &= 1.41 
\end{align}

and $\alpha$ the incoming abd $\beta$ the outgoing angle of the ray.

The refractionnumbers have the internal names \Lnotation{refractA} and \Lnotation{refractB}.

\clearpage

\section{\nxLcs{mirrorCVG}}

The syntax is 

\begin{BDef}
\Lcs{mirrorCVG}\OptArg{\Lkeyset{mirrorType=SPH}} 
\end{BDef}



\resetOpticOptions
\begin{LTXexample}[wide=false]
\mirrorCVG[mirrorType=SPH]
\end{LTXexample}

Without the option \Lkeyset{mirrorType=SPH} you'll get a parabolic mirror, which is the default.

\section{\nxLcs{mirrorDVG}}

The syntax is 

\begin{BDef}
\Lcs{mirrorDVG}\OptArg{\Lkeyset{mirrorType=SPH}} 
\end{BDef}


\begin{LTXexample}[wide=false]
\mirrorDVG[mirrorType=SPH]
\end{LTXexample}

Without the option \Lkeyset{mirrorType=SPH} you'll get a parabolic mirror (option PARA).

\section{\nxLcs{ABinterSPHLens}}

The syntax is 



\begin{LTXexample}[wide=false]
\begin{pspicture}[showgrid=true](-6,-3)(6,3)
\rput(0,0){\lensSPH[lensType=CVG,lensHeight=8,lensWidth=3,drawing=false]}
\qdisk(Center){2pt}\uput[-90](Center){Center}
\qdisk(Center'){2pt}\uput[-90](Center'){Center'}
\pnode(-5,1){A}\qdisk(A){2pt}\uput[90](A){A}
\pnode(-3,0){B}\qdisk(B){2pt}\uput[90](B){B}
\psline[linewidth=1.5pt,linecolor=red]{->}(A)(B)
\pnode(5,-2.5){C}\qdisk(C){2pt}\uput[90](C){C}
\pnode(3,-2){D}\qdisk(D){2pt}\uput[90](D){D}
\psline[linewidth=1.5pt,linecolor=red]{->}(C)(D)
\rput(-4.5,2){\Large $n_1$}\rput(0,2){\Large $n_2$}\rput(4.5,2){\Large $n_1$}
\ABinterSPHLens(A)(B)(Center'){E}\ABinterSPHLens(C)(D)(Center){F}
\psline[linestyle=dashed,linecolor=red](B)(E)
\psline[linestyle=dashed,linecolor=red](D)(F)
\qdisk(E){2pt}\uput[-20](E){E}\qdisk(F){2pt}\uput[-90](F){F}
\end{pspicture}
\end{LTXexample}
The macro needs two nodes for the rays, the coordinates/nodes of the center/middle of the 
sperical lens and a name of the intermediate node.In the figure the macro was called as 


\begin{lstlisting}[style=syntax]
\ABinterSPHLens(A)(B)(Center'){E}
\ABinterSPHLens(C)(D)(Center){F}
\end{lstlisting}



\section{\nxLcs{lensSPHRay}}

The syntax is 

\begin{BDef}
\Lcs{lensSPHRay}\OptArgs\Largr{A}\Largr{B}\Largb{refractA}\Largb{refractB}\Largb{NodeName} 
\end{BDef}

This macro calculates the coordinates of the given ray $\overline{\mathrm{AB}}$ on its way 
into the lens. The only possible option \Lkeyword{rightRay}=\false|\true\footnote{Default is \texttt{false}} 
enables rays from the right to the left. There are still some problems with this option but try it out.

\begin{LTXexample}
\begin{pspicture*}[showgrid=true](-5,-3)(7,3)
\rput(0,0){\lensSPH[lensType=CVG,lensHeight=8,lensWidth=3,drawing=false]}
\qdisk(Center){2pt}\uput[-90](Center){Center}
\qdisk(Center'){2pt}\uput[-90](Center'){Center'}
\pnode(-4,1){A}\qdisk(A){2pt}\uput[90](A){A}
\pnode(-2,0.5){B}\qdisk(B){2pt}\uput[90](B){B}
\rput(-4.5,2.5){\Large $n_1$}\rput(0,2.5){\Large $n_2=1.5$}\rput(4,2.5){\Large $n_1=1$}
\multido{\rA=3+-0.5,\rB=2.5+-0.5}{9}{%
  \lensSPHRay[rightRay=false](-4,\rA)(-2,\rB){1}{1.5}{F}
  \psline[linewidth=1.5pt,linecolor=red]{->}(-4,\rA)(F)
  \psline[linestyle=dashed,linecolor=red](-4,\rA)(F)(F')(F'')
  \psOutLine[linestyle=dashed,linecolor=red,length=4.5](F')(F''){FEnd}}
\psline[linewidth=1.5pt,linecolor=blue]{->}(A)(B)
\lensSPHRay[lensType=CVG](A)(B){1}{1.5}{E}
\psline[linestyle=dashed,linecolor=blue](B)(E)(E')(E'')
\qdisk(E){2pt}\uput[-20](E){E}\qdisk(E'){2pt}\uput[-20](E'){E'}
\qdisk(E''){2pt}\uput[-20](E''){E''}
\end{pspicture*}
\end{LTXexample}

And the same with $n_2=3$:
\begin{LTXexample}
\begin{pspicture*}[showgrid=true](-5,-3)(7,3)
\rput(0,0){\lensSPH[lensType=CVG,lensHeight=8,lensWidth=3,drawing=false]}
\qdisk(Center){2pt}\uput[-90](Center){Center}
\qdisk(Center'){2pt}\uput[-90](Center'){Center'}
\pnode(-4,1){A}\qdisk(A){2pt}\uput[90](A){A}
\pnode(-2,0.5){B}\qdisk(B){2pt}\uput[90](B){B}
\rput(-4.5,2.5){\Large $n_1$}\rput(0,2.5){\Large $n_2=3$}\rput(4,2.5){\Large $n_1=1$}
\multido{\rA=3+-0.5,\rB=2.5+-0.5}{11}{%
  \lensSPHRay[rightRay=false](-4,\rA)(-2,\rB){1}{3}{F}
  \psline[linewidth=1.5pt,linecolor=red]{->}(-4,\rA)(F)
  \psline[linestyle=dashed,linecolor=red](-4,\rA)(F)(F')(F'')
  \psOutLine[linestyle=dashed,linecolor=red](F')(F''){FEnd}}
\psline[linewidth=1.5pt,linecolor=blue]{->}(A)(B)
\lensSPHRay[lensType=CVG](A)(B){1}{3}{E}
\psline[linestyle=dashed,linecolor=blue](B)(E)(E')(E'')
\qdisk(E){2pt}\uput[-20](E){E}\qdisk(E'){2pt}\uput[-20](E'){E'}
\qdisk(E''){2pt}\uput[-20](E''){E''}
\end{pspicture*}
\end{LTXexample}


\clearpage

\section{\nxLcs{reflectionRay}}

The syntax is 

\begin{BDef}
\Lcs{reflectionRay}\OptArgs\Largr{A}\Largr{B}\Largb{NodeName}
\end{BDef}

This macro calculates the coordinates of the given ray $\overline{\mathrm{AB}}$ 
on its way out of the mirror. The only senseful option is \Lkeyset{mirrorType=CVG} or \Lkeyset{mirrorType=DVG}. 
The most important fact is that the point \verb|B| must be the one on the mirror. 
If you do not know it's coordinates you can use the macro 
\verb|ABinterSPHLens[lensType=CVG](A1)(A2)(Center){NodeName)|, which calculates the coordinates of the intermediate point.



\begin{LTXexample}
\begin{pspicture*}[showgrid=true](-1,-3)(6,3)
\rput(0,0){%
  \mirrorCVG[mirrorType=SPH,mirrorHeight=5,mirrorWidth=0.2,yBottom=-3,yTop=3,drawing=false,mirrorDepth=3]
  \qdisk(Center){2pt}\qdisk(Focus){2pt}\uput[-90](Center){Center}\uput[-90](Focus){F}
  \psline(O)(xRight)}
\ABinterSPHLens(5,1)(3,1)(Center){C}
\reflectionRay[mirrorType=CVG-SPH](5,1)(C){D}
\qdisk(5,1){2pt}\uput[-90](5,1){A}\qdisk(3,1){2pt}\uput[-90](3,1){B}
\qdisk(C){2pt}\uput[180](C){C}\qdisk(D){2pt}\uput[45](D){D}
\psset{linewidth=1.5pt,linecolor=red,arrows=->}
\psline(5,1)(3,1)\psline(3,1)(C)\psline(C)(D)
\end{pspicture*}
\end{LTXexample}



\clearpage
\subsection{Refraction at a Spherical surface}

Construction for finding the position of the image point P' of a point object P formed by refraction at a sperical surface.


\begin{LTXexample}
\begin{pspicture*}[showgrid=true](-10,-4)(3,4)
\rput(0,0){%
  \lensSPH[lensType=CVG,lensHeight=12,lensWidth=10,yBottom=-6,yTop=6,xLeft=-6,xRight=6,drawing=false]}
\psset{linecolor=red,linewidth=1.5pt,dotstyle=|}
\pnode(-9,0){P}\psdots(P)\uput[-90](P){P}\psline(P)(xRight)
\lensSPHRay(P)(-5,2){1}{9}{Q}  \psline(P)(Q)(Q') \psdots(Q)\uput[90](Q){B}
\ABinterCD(Q)(Q')(0,0)(5,0){P'}\psdots(Q')\uput[-90](P'){P'}
\psline[linewidth=0.5pt,linecolor=black](Center')(Q)\psline[linewidth=0.5pt,linecolor=black](Q)(Q|0,0)
\psdots(Center')\uput[-90](Center'){C}
\end{pspicture*}
\end{LTXexample}


\clearpage
Construction for determining the height of an image formed by refraction at a sperical surface.

\begin{LTXexample}
\begin{pspicture*}[showgrid=true](-13,-3)(3,5)
\rput(0,0){%
  \lensSPH[lensType=CVG,lensHeight=12,lensWidth=10,yBottom=-4,yTop=4,xLeft=-5,xRight=5,drawing=false]}
\psset{linecolor=red,linewidth=1.5pt,dotstyle=|}
\pnode(-12,0){P}\psdots(P)\uput[-90](P){P}\pnode(-12,4){Q}\psdots(Q)\uput[90](Q){Q}
\psline[linecolor=blue,linewidth=3pt,arrows=->](P)(Q)\psline(P)(xRight)
\lensSPHRay(Q)(Center'){1}{9}{S1}\lensSPHRay(Q)(-5,0){1}{9}{S2}%
\psline(Q)(S1')\psline(Q)(S2)(S2')\ABinterCD(Q)(S1')(S2)(S2'){Q'}\pnode(Q'|0,0){P'}
\psline[linecolor=blue,linewidth=3pt,arrows=->](P')(Q')
\uput[90](P'){P'}\uput[-90](Q'){Q'}\psdots(Center')\uput[90](Center'){C}
\end{pspicture*}
\end{LTXexample}


\section{Utility Macros}

\subsection{\nxLcs{eye}}

Syntax:

\begin{BDef}
\Lcs{eye}
\end{BDef}
There are no Options for this symbol of an human eye (Figure~\ref{fig:auge}).
Use the \Lcs{rput}-macro to put the eye elsewhere.

\begin{LTXexample}[width=2cm,pos=l,wide=false]
\begin{pspicture}(-1,-0.75)(1,0.75)
  \rput(1,0){\eye}
\end{pspicture}
\end{LTXexample}

\section{\nxLcs{Arrows}}

Syntax wirh the following options:

\begin{BDef}
\Lcs{Arrows}\OptArgs\Largr{NodeA}\Largr{NodeB}
\end{BDef}

\begin{center}
\begin{tabular}{@{} l l l @{}}
\emph{Option} & \emph{Name} & \emph{Standard}\\\hline
Offset for arrow start in cm     & \Lkeyword{posStart} & 0\\
Length of the arrow in cm        & \Lkeyword{length}   & 2\\\hline
\end{tabular}
\end{center}




\begin{LTXexample}[width=6cm,pos=l]
\begin{pspicture}[showgrid=true](-3,-3)(3,3)
\psset{linecolor=red}
\Arrows[posStart=2,length=4](-3,-3)(3,3)
\qdisk(-3,-3){2pt}\uput[45](-3,-3){A}
\qdisk(3,3){2pt}\uput[-90](3,3){B}
\psset{linecolor=green}
\Arrows[linewidth=3pt,length=2](0,-3)(0,0.5)
\qdisk(0,-3){2pt}\uput[45](0,-3){A}
\qdisk(0,0.5){2pt}\uput[-90](0,0.5){B}
\psset{linecolor=blue}
\Arrows[linewidth=5pt,linestyle=dashed](3,0)(2,3)
\qdisk(3,0){2pt}\uput[-90](3,0){A}
\qdisk(2,3){2pt}\uput[-90](2,3){B}
\psset{linecolor=black}
\Arrows[posStart=1,linewidth=5pt,linestyle=dotted,
  length=3](-3,2)(1,2)
\qdisk(-3,2){2pt}\uput[-45](-3,2){A}
\qdisk(1,2){2pt}\uput[-90](1,2){B}
\end{pspicture}
\end{LTXexample}



\section{\nxLcs{psOutLine} and \nxLcs{psBeforeLine}}

Syntax:

\begin{BDef}
\Lcs{psOutLine}\OptArgs\Largr{NodeA}\Largr{NodeB}\Largb{EndNode}\\
\Lcs{psBeforeLine}\OptArgs\Largr{NodeA}\Largr{NodeB}\Largb{StartNode}
\end{BDef}

The only special option is \Lkeyword{length}=\Larga{value}. All other which are possible for
\Lcs{psline} can be used, too.

\begin{LTXexample}[width=6cm,pos=l]
\begin{pspicture}[showgrid=true](-3,-3)(3,3)
\psset{linecolor=red, linewidth=1.5pt}
\psOutLine[length=3](-2,-2)(0,0){End}
\qdisk(-2,-2){2pt}\uput[80](-2,-2){A}
\qdisk(0,0){2pt}\uput[-45](0,0){B}
\qdisk(2.12,2.12){2pt}\uput[90](2.12,2.12){END}
%
\psset{linecolor=blue}
\psBeforeLine[length=3](1,0)(3,2){START}
\qdisk(3,2){2pt}\uput[80](3,2){B}
\qdisk(1,0){2pt}\uput[-45](1,0){A}
\qdisk(-1.12,-2.12){2pt}\uput[-90](-1.12,-2.12){START}
\end{pspicture}
\end{LTXexample}

\section{\nxLcs{Parallel}}

Syntax:

\begin{BDef}
\Lcs{Parallel}\OptArgs\Largr{NodeA}\Largr{NodeB}\Largr{StartNode}\Largb{End node}
\end{BDef}

The only special option for \Lcs{Parallel} is \Lkeyword{length}=\Larga{value}. 
The nodes \verb|nodeA| and \verb|nodeB| are known nodes of a given line 
and \verb|Start node| is the given node of a parallel line. \verb|End node| is 
the name of the calculated line end. 

\begin{LTXexample}
\begin{pspicture*}[showgrid=true](-5,-3.5)(5,3.5)
\pnode(2,-2){FF}\qdisk(FF){1.5pt} \pnode(-5,5){A} \pnode(0,0){O}
\multido{\nCountA=-2.4+0.4}{9}{%
  \Parallel[linecolor=red,length=9](O)(A)(0,\nCountA){P1}
  \psline[linecolor=red](0,\nCountA)(FF)
  \psOutLine[linecolor=red,length=9](0,\nCountA)(FF){P2}}
\psline[linecolor=blue](A)(FF)
\psOutLine[linecolor=blue,length=5](A)(FF){END1}
\rput(0,0){\lens[yBottom=-3.5,yTop=3.5,lensGlass=true,lensHeight=6.5,drawing=false,
    spotFi=315,lensWidth=0.5]
  \psline[linewidth=1pt](xLeft)(xRight)
  \psline[length=2,linewidth=2pt,arrows=->](F')(FF)}
\end{pspicture*}
\end{LTXexample} 


\section{\nxLcs{ABinterCD} and \Lcs{nodeBetween}}

This macro is used by the \Lcs{telescop} macro. It determines the intersection 
point of two lines, in this case a ray and the mirror axis.
The following figure shows a part of figure~\ref{fig:defaultTele}. Given are 
the points A, B (focus), C/D (mirror axis). We need the point E to draw the other 
rays for the ocular, which can be done with the \Lcs{ABinterCD} macro. The syntax is:


\begin{BDef}
\Lcs{ABinterCD}\Largr{A}\Largr{B}\Largr{C}\Largr{D}\Largb{E}\\
\Lcs{nodeBetween}\Largr{A}\Largr{B}\Largb{C}
\end{BDef}

\begin{LTXexample}
\begin{pspicture*}(-0.5,-2.25)(9,2.25)
\rput(0,0){\mirrorCVG[mirrorHeight=4,mirrorWidth=0.25,mirrorDepth=0.25,drawing=false]}
\mirrorCVGRay[mirrorHeight=4,mirrorWidth=0.25,mirrorDepth=0.25,drawing=false](8,1.25)(2,1.25){A}
\psline[linewidth=0.5\pslinewidth](9,0)
\rput{-45}(6,0){\mirrorTwo}
\qdisk(A){2pt}\uput[30](A){A}\pnode(8,0){B}\qdisk(B){2pt}\uput[-45](B){B}
\pnode(! 6 1 45 cos mul sub 1 45 sin mul){C}
\qdisk(C){2pt}\uput[90](C){C}\pnode(! 6 1 45 cos mul add 1 45 sin mul neg){D}
\uput[-45](D){D}\qdisk(D){2pt}\psline[linestyle=dashed](A)(B)
\ABinterCD(A)(B)(C)(D){Inter1}\qdisk(A){2pt}
\nodeBetween(A)(B){M}\qdisk(M){2pt}
{\psset{linecolor=red}
\qdisk(Inter1){2pt}\uput[220](Inter1){E}\uput[220](M){M}}
\end{pspicture*}
\end{LTXexample}

\section{\nxLcs{rotateNode}}
The syntax is

\begin{BDef}
\Lcs{rotateNode}\Largb{NodeName}\Largb{Degrees}
\end{BDef}

The coordinates of the node \verb|A| are changed to the new ones. Negative values are possible for rotating clockwise.

\begin{LTXexample}[width=4cm,pos=l]
\begin{pspicture}[showgrid=true](4,4)
\pnode(3,1){A}\qdisk(A){2pt}\uput[20](A){A}
\rotateNode(A){45}
\qdisk(A){2pt}\uput[20](A){A}
\psarc[linecolor=red,
  linewidth=0.5pt]{->}(0,0){3.16}{19.47}{64.47}
\end{pspicture}
\end{LTXexample}


\section{\nxLcs{rotateTriangle}}
The syntax is

\begin{BDef}
\Lcs{rotateNode}\Largb{NodeNameA}\Largb{NodeNameB}\Largb{NodeNameC}\Largb{Degrees}
\end{BDef}

The coordinates of the nodes \verb|A,B,C| are changed to the new ones. Negative values are possible for rotating clockwise.

\begin{LTXexample}[width=5cm,pos=l]
\begin{pspicture}[showgrid=true](-1,0)(4,4)
\pnode(1,1){A}\pnode(3,1){B}\pnode(2,3){C}
\qdisk(A){2pt}\uput[180](A){A}\qdisk(B){2pt}\uput[0](B){B}
\qdisk(C){2pt}\uput[90](C){C}
\psline(A)(B)(C)(A) \rotateTriangle(A)(B)(C){45}
\qdisk(A){2pt}\uput[180](A){A}\qdisk(B){2pt}\uput[0](B){B}
\qdisk(C){2pt}\uput[90](C){C}\psline[linecolor=red](A)(B)(C)(A)
\psarc[linecolor=red,linewidth=0.5pt]{->}(0,0){3.16}{19.47}{64.47}
\psarc[linecolor=red,linewidth=0.5pt]{->}(0,0){1.41}{45}{90}
\psarc[linecolor=red,linewidth=0.5pt]{->}(0,0){3.61}{56.31}{101.31}
\end{pspicture}
\end{LTXexample}



\section{\nxLcs{rotateFrame}}
The syntax is

\begin{BDef}
\Lcs{rotateFrame}\Largb{NodeNameA}\Largb{NodeNameB}\Largb{NodeNameC}\Largb{NodeNameD}\Largb{Degrees}
\end{BDef}

The coordinates of the nodes \verb|A,B,C,D| are changed to the new ones. Negative values are possible for rotating clockwise.

\begin{LTXexample}[width=6cm,pos=l]
\begin{pspicture}[showgrid=true](-2,0)(4,5)
\pnode(1,1){A}\pnode(3,1){B}\pnode(3,3){C}\pnode(1,3){D}
\qdisk(A){2pt}\uput[180](A){A}\qdisk(B){2pt}\uput[0](B){B}
\qdisk(C){2pt}\uput[90](C){C} \qdisk(D){2pt}\uput[180](D){D}
\psline(A)(B)(C)(D)(A)
\rotateFrame(A)(B)(C)(D){45}
\qdisk(A){2pt}\uput[180](A){A}\qdisk(B){2pt}\uput[0](B){B}
\qdisk(C){2pt}\uput[90](C){C} \qdisk(D){2pt}\uput[180](D){D}
\psline[linecolor=red](A)(B)(C)(D)(A)
\psarc[linecolor=red,linewidth=0.5pt]{->}(0,0){3.16}{19.47}{64.47}
\psarc[linecolor=red,linewidth=0.5pt]{->}(0,0){1.41}{45}{90}
\psarc[linecolor=red,linewidth=0.5pt]{->}(0,0){4.24}{45}{90}
\psarc[linecolor=red,linewidth=0.5pt]{->}(0,0){3.16}{71.57}{116.57}
\end{pspicture}
\end{LTXexample}


\section{\nxLcs{arrowLine}}
The syntax is

\begin{BDef}
\Lcs{arrowLine}\OptArgs\Largr{Start}\Largr{End}\Largb{ArrowNumber}
\end{BDef}

Draws a line from \verb|Start| to \verb|End| with \verb|ArrowNumber| arrows inside. 

\begin{LTXexample}[width=4cm,pos=l]
\begin{pspicture}[showgrid=true](4,4)
\arrowLine[linecolor=red](0,0)(4,3){3}
\arrowLine[linecolor=green,arrowsize=6pt,arrows=-|](0,0)(3,1){2}
\arrowLine[linecolor=blue,arrowOffset=0.75,arrowsize=6pt](4,0)(0,3){3}
\end{pspicture}
\end{LTXexample}


\subsection{Options}

A special option is \Lkeyword{arrowOffset}, which makes it possible to draw lines with 
different arrows. By default the arrows are placed symetrically. This can be moved by 
\Lkeyword{arrowOffset}. Additionally all other valid options for pslines are possible her, too.

\begin{LTXexample}[width=4cm,pos=l]
\begin{pspicture}[showgrid=true](4,2)
\arrowLine[arrowsize=6pt,linecolor=red](0,0.5)(4,0.5){3}
\arrowLine[arrowsize=6pt,linecolor=red,
  arrows=<-](0,1)(4,1){3}
\arrowLine[arrowsize=6pt,linecolor=red](0,1.5)(4,1.5){3}
\arrowLine[arrowsize=6pt,linecolor=blue,arrows=<-,
  arrowOffset=0.2](0,1.5)(4,1.5){3}
\end{pspicture}
\end{LTXexample}



\clearpage
\section{List of all optional arguments for \texttt{pst-optic}}

\xkvview{family=pst-optic,columns={key,type,default}}




\bgroup
\raggedright
\nocite{*}
\bibliographystyle{plain}
\bibliography{pst-optic-doc}
\egroup

\printindex


\end{document}
