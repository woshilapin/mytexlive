\documentclass[11pt,english,BCOR10mm,DIV12,bibliography=totoc,parskip=false,smallheadings
    headexclude,footexclude,oneside]{pst-doc}
\usepackage[utf8]{inputenc}
\usepackage{pst-poly}
\let\pstDocFV\fileversion

\def\bgImage{\PstPolygon[unit=3,linecolor=red,PolyOffset=2,linewidth=0.001,
    PolyNbSides=72,PolyEpicycloid]}

\lstset{explpreset={pos=t,width=-99pt,overhang=0pt,hsep=\columnsep,vsep=\bigskipamount,rframe={}}}
\begin{document}

\title{\texttt{pst-poly}}
\subtitle{A PSTricks package for drawing polygons; v.\pstDocFV}
\author{Denis Girou\\Herbert Vo\ss}
%\docauthor{Herbert Vo\ss}
\date{\today}
\maketitle

\tableofcontents

\clearpage

\begin{abstract}
\noindent
\LPack{pst-poly} allows to draw easily various kinds of regular or non regular
polygons, using the unique macro \Lcs{PstPolygon}, with various customization
parameters.
It is also a good example of the great power and flexibility of PSTricks,
as in fact it is a very short program (it body is only 100 lines long) but
nevertheless really powerful.
    And last, it is also a good pedagogical example of how to design and
program high level graphic objects above PSTricks own ones.
\end{abstract}

\clearpage

\section{Introduction}
%
\LPack{pst-poly} offers a unique macro (plus some aliases to define
some often used polygons) with few parameters to interact on it. But we can
also use all the relevant PSTricks parameters to change the size, the
characteristics of lines, to add filling, etc.
The polygons are always drawn counter clockwise.
%
The syntax is simply:
\begin{BDef}
\LcsStar{PstPolygon}\OptArgs\\
\Lcs{PstTriangle}\OptArgs\\
\Lcs{PstSquare}\OptArgs\\
\Lcs{PstPentagon}\OptArgs\\
\Lcs{PstHexagon}\OptArgs\\
\Lcs{PstHeptagon}\OptArgs\\
\Lcs{PstOctogon}\OptArgs\\
\Lcs{PstNonagon}\OptArgs\\
\Lcs{PstDecagon}\OptArgs\\
\Lcs{PstDodecagon}\OptArgs\\%
\Lcs{PstStarFiveLines}\OptArgs\\
\Lcs{PstStarFive}\OptArgs\\
\Lcs{pspolygonbox}\OptArgs\Largb{Text}\\
\end{BDef}

As for PSTricks closed objects, the \texttt{*} version uses a solid style
to fill the polygon, use the line color for fill color and set the linewidth
to 0.

By default the polygons are set with a radius of 1 unit for the outer circle.

\bigskip
\begin{LTXexample}[pos=l,width=0.3\linewidth]
\begin{pspicture}[showgrid=true](-1,-1)(1,1)
  \PstPolygon[PstPicture=false]
  \pscircle[linestyle=dashed]{1}
\end{pspicture}
\end{LTXexample}

There is no special optional argument for this radius, the polygon can
be scaled by using the key \Lkeyword{unit}. With \Lkeyword{unit}=1.5, the outer
radius will be of 1.5cm when the current unit is set to 1cm.


\section{Optional arguments}
%
There are eight specific optional arguments defined to change the way the
polygons are defined:

\begin{compactdesc}
  \item [\Lkeyword{PstPicture}] (boolean): to define or not a \Lenv{pspicture}
  environment for the polygon. We have to define this parameter to
  \false\ if we want to mix the polygon with other PSTricks objects
  --- see examples later
  (\emph{Default:~\true} --- which is not the case for basic PSTricks
  objects). With \Lkeyset{PstPicture=false} the image doesn't reserve
  any space, it overwrites the text. The resulting box has a width and a height of 0pt.

\bigskip
\begin{LTXexample}[pos=t]
  foo\PstPolygon bar\hfill foo\PstPolygon[PstPicture=false] bar
\end{LTXexample}

\bigskip
  \item [\Lkeyword{PolyRotation}] (real): rotation angle applied to the polygon
  (\emph{Default:~0} --- no rotation).

\bigskip
\begin{LTXexample}[pos=t]
  \PstPolygon\hfill
  \PstPolygon[PolyRotation=18]\hfill
  \PstPolygon[PolyRotation=36]\hfill
  \PstPolygon[PolyRotation=45]
\end{LTXexample}

\bigskip
  \item [\Lkeyword{PolyNbSides}] (integer): number of sides of the polygon
  (\emph{Default:~5}).

\bigskip
\begin{LTXexample}[pos=t]
  \PstPolygon\hfill
  \PstPolygon[PolyNbSides=3]\hfill
  \PstPolygon[PolyNbSides=4]\hfill
  \PstPolygon[PolyNbSides=5]\hfill
  \PstPolygon[PolyNbSides=8]\hfill
  \PstPolygon[PolyNbSides=50]
\end{LTXexample}

\bigskip
  \item [\Lkeyword{PolyOffset}] (integer): number of nodes to bypass to obtain each
  time the next one (\emph{Default:~1} --- no node bypassed).

\bigskip
\begin{LTXexample}[pos=t]
  \PstPolygon
  \PstPolygon[PolyOffset=2]
  \PstPolygon[PolyOffset=3]
  \PstPolygon[PolyNbSides=7,PolyOffset=2]
  \PstPolygon[PolyNbSides=9,PolyOffset=4]
  \PstPolygon[PolyNbSides=17,PolyOffset=6]
\end{LTXexample}

\bigskip
  \item [\Lkeyword{PolyIntermediatePoint}] (real): position of the intermediate point
  used to join each time the next node (\emph{Default:~empty} --- not used).

\bigskip
\begin{LTXexample}[pos=t]
  \PstPolygon
  \PstPolygon[PolyIntermediatePoint=0.38]
  \PstPolygon[PolyIntermediatePoint=0.2]
  \PstPolygon[PolyIntermediatePoint=1.2]
  \PstPolygon[PolyNbSides=7,PolyOffset=2,
              PolyIntermediatePoint=0.38]
  \PstPolygon[PolyNbSides=21,PolyOffset=2,
              PolyIntermediatePoint=-1.25]
\end{LTXexample}

\bigskip
  \item [\Lkeyword{PolyCurves}] (boolean): boolean value to choose between straight
  line and curve to join each time the next node (\emph{Default:~\false} --- straight lines).

\bigskip
\begin{LTXexample}[pos=t]
  \psset{PolyCurves=true}
  \PstPolygon[PolyNbSides=7,PolyIntermediatePoint=0.38]\hfill
  \PstPolygon[PolyNbSides=11,PolyIntermediatePoint=0.6]\hfill
  \PstPolygon[PolyNbSides=7,PolyIntermediatePoint=0.2,
              PolyOffset=2]\hfill
  \PstPolygon[PolyNbSides=9,PolyIntermediatePoint=0.1]\hfill
  \PstPolygon[PolyNbSides=15,PolyIntermediatePoint=1.2]
\end{LTXexample}

\bigskip
  \item [\Lkeyword{PolyEpicycloid}] (boolean): boolean value to choose between
  polygon and epicycloid (\emph{Default:~\false} --- polygon).

\bigskip
\begin{LTXexample}[pos=t]
  \psset{linewidth=0.001,PolyNbSides=72,PolyEpicycloid=true}
%  Epicycloid of factor 1 is cardioid and of factor 2 nephroid
  \multido{\i=2+1}{4}{\PstPolygon[PolyOffset=\i]\hfill}
  \PstPolygon[PolyOffset=72]\hfill% Epicycloid of factor 71
  \PstPolygon[PolyOffset=73]      % Epicycloid of factor 72
\end{LTXexample}

\bigskip
  \item [\Lkeyword{PolyName}] (string): name of the polygon, useful to have different
  names for the nodes of different polygons (\emph{Default:~empty} --- no name).

  The center of the polygon has name \texttt{PolyName0} and the nodes
(vertices) have names \texttt{PolyName1} to \texttt{PolyNameN}.
  With this parameter, we can connect as we want nodes of different polygons:

\bigskip
\begin{LTXexample}[pos=l,width=0.3\linewidth,pos=l]
  \psset{PstPicture=false}
  \begin{pspicture}(-1,-1)(1,1)
    \PstPolygon[unit=0.8,PolyName=A,PolyNbSides=6]
    \PstPolygon[unit=1.2,PolyName=B,PolyNbSides=12]
  \end{pspicture}
  \multido{\i=1+2}{6}{%
    \ncline{A\the\multidocount}{B\i}}
\end{LTXexample}

It is also a way (limited in fact...) to define three dimensional objects in perspective:

\bigskip
\begin{LTXexample}[pos=l,width=0.3\linewidth,pos=l]
\psset{unit=0.8}
\begin{pspicture}(3,2.5)
%  \PstSquare is described later
  \rput[lb](0,0){\PstSquare[PolyName=A]}
    \rput[lb](2.5,2){\PstSquare[unit=0.5,PolyName=B]}
    \multido{\i=1+1}{4}{\ncline{A\i}{B\i}}
  \end{pspicture}
\end{LTXexample}

\end{compactdesc}

\bigskip
  Of course, we can mix specific parameters of \LPack{pst-poly} with
relevant PSTricks ones and combine it with other generic macros
(for repetitions, projection in the 3d space, etc.)


\begin{LTXexample}[pos=t]
  \PstPolygon[unit=1.5]
  \PstPolygon[linewidth=1mm]
  \PstPolygon[fillstyle=crosshatch,PolyNbSides=6]
  \PstPolygon[linecolor=red,fillstyle=solid,fillcolor=yellow,
              PolyRotation=18,PolyOffset=2]
  \PstPolygon*[linecolor=cyan,PolyNbSides=7,PolyOffset=2]
\end{LTXexample}

\begin{LTXexample}[pos=t]
  \multido{\i=5+2}{6}{%
    \PstPolygon[PolyNbSides=\i,PolyCurves=true,
                PolyIntermediatePoint=0.1,PolyOffset=3]}
\end{LTXexample}


\section{Pre-defined polygons}
%
  Some often used polygons and other related geometric objects are
pre-defined, for immediate usage :

\begin{LTXexample}[pos=t]
  \PstTriangle 
  \PstSquare
  \PstPentagon[unit=0.5,linestyle=dotted]
  \PstHexagon[fillstyle=hlines,hatchangle=90]
  \PstHeptagon[fillstyle=vlines]
  \PstOctogon*[unit=0.8,linecolor=red]
  \PstNonagon[unit=0.5]
  \PstDecagon
  \PstDodecagon[linecolor=green] \par
  \PstStarFiveLines
  \PstStarFive
\end{LTXexample}


\section{Non regular polygons}

Until now, we have described only the so-called \emph{regular} polygons,
which are from far the most useful ones (all of them have equal edges and
angles). Nevertheless, it is not so difficult to extend these polygon to
\emph{non regular} ones, using a different value for horizontal and vertical
units (nevertheless, the code is more tricky, as we must do all the
trigonometry explicitely...)
%
\begin{LTXexample}[pos=t]
  \PstPentagon[xunit=0.5]\hfill
  \PstHexagon[yunit=0.5]\hfill
  \PstStarFive[xunit=0.5,yunit=1.5]\hfill
  \PstPolygon[xunit=0.8,yunit=1.5,PolyNbSides=9,PolyOffset=2,
              PolyIntermediatePoint=0.1,PolyCurves=true]
\end{LTXexample}
%
\section{Nodes (vertices)}
%
And another powerful possibility is to define a command
\Lcs{PstPolygonNode}  which will be excuted at each node (\emph{Default:~empty} --- nothing executed). 
The counter name for nodes is \Lctr{INode}, starting from 0. The \Lcs{multidocount} counter, from the
\Lcs{multido} command, start itself from 1.

\begin{LTXexample}[width=0.3\linewidth,pos=l]
  \providecommand{\PstPolygonNode}{%
    \psdots[dotsize=0.2,linecolor=cyan](1;\INode)}
  \PstPentagon
\end{LTXexample}
%
\begin{LTXexample}[width=0.3\linewidth,pos=l]
  \newcounter{Letter}
  \providecommand{\PstPolygonNode}{%
    \setcounter{Letter}{\the\multidocount}%
    \rput*{*0}(1;\INode){\small\Alph{Letter}}}
  \PstHeptagon[PolyOffset=3]
\end{LTXexample}
%
\begin{LTXexample}[pos=l,width=0.3\linewidth]
  \providecommand{\PstPolygonNode}{%
    \psdots[dotstyle=o,dotsize=0.2](1;\INode)
    \psline[linecolor=red]{->}(0.9;\INode)}
  \PstPolygon[PolyNbSides=8]
\end{LTXexample}
%
\vspace{3mm}
  It is also a way to nest polygons:
\vspace{3mm}
%
\begin{LTXexample}[pos=l,width=0.3\linewidth]
  \newbox{\Star}
  \savebox{\Star}{%
    \PstStarFive*[unit=0.15,linecolor=red]}
  \providecommand{\PstPolygonNode}{%
    \rput{*0}(1;\INode){\usebox{\Star}}}
  \shortstack{\PstNonagon\\[5mm]
              \PstDodecagon[linestyle=none]}
\end{LTXexample}

\begin{LTXexample}[pos=t]
  \psset{unit=2,PstPicture=false}
  \begin{pspicture}(-1.6,-1.6)(1.6,1.6)
%    Just to name the nodes, to be able to join some of them.
%    by solid curves. We can't draw the polygons now,
%    as the node numbers must erase the solid curves...
    \PstHexagon[unit=0.8,linestyle=none,PolyName=H]\PstDodecagon[unit=1.2,linestyle=none,PolyName=D]
    {\SpecialCoor
     \psset{linewidth=0.4pt,border=2pt,nodesep=0.45}
     \psccurve(H1)(H4)(D8)([angle=-30]D9)(D11)\psccurve(D3)(H3)(H6)(D12)([angle=90]D1)
     \psccurve(D4)(H2)(H5)(D7)([angle=90]D6)}
    \psset{linestyle=dotted,framesep=1pt}
    \PstTriangle[unit=0.23]
    \providecommand{\PstPolygonNode}{\rput*{*0}(1;\INode){\small\the\multidocount}}
    \PstHexagon[unit=0.8] \PstDodecagon[unit=1.2] \PstDodecagon[unit=1.6]
  \end{pspicture}
\end{LTXexample}
%
\section{Polygonbox}

The valid options with the predefined values are \Lkeyword{PolyNbSides}=3 and \Lkeyword{PolyRotation}=0

\begin{compactitem}
\item There maybe some problems with linearcs and rounding errors. 
\item To rotate the text inside the box, one can use the \Lcs{rotatebox} macro from
      the \LPack{rotating} package (see examples).
\end{compactitem}

\begin{LTXexample}[pos=t]
  \multido{\nA=3+1}{8}{%
     \pspolygonbox[PolyNbSides=\nA,framesep=3mm]{Text}~}
\end{LTXexample}

\begin{LTXexample}[pos=t]
   \multido{\nA=3+1}{8}{\pspolygonbox[PolyNbSides=\nA,%
        framesep=2mm,doubleline=true]{Text}~}
\end{LTXexample}


\begin{LTXexample}[pos=t]
  \psset{PolyNbSides=4}
  \multido{\nA=0+60}{8}{\pspolygonbox[PolyRotation=\nA,%
      framesep=2mm,doubleline=true,%
      linecolor=red,doublecolor=blue]{\magenta Text}~}
\end{LTXexample}

\begin{LTXexample}[pos=t]
  \multido{\nA=0+60,\nB=45+60}{8}{\pspolygonbox%
     [PolyRotation=\nA,framesep=2mm,doubleline=true,%
       linecolor=red,doublecolor=blue,linearc=0.4]%
       {\rotatebox{\nB}{\blue Text}}~}\\[1cm]
\end{LTXexample}

\begin{LTXexample}[pos=t]
  \multido{\nA=3+1}{8}{%
    \pspolygonbox[PolyNbSides=\nA,rot=60,framesep=2pt,%
      doubleline=true,linearc=0.2]{Text}~}
\end{LTXexample}

\begin{LTXexample}[pos=t]
  \multido{\nA=3+1}{8}{%
     \pspolygonbox[PolyNbSides=\nA,framesep=10pt,%
     fillstyle=solid,fillcolor=cyan,linearc=0.2]{Text}~}
\end{LTXexample}

\begin{LTXexample}[pos=t]
  \multido{\nA=-5+5}{8}{\pspolygonbox[framesep=\nA pt,%
     PolyRotation=90]{\Huge\red T}~}
\end{LTXexample}

\vspace{1cm}
\begin{LTXexample}[pos=t]
  \multido{\nA=-5+5}{7}{\pspolygonbox[framesep=\nA pt,%
     PolyRotation=90,boxsep=false]{\Huge\red T}~}\par
  \rule{0pt}{1cm}
\end{LTXexample}

\section{Some more examples}

\begin{LTXexample}[pos=t]
  \multido{\i=3+1}{6}{%
    \PstPolygon[PolyNbSides=\i]\hspace{5mm}}
\end{LTXexample}

\begin{LTXexample}[pos=t]
  \multido{\i=3+2}{6}{%
    \PstPolygon[PolyOffset=2,PolyNbSides=\i]\hspace{5mm}}
\end{LTXexample}

\begin{LTXexample}[pos=t]
  \multido{\i=3+1}{10}{%
    \PstPolygon[PolyOffset=3,PolyNbSides=\i]\hspace{5mm}}
\end{LTXexample}

\begin{LTXexample}[pos=t]
  \multido{\i=5+1}{10}{%
    \PstPolygon[PolyOffset=4,PolyNbSides=\i]\hspace{5mm}}
\end{LTXexample}

\begin{LTXexample}[pos=t]
  \multido{\i=5+2}{10}{%
    \PstPolygon[PolyOffset=5,PolyNbSides=\i]\hspace{5mm}}
\end{LTXexample}

\begin{LTXexample}[pos=t]
  \multido{\i=5+2}{10}{%
    \PstPolygon[PolyOffset=7,PolyNbSides=\i]\hspace{5mm}}
\end{LTXexample}

\begin{LTXexample}[pos=t]
  \multido{\i=5+2}{10}{%
    \PstPolygon[PolyOffset=8,PolyNbSides=\i]\hspace{5mm}}
\end{LTXexample}

\begin{LTXexample}[pos=t]
  \multido{\i=1+1}{10}{%
    \PstPolygon[PolyOffset=\i,PolyNbSides=5]\hspace{5mm}}
\end{LTXexample}

\begin{LTXexample}[pos=t]
  \multido{\i=1+1}{10}{%
    \PstPolygon[PolyOffset=\i,PolyNbSides=7]\hspace{5mm}}
\end{LTXexample}

\begin{LTXexample}[pos=t]
  \multido{\i=5+1}{10}{%
    \PstPolygon[PolyCurves,PolyIntermediatePoint=0.1,PolyNbSides=\i]
    \hspace{5mm}}
\end{LTXexample}

\begin{LTXexample}[pos=t]
  \multido{\i=5+1}{10}{%
    \PstPolygon[PolyCurves,PolyIntermediatePoint=0.2,
                PolyOffset=2,PolyNbSides=\i]\hspace{5mm}}
\end{LTXexample}

\begin{LTXexample}[pos=t]
  \multido{\i=5+2}{10}{%
    \PstPolygon[PolyCurves,PolyIntermediatePoint=0.1,
                PolyOffset=3,PolyNbSides=\i]\hspace{5mm}}
\end{LTXexample}

% Second serie

\begin{LTXexample}[pos=t]
  \multido{\n=-1.4+0.5}{10}{%
    \PstPolygon[PolyNbSides=3,PolyOffset=2,PolyIntermediatePoint=\n]
    \hspace{5mm}}\par
    \vspace{0pt}{2cm}
\end{LTXexample}

\begin{LTXexample}[pos=t]
  \multido{\n=-1.4+0.5}{6}{%
    \PstPolygon[PolyNbSides=5,PolyOffset=2,PolyIntermediatePoint=\n]
    \hspace{5mm}}
\end{LTXexample}

\begin{LTXexample}[pos=t]
  \multido{\n=-1.4+0.5}{6}{%
    \PstPolygon[PolyNbSides=13,PolyOffset=2,PolyIntermediatePoint=\n]
    \hspace{5mm}}
\end{LTXexample}

\begin{LTXexample}[pos=t]
  \multido{\n=-1.4+0.5}{6}{%
    \PstPolygon[PolyNbSides=21,PolyOffset=2,PolyIntermediatePoint=\n]
    \hspace{5mm}}
\end{LTXexample}

% Third serie

\begin{LTXexample}[pos=t]
  \psset{unit=1.4,linewidth=0.001,PolyNbSides=72,PolyEpicycloid}
  \multido{\i=2+1}{4}{%
    % Epicycloid of factor 1 is cardioid and of factor 2 nephroid
    \PstPolygon[PolyOffset=\i]\hspace{5mm}}
\end{LTXexample}

\begin{LTXexample}[pos=l,width=0.3\linewidth]
  % Epicycloid of factor 10
  \PstPolygon[unit=2,linewidth=0.003,
              PolyEpicycloid,PolyNbSides=72,PolyOffset=11]
\end{LTXexample}

\begin{LTXexample}[pos=l,width=0.3\linewidth]
  % Epicycloid of factor 22
  \PstPolygon[unit=2,linewidth=0.003,
              PolyEpicycloid,PolyNbSides=72,PolyOffset=23]
\end{LTXexample}

\begin{LTXexample}[pos=t]
  \psset{unit=1.9,linewidth=0.001,PolyNbSides=72,PolyEpicycloid}
  \multido{\i=71+1}{3}{%
    \PstPolygon[PolyOffset=\i]\hspace{5mm}}
\end{LTXexample}

\begin{LTXexample}[pos=l,width=0.3\linewidth]
  % Epicycloid of factor 100
  \PstPolygon[unit=2,linewidth=0.003,
              PolyEpicycloid,PolyNbSides=72,PolyOffset=101]
\end{LTXexample}

\begin{LTXexample}[pos=l,width=0.3\linewidth]
  % Epicycloid of factor 153
  \PstPolygon[unit=2,linewidth=0.003,
              PolyEpicycloid,PolyNbSides=72,PolyOffset=154]
\end{LTXexample}

% Fourth serie

\begin{LTXexample}[pos=l,width=0.3\linewidth]
  \providecommand{\PstPolygonNode}{%
    \psdots[dotsize=0.2,linecolor=cyan](1;\INode)}
  \PstPentagon[unit=2]
\end{LTXexample}

\begin{LTXexample}[pos=l,width=0.3\linewidth]
  \providecommand{\PstPolygonNode}{%
    \rput{*0}(1.2;\INode){\small\the\multidocount}}
  \PstPolygon[unit=2,PolyNbSides=7,PolyOffset=2]
\end{LTXexample}

\begin{LTXexample}[pos=l,width=0.3\linewidth]
  \providecommand{\PstPolygonNode}{%
    \rput*{*0}(1;\INode){\small\the\multidocount}}
  \PstHeptagon[unit=2,PolyOffset=2]
\end{LTXexample}

\begin{LTXexample}[pos=l,width=0.3\linewidth]
%  \newcounter{Letter}
  \providecommand{\PstPolygonNode}{%
    \setcounter{Letter}{\the\multidocount}%
    \rput*{*0}(1;\INode){\small\Alph{Letter}}}
  \PstHeptagon[unit=2,PolyOffset=3]
\end{LTXexample}

\begin{LTXexample}[pos=l,width=0.3\linewidth]
  \providecommand{\PstPolygonNode}{%
    \SpecialCoor
    \degrees[3]
    \rput{0.5}(0.5;\INode){%
      \pspolygon*(0.5;0.5)(0.5;1.5)(0.5;2.5)}}
  \PstTriangle
\end{LTXexample}

\begin{LTXexample}[pos=l,width=0.3\linewidth]
  \providecommand{\PstPolygonNode}{%
    \psdots[dotstyle=o,dotsize=0.2](1;\INode)
    \psline[linecolor=red]{->}(0.9;\INode)}
  \PstPolygon[unit=2,PolyNbSides=8]
\end{LTXexample}

\begin{LTXexample}[pos=l,width=0.3\linewidth]
  \providecommand{\PstPolygonNode}{%
    \psline[linewidth=0.1mm,doubleline=true,
            linecolor=green]{<->}(0;0)(1;\INode)}
  \PstHexagon[unit=2]
\end{LTXexample}

\begin{LTXexample}[pos=l,width=0.3\linewidth]
  \newbox{\Star}
  \savebox{\Star}{%
    \PstStarFive*[unit=0.15,linecolor=red]}
  \providecommand{\PstPolygonNode}{%
    \rput{*0}(1;\INode){\usebox{\Star}}}
  \shortstack{%
    \PstNonagon\\[5mm]
    \PstDodecagon[linestyle=none]}
\end{LTXexample}

\begin{LTXexample}[pos=l,width=5cm]
%\newcounter{Letter}% if not defined
\setcounter{Letter}{\the\multidocount}%
\begin{pspicture}(-2.5,-2.5)(2.5,2.5)
\PstPolygon[PolyName=A,fillstyle=solid,fillcolor=lightgray,linewidth=2pt,PolyNbSides=5,unit=0.5,PstPicture=false]
\PstPolygon[PolyName=B,PolyNbSides=5,linewidth=2pt,PstPicture=false,unit=2.5]
\multido{\iA=1+1,\iB=2+1}{5}{%
  \setcounter{Letter}{\the\multidocount}%
  \ncline{A\iA}{B\iA}\ncput*{\bf\alph{Letter}}%
  \ifnum\iB>5 \def\iB{1}\fi%
  \pcline[linestyle=none](B\iA)(B\iB)\ncput*{\bf\iA}}
\end{pspicture}
\end{LTXexample}

\clearpage
\section{List of all optional arguments for \texttt{pst-poly}}

\xkvview{family=pst-poly,columns={key,type,default}}


\nocite{*}
\bgroup
\RaggedRight
\bibliographystyle{plain}
\bibliography{pst-poly-doc}
\egroup

\printindex

\end{document}


