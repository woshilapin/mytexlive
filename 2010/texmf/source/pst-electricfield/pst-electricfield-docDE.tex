%% $Id: pst-electricfield-docDE.tex 479 2011-03-26 10:12:49Z herbert $
\documentclass[11pt,english,ngerman,BCOR10mm,DIV12,bibliography=totoc,parskip=false,smallheadings
    headexclude,footexclude,oneside]{pst-doc}
\usepackage[latin1]{inputenc}
\usepackage{pst-electricfield}
\let\pstEFfv\fileversion
\usepackage{pst-func}
\usepackage{pst-exa}% only when running pst2pdf
%\let\PSTexample\LTXexample        % when not running pst2pdf
%\let\endPSTexample\endLTXexample  % when not running pst2pdf


\usepackage{esint}
\lstset{pos=t,language=PSTricks,
    morekeywords={psElectricfield,psEquipotential},basicstyle=\footnotesize\ttfamily}
\newcommand\Cadre[1]{\psframebox[fillstyle=solid,fillcolor=black,linestyle=none,framesep=0]{#1}}
%
\title{\texttt{pst-electricfield}}
\subtitle{Feldlinien und \"{A}quipotentiallinien elektrischer Punktladungen; v.\pstEFfv}
\author{J\"{u}rgen Gilg\\ Manuel Luque\\ Patrice M\'egret\\ Herbert Vo\ss}


\begin{document}
\maketitle
\begin{abstract}
Das Paket \texttt{pst-electricfield} hat sich zum Ziel gesetzt Feldlinien und \"{A}quipotentiallinien 
zu zeichnen f\"{u}r eine beliebige  Anordnung von elektrischen Punktladungen. Die Idee f\"{u}r ein 
solches Paket ist entstanden durch eine Diskussion \"{u}ber das Darstellen von Feldlinien in der 
PSTricks Liste \url{http://www.tug.org/pipermail/pstricks/}. Es gibt verschiedene Methoden und 
Ans\"{a}tze -- diese wollen wir auch in dieser Dokumentation vorstellen.

In diesem Paket werden die Feldlinien mit dem Euler-Verfahren errechnet; dieses Verfahren ist 
einerseits ausreichend f\"{u}r die Pr\"{a}zision der Darstellung und liefert andererseits eine gute 
Rechengeschwindigkeit (Kompilierungsdauer). Die numerische L\"{o}sung der impliziten Gleichung 
f\"{u}r das Potential $V(x,y)=\Sigma V_i$ erlaubt es die \"{A}quipotentiallinien darzustellen, 
die Rechengeschwindigkeit hierf\"{u}r ist jedoch sehr viel kleiner. Das Paket stellt zwei Befehle 
zur Verf\"{u}gung, einen f\"{u}r die Feldlinien und einen f\"{u}r die \"{A}quipotentiallinien. 
Wegen der erh\"{o}hten Rechendauer f\"{u}r die \"{A}quipotentiallinien ist es zu erw\"{a}gen sich 
nur auf die Feldlinien zu beschr\"{a}nken.

Jede Ladung ist charakterisiert durch ihren Wert $q_i$ und ihre Position $(x_i,y_i)$. Die Anzahl 
der Ladungen ist frei w\"{a}hlbar, jedoch steigt mit ihr auch erheblich die Rechendauer f\"{u}r 
die \"{A}quipotentiallinien.
\end{abstract}

\section{Vorgeschlagene Methode von Patrice M\'egret}

Mit dem Paket \LPack{pst-func} und dem Befehl \Lcs{psplotImp}\verb+[options](x1,y1)(x2,y2)+
kann man die Feldlinien \textbf{und} die \"{A}quipotentiallinien zeichnen.

Wie leitet man die implizite Funktion der Feldlinien mit Hilfe des elektrischen Potentials her?

Der Gau{\ss}sche Satz sagt aus, dass der Flu{\ss} durch eine geschlossene Oberfl\"{a}che $S$ 
durch folgende Gleichung definiert ist:
\begin{equation}\label{pm-eq-a}
\psi = \oiint\limits_S \vec{D} \cdot \vec{u}_n \mathrm{d} S = Q
\end{equation}
ist gleich der Ladung $Q$ im Inneren von $S$. Au{\ss}erhalb der geschlossenen Oberfl\"{a}che ($Q=0$). 
Der elektrische Flu{\ss} ist konservativ.

Eine Flu{\ss}r\"{o}hre ist eine R\"{o}hre, die um die Linien der dielektrischen Verschiebung $\vec{D}$ 
gebaut ist au{\ss}erhalb der Ladungen. Der eintretende Flu{\ss} in diese R\"{o}hre ist gleich dem 
austretenden Flu{\ss} aus der R\"{o}hre (der Flu{\ss} ist konservativ).

Folgt man einer Flussr\"{o}hre konstanter Gr\"{o}{\ss}e, so folgt man auch einer Feldlinie $\vec{D}$ 
und dieser Ansatz wird gew\"{a}hlt, um eine implizite Gleichung von Feldlinien einfacher geometrischer Anordungen zu erhalten.

In unserem Fall begn\"{u}gen wir uns mit Punktladungen und der Identit\"{a}t von der dielektrischen 
Verschiebung und der Feldst\"{a}rke (da wir keine Polarisation ber\"{u}cksichtigen).

F\"{u}r eine elektrische Punktladung $q$ im Ursprung eines Koordinatensystems ist die elektrische 
Feldst\"{a}rke und das Potential gegeben durch:
\begin{equation}\label{pm-eq-b}
\vec{E} = \frac{1}{4 \pi \varepsilon_0 \varepsilon_r} q \frac{\vec{r}}{|\vec{r}|^3}
\end{equation}
\begin{equation}\label{pm-eq-c}
V = \frac{1}{4 \pi \varepsilon_0 \varepsilon_r} \frac{q}{r}
\end{equation}

Der Flu{\ss} durch eine Kugelkappe mit der Oberfl\"{a}che $S$ deren halber \"{O}ffnungswinkel $\theta$ ist, ist gleich:
\begin{equation}\label{pm-eq-d}
\psi = \varepsilon_0 \varepsilon_r E S = \frac{1}{2} q (1 -\cos\theta)
\end{equation}
denn $S= 2\pi r^2 (1 - \cos\theta)$ und auf Grund von (\ref{pm-eq-a}) $4 \pi r^2 \varepsilon_0 \varepsilon_r E =q$

\begin{center}
\begin{pspicture}(-3,-3)(3,3)
%\psgrid
\psdot[dotscale=2](0,0)
\uput[-135](0,0){$q$}
\psaxes[labels=none,ticks=none]{->}(0,0)(-2.5,-2.5)(2.5,2.5)[$x$,-90][$y$,0]
\pswedge(0,0){2}{-30}{30}
\psarc{->}(0,0){1}{0}{30}
\rput(1.2,0.2){$\theta$}
\rput(2.2,0.7){$S$}
\end{pspicture}
\end{center}

Um einen impliziten Ausdruck f\"{u}r die Feldlinien zu erhalten, gen\"{u}gt es die Konstanz des Flusses zum Ausdruck zu bringen: 
\begin{equation}\label{pm-eq-e}
\psi(x,y) = \frac{1}{2} q (1 -\cos\theta) = \mathrm{konst.}
\end{equation}
Man sieht sofort, dass die Feldlinien f\"{u}r $\theta=\mathrm{konst.}$ radial verlaufen.

Daraus folgt f\"{u}r die Feldlinien in der $xy$-Ebene in kartesischen Koordinaten:
\begin{equation}\label{pm-eq-f}
\frac{x}{\sqrt{x^2+y^2}} = \mathrm{konst.}
\end{equation}
F\"{u}r die \"{A}quipotentiallinien ist die Gleichung (\ref{pm-eq-c}) schon in impliziter Form, es gen\"{u}gt $V=\mathrm{konst.}$ zu setzen, dies liefert:
\begin{equation}\label{pm-eq-g}
\frac{1}{\sqrt{x^2+y^2}} = \mathrm{konst.}
\end{equation}

\begin{center}
\begin{pspicture*}(-5,-5)(5,5)
\psframe*[linecolor=green!20](-5,-5)(5,5)
\psgrid[subgriddiv=0,gridcolor=lightgray,griddots=10]
\psElectricfield[Q={[1 0 0]}]
\psEquipotential[Q={[1 0 0]}](-5,-5)(5,5)
\multido{\r=-1+0.1}{20}{%
\psplotImp[linestyle=solid,linecolor=blue](-6,-6)(6,6){%
x y 2 exp x 2 exp add sqrt div \r \space sub}}
\multido{\r=0.0+0.1}{10}{%
\psplotImp[linestyle=solid,linecolor=red](-6,-6)(6,6){%
x 2 exp y 2 exp add sqrt 1 exch div \r \space sub}}
\end{pspicture*}
\end{center}

\begin{verbatim}
%% lignes de champ
\psplotImp[linestyle=solid,linecolor=blue](-6,-6)(6,6){%
x y 2 exp x 2 exp add sqrt div \r \space sub}}

%% \'{e}quipotentielles
\multido{\r=0.0+0.1}{10}{%
\psplotImp[linestyle=solid,linecolor=red](-6,-6)(6,6){%
x 2 exp y 2 exp add sqrt 1 exch div \r \space sub}} 
\end{verbatim}

Nun verallgemeinern wir eine Punktladungsverteilung l\"{a}ngs einer \textbf{Geraden}. Gegeben sind die 
Punktladungen $q_i$ mit ihren Koordinaten $(x_i,0)$.
\begin{center}
\begin{pspicture}(0,-3)(12,3)
%\psgrid
\psset{dotscale=2}
\dotnode(0,0){NA}\nput{-45}{NA}{$q_1$}
\dotnode(2,0){NB}\nput{-90}{NB}{$q_2$}
\dotnode(5,0){NC}\nput{-90}{NC}{$q_n$}
\dotnode[linecolor=red](4,2){ND}\nput{90}{ND}{$P(x,y)$}
\ncline{NA}{ND}\naput{$r_1,\theta_1$}
\ncline{NB}{ND}\nbput{$r_2,\theta_2$}
\ncline{NC}{ND}\nbput{$r_n,\theta_n$}
\psaxes[labels=none,ticks=none]{->}(0,0)(0,-2.5)(11,2.5)[$x$,-90][$y$,0]
\psarc{->}(5,0){0.7}{0}{116.5}
\rput(6,0.5){$\theta_n$}
\dotnode[linecolor=blue](4,-2){NE}
\nccurve[ncurv=2,linecolor=green!40!black]{ND}{NE}
\end{pspicture}
\end{center}
Es liegt eine Zylindersymmetrie vor; es gen\"{u}gt deshalb die Feldlinien und das Potential in der 
oberen Halb-Ebene $xy$ zu untersuchen und mit einer Rotation um die $x$-Achse erh\"{a}lt man somit 
die Gesamtl\"{o}sung.



Bei Rotation um die $x$-Achse, erzeugt die Feldlinie, die durch den Punkt $P$ geht, eine Flussr\"{o}hre, 
deren elektrischer Flu{\ss} 
 durch eine beliebige Oberfl\"{a}che durch $P(x,y)$ hindurchflie{\ss}t und die $x$-Achse jenseits 
 der letzten Ladung schneidet (diese Oberfl\"{a}che schneidet die $xy$-Ebene in dem gr\"{u}nen 
 Bogen) gem\"{a}{\ss} (\ref{pm-eq-d}):
\begin{equation}\label{pm-eq-h}
\psi = \frac{1}{2} \sum_{i=1}^{n} q_i (1 -\cos\theta_i)
\end{equation}
Die Feldlinien erh\"{a}lt man sehr einfach, wenn man $\psi = \mathrm{konst.}$ setzt. In kartesischen Koordinaten:
\begin{equation}\label{pm-eq-i}
\sum_{i=1}^{n} q_i \frac{x-x_i}{\sqrt{(x-x_i)^2+y^2}} = \mathrm{konst.}
\end{equation}

F\"{u}r das Potential erh\"{a}lt man trivial:
\begin{equation}\label{pm-eq-j}
\sum_{i=1}^{n} \frac{q_i}{\sqrt{(x-x_i)^2+y^2}} = \mathrm{konst.}
\end{equation}

\begin{center}
\begin{pspicture*}(-5,-5)(5,5)
\psframe*[linecolor=green!20](-5,-5)(5,5)
\psgrid[subgriddiv=0,gridcolor=lightgray,griddots=10]
\psElectricfield[Q={[1 -2 0][-1 2 0]}]
\psEquipotential[Q={[1 -2 0][-1 2 0]},Vmin=-2,Vmax=2,stepV=0.25](-5,-5)(5,5)
\multido{\r=-2+0.2}{20}{%
\psplotImp[linestyle=solid,linecolor=red](-6,-6)(6,6){%
x 2 add dup 2 exp y 2 exp add sqrt div 1 mul
x -2 add dup 2 exp y 2 exp add sqrt div -1 mul add
\r \space sub}}
\multido{\r=-0.5+0.1}{10}{%
\psplotImp[linestyle=solid,linecolor=blue](-6,-6)(6,6){%
x 2 add 2 exp y 2 exp add sqrt 1 exch div 1 mul
x -2 add 2 exp y 2 exp add sqrt 1 exch div -1 mul add
\r \space sub}}
\end{pspicture*}
\end{center}

\begin{verbatim}
%% lignes de champ
\multido{\r=-2+0.2}{20}{%
\psplotImp[linestyle=solid,linecolor=red](-6,-6)(6,6){%
x 2 add dup 2 exp y 2 exp add sqrt div 1 mul
x -2 add dup 2 exp y 2 exp add sqrt div -1 mul add
\r \space sub}}
%% \'{e}quipotentielles
\multido{\r=-0.5+0.1}{10}{%
\psplotImp[linestyle=solid,linecolor=blue](-6,-6)(6,6){%
x 2 add 2 exp y 2 exp add sqrt 1 exch div 1 mul
x -2 add 2 exp y 2 exp add sqrt 1 exch div -1 mul add
\r \space sub}}
\end{verbatim}

Das dargestellte Beispiel besitzt eine Ladung $+1$ in $(-2,0)$ und eine Ladung $-1$ in $(2,0)$ und 
zeigt die \"{U}berlagerung der Resultate von  impliziter Methode und direkter Integration. Das deckt 
sich gut, jedoch ist die implizite Methode langsamer und auf ein zylindersymmetrisches Problem 
eingeschr\"{a}nkt (Ladungsanordnung l\"{a}ngs einer Geraden).


\newpage
\section{Vorgeschlagene Methode von J\"{u}rgen Gilg}
Mit dem Paket \textsf{pst-func} und dem Befehl \verb+\psplotDiffEqn+ kann man Feldlinien \textbf{und} \"{A}quipotentiallinien zeichnen.

\textbf{Feldlinien}

Gegeben sind die Punktladungen $\{q_1, \,\ldots, \,q_n\}$ und ihre Ortsvektoren $\{\vec{r}_1, \,\ldots, \,\vec{r}_n\}$.
\begin{equation*}
\vec{r}_1=\begin{pmatrix}
x_1\\y_1
\end{pmatrix},\,\ldots,\,
\vec{r}_n=\begin{pmatrix}
x_n\\y_n
\end{pmatrix};\,
\vec{r}=\begin{pmatrix}
x\\y
\end{pmatrix}
\end{equation*}
Mit dem Prinzip der Superposition erh\"{a}lt man die resultierende Feldst\"{a}rke im Punkt $M$ mit $\overrightarrow{r}(M)$:
\begin{equation}
\vec{E} = \frac{1}{4 \pi \varepsilon_0 \varepsilon_r} \sum\limits_{i=1}^n q_i \frac{\vec{r} - \vec{r}_i}{|\vec{r} - \vec{r}_i|^3}
\end{equation}
In Komponentendarstellung:
\begin{equation}
\vec{E} =\begin{pmatrix}
E_x\\E_y
\end{pmatrix}=
 \frac{1}{4 \pi \varepsilon_0 \varepsilon_r} \sum\limits_{i=1}^n  \frac{q_i}{\sqrt{(x-x_i)^2+(y-y_i)^2}^3}\begin{pmatrix}
   x-x_i\\y-y_i
 \end{pmatrix}
\end{equation}
oder
\begin{align*}
E_x&=\frac{1}{4 \pi \varepsilon_0 \varepsilon_r} \sum\limits_{i=1}^n  \frac{q_i(x-x_i)}{\sqrt{(x-x_i)^2+(y-y_i)^2}^3}\\
E_y&=\frac{1}{4 \pi \varepsilon_0 \varepsilon_r} \sum\limits_{i=1}^n  \frac{q_i(y-y_i)}{\sqrt{(x-x_i)^2+(y-y_i)^2}^3}
\end{align*}
Feldlinien verlaufen tangential zu $\vec{E}$.
\begin{equation*}
\frac{\text{d}y}{\text{d}x}=\frac{E_y}{E_x}
\end{equation*}
Dies ist eine Differentialgleichung 1.~Ordnung.

Es folgt ein Beispiel mit dem Befehl \verb!\psplotDiffEqn! zum Zeichnen der Feldlinien:
\begin{verbatim}
\pstVerb{%
/q1 1 def
/q2 -0.5 q1 mul def
/xA 1.8 def
}

\multido{\rx=-250+10.2}{50}{%
\psplotDiffEqn[%
linewidth=0.25pt,%
linecolor=red,%
varsteptol=.001,%
method=rk4,%
algebraic,
plotpoints=200%
]{-20}{20}{\rx}{%
(q1*(y[0])/(sqrt((x+xA)^2+(y[0])^2))^3+q2*(y[0])/(sqrt((x-xA)^2+(y[0])^2))^3)%
/%
(q1*(x+xA)/(sqrt((x+xA)^2+(y[0])^2))^3+q2*(x-xA)/(sqrt((x-xA)^2+(y[0])^2))^3)%
}%
}
\pscircle*(!xA 0){0.25}\pscircle*(!xA neg 0){0.25}
\end{verbatim}



\textbf{Elektrisches Potential}

Das elektrische Potential $V$ ist gegeben durch:
\begin{equation}
\vec{E}=\begin{pmatrix}
E_x\\E_y
\end{pmatrix}=-\text{grad}\, V=-\nabla V =-\begin{pmatrix}
\frac{\partial V}{\partial x}\\[4pt]
\frac{\partial V}{\partial y}
\end{pmatrix}
\end{equation}
oder
\begin{align*}
E_x=-\frac{\partial V}{\partial x}\\
E_y=-\frac{\partial V}{\partial y}
\end{align*}

\textbf{\"{A}quipotentiallinien}

\begin{equation*}
V=\text{Const}
\end{equation*}

\"{A}quipotentiallinien stehen stets senkrecht auf den Feldlinien.

\begin{equation*}
\frac{\text{d}y}{\text{d}x}=-\frac{E_x}{E_y}
\end{equation*}
Dies ist eine Differentialgleichung 1.~Ordnung. Man benutzt erneut: \verb!\psplotDiffEqn! um die \"{A}quipotentiallinien zu zeichnen.
\begin{verbatim}
\pstVerb{%
/q1 1 def
/q2 1 q1 mul def
/xA 3.25 def
}
\multido{\rx=-4.1+0.75}{20}{%
\psplotDiffEqn[%
linewidth=0.85pt,%
linecolor=blue,%
varsteptol=.00001,%
method=rk4,%
algebraic,
plotpoints=300%
]{-6}{6}{\rx}{%
-((q1*(x+xA)/(sqrt((x+xA)^2+(y[0])^2))^3+q2*(x-xA)/(sqrt((x-xA)^2+(y[0])^2))^3))
/
(q1*(y[0])/(sqrt((x+xA)^2+(y[0])^2))^3+q2*(y[0])/(sqrt((x-xA)^2+(y[0])^2))^3)%
}%
}
\end{verbatim}
Hier ein vollst\"{a}ndiges Beispiel: \url{http://tug.org/mailman/htdig/pstricks/2010/007468.html}

Dies ist eine einfache Methode, jedoch mit einem nicht befriedigendenden Resultat, was mit eine 
Motivation war, dieses Paket zu entwickeln.


\section{Feldlinien}
Das Zeichnen der Feldlininen wird mit dem Befehl \Lcs{psElectricfield}\OptArgs\ aufgerufen.  
Dieser besitzt folgende Parameter:
\begin{enumerate}
  \item Die Ladungen, ihre Ortskoordinaten und die Anzahl der Linien, die von jeder einzelnen 
  ausgeht (oder bei ihr endet) werden mit mit demselben Parameter aufgerufen 
  $\mathsf{Q=\{[q_1\, x_1\, y_1\, N_1] [q_2\, x_2\, y_2\, N_2]\ldots[q_i\, x_i\, y_i\, N_i]
    \ldots [q_n\, x_n\, y_n\, N_n]\}}$. Die Anzahl der Linien ist hierbei optional -- wenn diese 
    Angabe weggelassen wird, wird ein vordefinierter Wert \textsf{N=19} genommen, der sich aus $360/18=20^\circ$ 
    ergibt (zwischen zwei Feldlinien, die von jeder einzelnen Ladung ausgeht oder dort endet).
  \item Die Farbe und Linienst\"{a}rke kann mit den g\"{a}ngigen Parametern von PSTricks 
  gesetzt werden: \textsf{linecolor} und \textsf{linewidth}.
  \item Die Anzahl der Berechnungspunkte einer jeden Linie ist vordefiniert mit \textsf{points=400} 
  und die Schrittweite ist \textsf{Pas=0.025}. Sollten diese Voreinstellungen nicht optimal f\"{u}r 
  eine Zeichnung sein, dann muss man sie \"{a}ndern.
  \item Die Position eines Pfeils auf einer Feldlinie kann mit dem Parameter \textsf{posArrow=0.25} 
  gesetzt werden, der das Verh\"{a}ltnis der Punktanzahl angibt, jeweils beginnend bei der Ladung.
\end{enumerate}
\section{\"{A}quipotentiallinien}
Die \"{A}quipotentiallinien werden mit folgendem Befehl gezeichnet: 
\verb+\psEquipotential[options](xmin,ymin)(xmax,ymax)+. Die Option f\"{u}r die Ladungen 
\textsf{Q} ist dieselbe wie bei den Feldlinien, es ist jedoch \"{u}berfl\"{u}ssig~\textsf{N} anzugeben.
\begin{enumerate}
  \item Man muss den Maximal- und Minimalwert des Potential vorab berechnen: \textsf{Vmax=3} und 
  \textsf{Vmin=-1} sind die voreingestellten Werte.
  \item Intervall zwischen zwei Werten des Potentials \textsf{stepV=0.5}, dies bestimmt die Anzahl der 
  \"{A}quipotentiallinien.
  \item Die Farbe und Linienst\"{a}rke kann mit den g\"{a}ngigen Parametern von PSTricks gesetzt werden: 
  \textsf{linecolor} und \textsf{linewidth}.
\end{enumerate}

\clearpage
\section{Beispiele}
\xLcs{psElectricfield}\xLcs{psEquipotential}
\begin{PSTexample}[pos=t]
\begin{pspicture*}(-6,-6)(6,6)
\psframe*[linecolor=lightgray!50](-6,-6)(6,6)
\psgrid[subgriddiv=0,gridcolor=gray,griddots=10]
\psElectricfield[Q={[-1 -2 2][1 2 2][-1 2 -2][1 -2 -2]},linecolor=red]
\psEquipotential[Q={[-1 -2 2][1 2 2][-1 2 -2][1 -2 -2]},linecolor=blue](-6.1,-6.1)(6.1,6.1)
\psEquipotential[Q={[-1 -2 2][1 2 2][-1 2 -2][1 -2 -2]},linecolor=green,linewidth=2\pslinewidth,Vmax=0,Vmin=0](-6.1,-6.1)(6.1,6.1)
\end{pspicture*}
\end{PSTexample}

\xLkeyword{Q}\xLkeyword{Vmin}\xLkeyword{Vmax}
\begin{PSTexample}[pos=t]
\begin{pspicture*}(-6,-6)(6,6)
\psframe*[linecolor=lightgray!50](-6,-6)(6,6)
\psgrid[subgriddiv=0,gridcolor=gray,griddots=10]
\psElectricfield[Q={[-1 -2 2 false][1 2 2 false][-1 2 -2 false][1 -2 -2 false]},radius=1.5pt,linecolor=red]
\psEquipotential[Q={[-1 -2 2][1 2 2][-1 2 -2][1 -2 -2]},linecolor=blue](-6,-6)(6,6)
\psEquipotential[Q={[-1 -2 2][1 2 2][-1 2 -2][1 -2 -2]},linecolor=green,linewidth=2\pslinewidth,Vmax=0,Vmin=0](-6.1,-6.1)(6.1,6.1)
\end{pspicture*}
\end{PSTexample}


\xLkeyword{Pas}\xLkeyword{points}\xLkeyword{posArrow}\xLkeyword{N}
\begin{PSTexample}[pos=t]
\begin{pspicture*}(-5,-5)(5,5)
\psframe*[linecolor=lightgray!40](-5,-5)(5,5)
\psgrid[subgriddiv=0,gridcolor=lightgray,griddots=10]
\psElectricfield[Q={[-1 -3 1][1 1 -3][-1 2 2]},N=9,linecolor=red,points=1000,posArrow=0.1,Pas=0.015]
\psEquipotential[Q={[-1 -3 1][1 1 -3][-1 2 2]},linecolor=blue](-6,-6)(6,6)
\psEquipotential[Q={[-1 -3 1][1 1 -3][-1 2 2]},linecolor=green,Vmin=-5,Vmax=-5,linewidth=2\pslinewidth](-6,-6)(6,6)
\end{pspicture*}
\end{PSTexample}



\begin{PSTexample}[pos=t,vsep=5mm]
\psset{unit=0.75cm}
\begin{pspicture*}(-5,-5)(5,5)
\psframe*[linecolor=green!20](-5,-5)(5,5)
\psgrid[subgriddiv=0,gridcolor=lightgray,griddots=10]
\psElectricfield[Q={[1 -2 0][-1 2 0]},linecolor=red]
\psEquipotential[Q={[1 -2 0][-1 2 0]},linecolor=blue](-5,-5)(5,5)
\psEquipotential[Q={[1 -2 0][-1 2 0]},linecolor=green,Vmin=0,Vmax=0](-5,-5)(5,5)
\end{pspicture*}
\end{PSTexample}

\begin{PSTexample}[pos=t,vsep=5mm]
\psset{unit=0.75cm}
\begin{pspicture*}(-5,-5)(5,5)
\psframe*[linecolor=green!20](-5,-5)(5,5)
\psgrid[subgriddiv=0,gridcolor=lightgray,griddots=10]
\psElectricfield[Q={[1 -2 0][1 2 0]},linecolor=red,N=15,points=500]
\psEquipotential[Q={[1 -2 0][1 2 0]},linecolor=blue,Vmin=0,Vmax=20,stepV=2](-5,-5)(5,5)
\psEquipotential[Q={[1 -2 0][1 2 0]},linecolor=green,Vmin=9,Vmax=9](-5,-5)(5,5)
\end{pspicture*}
\end{PSTexample}

\begin{PSTexample}[pos=t,vsep=5mm]
\psset{unit=0.75cm}
\begin{pspicture*}(-10,-5)(6,5)
\psframe*[linecolor=lightgray!40](-10,-5)(6,5)
\psgrid[subgriddiv=0,gridcolor=lightgray,griddots=10]
\psElectricfield[Q={[600 -60 0 false][-4 0 0] },N=50,points=500,runit=0.8]
\psEquipotential[Q={[600 -60 0 false][-4 0 0]},linecolor=blue,Vmax=100,Vmin=50,stepV=2](-10,-5)(6,5)
\psframe*(-10,-5)(-9.5,5)
\rput(0,0){\textcolor{white}{\large$-$}}
\multido{\rA=4.75+-0.5}{20}{\rput(-9.75,\rA){\textcolor{white}{\large$+$}}}
\end{pspicture*}
\end{PSTexample}

\begin{PSTexample}[pos=t,vsep=5mm]
\psset{unit=0.75cm}
\begin{pspicture*}(-5,-5)(5,5)
\psframe*[linecolor=green!20](-6,-5)(6,5)
\psgrid[subgriddiv=0,gridcolor=lightgray,griddots=10]
\psElectricfield[Q={[1 -2 -2][1 -2 2][1 2 2][1 2 -2]},linecolor={[HTML]{006633}}]
\psEquipotential[Q={[1 -2 -2][1 -2 2][1 2 2][1 2 -2]},Vmax=15,Vmin=0,stepV=1,linecolor=blue](-6,-6)(6,6)
\end{pspicture*}
\end{PSTexample}

\begin{PSTexample}[pos=t,vsep=5mm]
\psset{unit=0.75cm}
\begin{pspicture*}(-5,-5)(5,5)
\psframe*[linecolor=green!20](-5,-5)(5,5)
\psgrid[subgriddiv=0,gridcolor=lightgray,griddots=10]
\psElectricfield[Q={[1 2 0][1 1 1.732][1 -1 1.732][1 -2 0][1 -1 -1.732][1 1 -1.732]},linecolor=red]
\psEquipotential[Q={[1 2 0][1 1 1.732 12][1 -1 1.732][1 -2 0][1 -1 -1.732][1 1 -1.732]},linecolor=blue,Vmax=50,Vmin=0,stepV=5](-5,-5)(5,5)
\end{pspicture*}
\end{PSTexample}

\begin{PSTexample}[pos=t,vsep=5mm]
\psset{unit=0.75cm}
\begin{pspicture*}(-5,-5)(5,5)
\psframe*[linecolor=green!20](-5,-5)(5,5)
\psgrid[subgriddiv=0,gridcolor=lightgray,griddots=10]
\psElectricfield[Q={[1 2 0][1 1 1.732][1 -1 1.732][1 -2 0][1 -1 -1.732][1 1 -1.732][-1 0 0]},linecolor=red]
\psEquipotential[Q={[1 2 0][1 1 1.732 12][1 -1 1.732][1 -2 0][1 -1 -1.732][1 1 -1.732][-1 0 0]},Vmax=40,Vmin=-10,stepV=5,linecolor=blue](-5,-5)(5,5)
\end{pspicture*}
\end{PSTexample}

\begin{PSTexample}[pos=t,vsep=5mm]
\psset{unit=0.75cm}
\begin{pspicture*}(-6,-5)(6,5)
\psframe*[linecolor=green!20](-6,-5)(6,5)
\psgrid[subgriddiv=0,gridcolor=lightgray,griddots=10]
\psElectricfield[Q={[1 -4 0][1 -2 0 12][1 0 0][1 2 0][1 4 0]},linecolor=red]
\psEquipotential[Q={[1 -4 0][1 -2 0][1 0 0][1 2 0][1 4 0]},linecolor=blue,Vmax=30,Vmin=0,stepV=2](-7,-5)(7,5)
\end{pspicture*}
\end{PSTexample}



\clearpage
\section{Liste der optionalen Argumente f\"ur \texttt{pst-electricfield}}

\xkvview{family=pst-electricfield,columns={key,type,default}}

\nocite{*}
\bgroup
\raggedright
\bibliographystyle{plain}
\bibliography{pst-electricfield-doc}
\egroup


\printindex




\end{document}
