%% $Id: pst-3dplot-doc.tex 15 2008-02-08 09:23:35Z herbert $
\listfiles
\documentclass[11pt,a4paper]{article}

\begin{filecontents}{data3D.Roessler}
2.0, 0.0, 0.0
1.9737089990827656, 0.31348417289942826, 0.024766924774833776
1.8955421029937392, 0.6286215677998942, 0.040068360866723524
1.7670706651447274, 0.9374105709682867, 0.04899535062052031
1.590723362917843, 1.231909925153328, 0.053481257255618045
1.3698988432765893, 1.5043858076910284, 0.05484157950815698
1.1089929624834642, 1.7474740837870646, 0.05405054060802597
0.8133700003182118, 1.9543472822476953, 0.051871025675487174
0.48929474661269506, 2.1188790687368217, 0.048905975991643626
0.14383482236825765, 2.2358007890268707, 0.04561506618192828
-0.21526210928252393, 2.3008455575942284, 0.042324015908839975
-0.5797140462526329, 2.3108757369567967, 0.03923987600327629
-0.9408715928640635, 2.263989804617317, 0.03647450352611535
-1.2899036444077485, 2.1596047939957033, 0.03407200270285084
-1.6179979190813427, 1.9985108763333657, 0.032034456961492486
-1.916571194474139, 1.7828952581841464, 0.030342125259363112
-2.177483349570605, 1.5163333562604153, 0.028967036053499737
-2.393249071003189, 1.203746100773438, 0.02788091910922922
-2.5572411971516513, 0.8513231435128283, 0.027059201419929343
-2.663879978198744, 0.4664126663883511, 0.02648269287557226
-2.708802925631934, 0.05737937874793611, 0.02613812515779725
-2.689010380642566, -0.366566854583293, 0.02601824308118291
-2.6029824533543895, -0.7955685089005423, 0.02612181932183788
-2.4507635874341873, -1.2193400325823622, 0.026453773503831633
-2.2340116946856416, -1.627404968883053, 0.02702548109576835
-1.9560095794567929, -2.009345350292753, 0.027855313387445277
-1.6216372226896099, -2.3550576916809978, 0.0289694270702017
-1.237304403725046, -2.6550095751380067, 0.030402801066270472
-0.8108440834310077, -2.900490621836054, 0.032200483552030916
-0.3513679296671862, -3.0838515925562944, 0.03441894582781
0.13091369384807075, -3.1987254529413782, 0.037127316997633816
0.6249040658632549, -3.2402244813488377, 0.04040805872135605
1.1188607672806745, -3.205107879829453, 0.044356285531949
1.6006599982584453, -3.0919148586449468, 0.04907639328314185
2.0580794189191347, -2.901058780196999, 0.0546739031971644
2.479094141310295, -2.6348786393672534, 0.06123953904374337
2.8521805025686247, -2.2976448885720346, 0.06882184942853445
3.1666224148830464, -1.8955173551046565, 0.07738493864933163
3.4128151677144074, -1.4364537367227772, 0.08675049123655466
3.5825610419485394, -0.9300679485338773, 0.09653017105594026
3.669349217681727, -0.38743858293399275, 0.1060669508996468
3.66860848741213, 0.1791307830717498, 0.1144200706612233
3.577915325329169, 0.7563953873937568, 0.12043944425028498
3.397134032103045, 1.330489785389733, 0.12296507773843657
3.1284647683004594, 1.8872735777852254, 0.12113969095590112
2.7763851169553395, 2.4126873166475082, 0.11474338069347088
2.347493200120509, 2.893102419451378, 0.10439221252377785
1.8502875359900435, 3.3156518363301752, 0.09145709971549322
1.2949340704036636, 3.6685348067552614, 0.07768692767531428
0.6930599799313174, 3.9412962989715026, 0.06469485290776422
0.057580017439228595, 4.125085518611976, 0.0535544883671607
-0.5974736410304217, 4.212896026809753, 0.0446737246961642
-1.2571594448404535, 4.199784163977488, 0.03793342580559009
-1.9058990041940127, 4.08305672575688, 0.032943967026847994
-2.527814237674464, 3.8624164257608915, 0.02926777122899019
-3.1071099498819583, 3.5400549870758162, 0.026539600559695866
-3.628477534411971, 3.120686978802803, 0.024497408997019937
-4.077499745532988, 2.6115208596257036, 0.022967312670967185
-4.441042526876828, 2.022166249356186, 0.02183745323024855
-4.70762377765629, 1.3644782840310743, 0.02103647475244307
-4.867750520480961, 0.6523413832730903, 0.020519623345750852
-4.914216423301318, -0.09860386913565528, 0.020260779880335346
-4.842351978806912, -0.8712856726498582, 0.020248271372651484
-4.6502202511419455, -1.6475713988268783, 0.020483040487167408
-4.338752026039519, -2.4086757864272217, 0.020978468312466414
-3.9118154036016124, -3.135598187701918, 0.02176162966335394
-3.376216299081582, -3.8095791632607905, 0.02287609989009088
-2.7416279233519663, -4.412565984627867, 0.02438674375662763
-2.020449069764279, -4.927676106927448, 0.0263873000082302
-1.2275929169104387, -5.339647448336166, 0.029012132811414647
-0.3802100602126169, -5.635264385846258, 0.032454389441284426
0.5026483767307007, -5.803748777659692, 0.03699417971765483
1.4004193818863429, -5.837106085766177, 0.043042551339674814
2.2914576917034366, -5.730417843942741, 0.05121029037211145
3.153492702626417, -5.482073360400736, 0.0624150579030452
3.9640955128490303, -5.093935743066385, 0.07804527614909976
4.701135335990834, -4.571440165114393, 0.10020069599445576
5.343202094338152, -3.9236257147151097, 0.1320161372093747
5.869975027597972, -3.1631057496848705, 0.17801831977245255
6.2625361048395805, -2.305983838099523, 0.244311599794074
6.503681987676289, -1.3717191201774368, 0.3380681890205393
6.57840462343104, -0.38292840523916694, 0.465331169945956
6.474884726908676, 0.6349281065446967, 0.6259266868842355
6.186450753049299, 1.654097487676016, 0.8054981916019673
5.714653774106046, 2.645500975148737, 0.9687649074913285
5.485153457789435, 3.015596510728925, 1.0166012832905595
5.339921215942925, 3.226909997062504, 1.0384587046280915
5.186682071136008, 3.434465541003078, 1.0551261961946548
5.025763751435938, 3.6379611848927222, 1.0660892490337464
4.85752084015581, 3.8371048422086753, 1.070925616446375
4.68233081773646, 4.031615242907043, 1.0693258338419298
4.500589406948257, 4.221222733824442, 1.0611105450485159
4.3127053604557695, 4.4056699109614295, 1.04624341462459
4.119094870078487, 4.584712065919897, 1.024838603272049
3.920175808449217, 4.758117435522143, 0.9971620808818797
2.9858392075994002, 5.452943762343482, 0.812612270361229
1.9873096085271849, 6.017158252509305, 0.5761300699678681
0.9483591098403636, 6.4385215920732435, 0.35692931483850815
-0.11455517619792031, 6.707528542956261, 0.19734838656187356
-1.1872791673771168, 6.816712688735985, 0.10256792576424685
-2.253446438874668, 6.760732255312321, 0.055184392730956905
-3.2928943998362246, 6.536949843632011, 0.03431243820247646
-4.2825078678305575, 6.14607057676533, 0.025497726306147724
-5.197910469921709, 5.592581854725916, 0.021436720358505936
-6.014973211129965, 4.884947449140856, 0.01918889343414334
-6.710904523262051, 4.0356034038898, 0.017735990844505094
-7.265043812276318, 3.0608075975441884, 0.01673821309724795
-7.6594970491088885, 1.980373355821153, 0.016067973381140088
-7.879680210322718, 0.8173022776918248, 0.015667182591688332
-7.914786129315629, -0.4026740342570332, 0.015507722703657849
-7.758170400659039, -1.6516339387966874, 0.015580350992863473
-7.407646979008621, -2.9000765303331484, 0.01589179881627153
-6.865684276339126, -4.117590498576616, 0.016465388313506357
-6.13949432143193, -5.2735666277557565, 0.017344502041150147
-5.241009918259896, -6.337938044106181, 0.0185996519020146
-4.1867474992434826, -7.281931234281177, 0.020341131584772302
-2.9975565059563873, -8.078810200283934, 0.02274125667756507
-1.6982597726010134, -8.704595969360401, 0.0260741453349296
-0.3171939084766986, -9.138744160322855, 0.03078926449201528
1.1143349960221263, -9.364764641047929, 0.03765316948011792
2.562654596930381, -9.370769958811845, 0.04803576908048059
3.992342511395334, -9.149944248777466, 0.06451815290928255
5.3667650121129995, -8.700934303634424, 0.09225064072787606
6.648340617555662, -8.028185919678632, 0.14213766833161504
7.798060636152677, -7.142298355161241, 0.23861539188590877
8.773105938261706, -6.06059156350814, 0.4390837635074791
9.519689894888124, -4.808384473188431, 0.8817804579958542
9.715556843103172, -4.329500150199969, 1.1494809094253977
9.81773413178288, -4.016829077379845, 1.3658701874398937
9.900541029238141, -3.6987075313296836, 1.6268138317999121
9.962089675717698, -3.3758098556869536, 1.9410504302689513
10.00018948311941, -3.04888662875504, 2.31855764021317
10.012308012691376, -2.718776529596604, 2.770448931895757
9.995538942646686, -2.3864195564228607, 3.3086837048966977
9.946585194527254, -2.052871225277392, 3.94550210249929
9.861768678171382, -1.7193170331617214, 4.692479619636648
9.73708186383974, -1.387085996749398, 5.559093591932963
9.568299782347848, -1.0576614731948493, 6.550718962065328
9.351172749338911, -0.7326867662753687, 7.666044771925456
9.08171801634688, -0.4139623126097994, 8.894047024693492
8.756619968751496, -0.10343070822988398, 10.21087893622761
8.373730800484582, 0.19685424764189477, 11.577326972515616
7.932635745107376, 0.48477759808564364, 12.93775701555075
7.435211636522119, 0.7582336234791117, 14.221599174082469
6.886073311702125, 1.0152147309642172, 15.34821078463603
6.292783529977407, 1.253912865084775, 16.235280564528697
5.665714946560136, 1.4728231700121752, 16.809840791003843
5.017506967463606, 1.670836476399769, 17.019770020196443
4.362149705724899, 1.8473065769510475, 16.842950446240167
3.713825877381265, 2.0020810431825202, 16.291491858477364
3.0857122439713383, 2.1354902283238073, 15.409734893129361
2.488955133350004, 2.2482965651157674, 14.266615196739622
1.9319838250982082, 2.3416130942444733, 12.944584694656417
1.420233020012231, 2.4168044221813445, 11.527975739133392
0.956247998331239, 2.4753841494682987, 10.09331307070606
0.5400756034519507, 2.518920634293532, 8.703000830901596
0.16981526138056538, 2.5489590062894587, 7.402615450102852
-0.15778744089136104, 2.5669630387768843, 6.221163204496736
-0.446791500247129, 2.5742768833754552, 5.17327095280853
-0.7016055965842285, 2.572104267210342, 4.262291727847539
-0.9266456137826842, 2.5615015641386965, 3.483545485219481
-1.126100478315251, 2.543380911374715, 2.827214855473147
-1.3037900140486127, 2.5185199007406154, 2.280674836946092
-1.4630940545519222, 2.487575018607852, 1.8302151859844284
-1.6069325141944029, 2.45109672027427, 1.4622163609123322
-1.7377790868385545, 2.409544675056006, 1.1638834278782881
-1.9835141009235395, 2.3051188345791296, 0.7072590640032498
-2.3887006417654564, 2.0322388292260882, 0.2504419005182213
-2.7029450991651705, 1.694574861067702, 0.095879723140791
-2.9461595995808962, 1.3024699225902538, 0.046384648911208894
-3.120441752059532, 0.8651045444426645, 0.03102568413623744
-3.2226426402441297, 0.39212482135232124, 0.02628376880301836
-3.2490126324449267, -0.10594257459317138, 0.02482230257421722
-3.1968404376558923, -0.617718546647487, 0.02445966059229445
-3.065039502462036, -1.1311616703495304, 0.02457363465961232
-2.8543708677787727, -1.6338232811282312, 0.024998893728533096
-2.567525196349059, -2.113133158213107, 0.025704902922585416
-2.2091308259491678, -2.5567008953635995, 0.026705749958042275
-1.785706750424512, -2.952623596453139, 0.028036552844253267
-1.305566047862898, -3.2897919355609506, 0.02974888148444629
-0.778672311256114, -3.558186939832671, 0.03191197936863889
-0.2164516650596476, -3.749159998455242, 0.034616199441869416
0.3684362170460448, -3.85568883233519, 0.037977611037885356
0.9623636917766951, -3.872602545102222, 0.04214318386540144
1.5510320089789205, -3.7967694257303495, 0.04729557974764842
2.1197934352965464, -3.627241869327315, 0.05365555176228012
2.653992246923785, -3.365353587697903, 0.061478085033157916
3.13931832811565, -3.0147651331161143, 0.07103549757493581
3.562167899798263, -2.5814545673411113, 0.08257689459834969
3.910007397782812, -2.073650761579123, 0.0962499172326951
4.171738387102823, -1.5017072141949206, 0.11197133881160794
4.338062453621372, -0.8779144330251918, 0.12924497912729804
4.40184274907451, -0.21624914586490585, 0.14695809594983122
4.358449680500153, 0.46794032378242034, 0.16324440859893477
4.206059703699751, 1.1583098991424128, 0.17556317806347035
3.945852221923799, 1.837947349374204, 0.1811491492776709
3.582035878130457, 2.4898318659047374, 0.17785623663471162
3.1216564558450273, 3.0972954319275137, 0.16513442617814475
2.5742071345935607, 3.6444521233009963, 0.1446079484787909
1.9511516041571946, 4.116570803965888, 0.11976405507229838
1.265515665777983, 4.50038789165048, 0.09475388419321912
0.5316507955859152, 4.784378524644285, 0.07293467873241884
-0.23485145287864528, 4.9590119844316565, 0.05597412570263211
-1.0172176600385359, 5.017005746945389, 0.0438864063753544
-1.7976520000335394, 4.9535717414511655, 0.035710505226950406
-2.5576493535039337, 4.7666332999592615, 0.030257246575309142
-3.278432887475822, 4.456989159788667, 0.026552577356675305
-3.9414535770933643, 4.028408073527033, 0.02395081115268852
-4.528900136550702, 3.487646429671672, 0.022071478953300156
-5.024191706107253, 2.844387271156328, 0.02070024783607598
-5.412441335675305, 2.1111020142311916, 0.019715135865874634
-5.6808833433057995, 1.3028375909386654, 0.01904461478416901
-5.819257434483045, 0.4369328965050377, 0.01864674433575866
-5.820141438919434, -0.4673302493523888, 0.01849933187934492
-5.679224237377866, -1.3891349378201778, 0.01859553006488194
-5.395511002335657, -2.3065876101821967, 0.018942383450095683
-4.971454009298383, -3.1972174926703314, 0.01956141809463142
-4.4130037914876645, -4.0385061693137025, 0.02049116364319508
-3.729577201838434, -4.8084354074449545, 0.021792033072380295
-2.9339409597276824, -5.486040590331391, 0.02355456279546037
-2.0420114756136036, -6.051956660154793, 0.025912871771244884
-1.0725741792652026, -6.488943374424333, 0.029066646146450676
-0.04692829799898515, -6.782376979650935, 0.03331753837237578
1.011533780954803, -6.92069617783731, 0.039130596376046725
2.077799392177286, -6.895791618779578, 0.04724010498213066
3.125741893881876, -6.703330299827618, 0.05883550970166927
4.128632855587868, -6.343009575327733, 0.07589290221665458
5.05961830871354, -5.818740669533577, 0.10176969660424362
5.892093622654209, -5.138769866823729, 0.14226192916779143
6.599875333110678, -4.315758780556568, 0.20741828746874774
7.157025294453949, -3.3668651961391745, 0.3143937050896343
7.53717722488323, -2.3138915913501545, 0.4910838146045407
7.712402886847818, -1.183583822195186, 0.7781381478768689
7.702595198900638, -0.42812350737966265, 1.0439443206254988
7.67446561506387, -0.16104903030612147, 1.1543319115086785
7.632948617868841, 0.10667973505044293, 1.2740675192698314
7.577702707241738, 0.37459113695156165, 1.4030914042130438
7.508409601786327, 0.6421985954099348, 1.5410859148352525
7.424784720783314, 0.9090014912274651, 1.6874212963397872
7.3265895117122435, 1.174486461988975, 1.8411028818789186
7.213645486787356, 1.4381291707465322, 2.000724000730259
7.08584965988093, 1.6993966058812497, 2.1644299556565616
6.943190871410281, 1.9577499567045393, 2.3298992018535043
6.785766265556659, 2.2126480877923727, 2.4943481856925693
6.613796961846651, 2.463551605430367, 2.654565938477766
6.427641769878991, 2.7099274723317506, 2.806983257762296
6.227807666162028, 2.9512540835319805, 2.947779016098424
6.014955722117556, 3.187026669889763, 3.0730228287014674
5.789901274794582, 3.4167628500132214, 3.178849201542928
5.55360738762086, 3.640008111750338, 3.261653809410884
5.307171059256714, 3.8563409761556864, 3.3182983596224775
5.051802181036778, 4.065377585263705, 3.3463073348055143
4.788795867846453, 4.266775464006313, 3.3440384999603214
4.519499421522313, 4.460236238021824, 3.3108099223278877
4.245275745765818, 4.645507141955697, 3.2469695421247855
3.9674654353598813, 4.82238122322642, 3.153898750303703
3.6873499468628186, 4.9906962275751825, 3.0339482434427665
3.406118191349739, 5.150332236745906, 2.890311600159902
3.12483857935146, 5.301208206565302, 2.7268484372224377
2.8444380385334895, 5.443277617587807, 2.547873693097576
2.5656888886968816, 5.576523494457237, 2.357931944080772
2.2892037827722778, 5.70095307113798, 2.1615755320945222
2.0154382904471495, 5.81659237711927, 1.963162963248748
1.744700180423593, 5.923480997223305, 1.7666901373964123
1.4771640906470407, 6.021667219347792, 1.5756622622967296
1.2128900784178058, 6.111203735988321, 1.3930095469083668
0.9518445049863615, 6.192144012491572, 1.221045564426458
0.6939218051735664, 6.2645393828484, 1.0614639164553341
-0.13367115162002785, 6.442788205774359, 0.6342840983063279
-1.2100627320994148, 6.539796844326074, 0.2835219482191382
-2.2500752578957264, 6.47330114030099, 0.11884132977988426
-3.250285613906829, 6.2439718165630005, 0.053540783798716775
-4.196135046574074, 5.853976547089403, 0.030630197826803138
-5.06721808478353, 5.30836764538657, 0.022794185472664354
-5.841426690446052, 4.615743824149262, 0.01969558880997713
-6.49731251295517, 3.7884251060162706, 0.018090568801295958
-7.015335309385255, 2.8423586127367786, 0.01708190414542496
-7.378614942584927, 1.796869599135022, 0.016424502946090293
-7.573483778303709, 0.6743040965073674, 0.016040015635774094
-7.589935824875993, -0.5004180261965485, 0.015898292204984368
-7.421990842028393, -1.7003280993067817, 0.015990877115598788
-7.067969634538142, -2.8970006812322575, 0.016325589098541916
-6.53067286346075, -4.061200110703754, 0.01692684933937439
-5.817456614184374, -5.163567557246352, 0.017839167606383377
-4.940200121096248, -6.1753331992821865, 0.019134209420818293
-3.9151636942566634, -7.06903715332754, 0.02092332395624553
-2.762737912421797, -7.819242185681628, 0.023379365424123735
-1.507088639344538, -8.403221137138818, 0.026775335282775335
-0.17570671939801683, -8.801602493562855, 0.03155497686356822
1.2011228608846187, -8.998958841781647, 0.038466861583705506
2.5909080118789567, -8.984325476195657, 0.04883039563280974
3.959550464832346, -8.751641095998316, 0.06508832904145288
5.27187993096682, -8.300111543914891, 0.0920082395186199
6.491952262475553, -7.63451620969267, 0.13940840432511697
7.582728741990322, -6.7655182495985535, 0.22855155359934456
8.504235695667463, -5.710136660241023, 0.4073621076943347
9.20811152798848, -4.492766245374182, 0.7858121391867329
9.451165302102465, -3.855239780635789, 1.1055274280121736
9.538728049759913, -3.5487824808305026, 1.3015294428148347
9.60797725252403, -3.2374138882555825, 1.5351398887405352
9.657320484583641, -2.921770035363618, 1.8130777525095638
9.684939493127859, -2.602551239693463, 2.1428510382632773
9.688766889588388, -2.280530909880685, 2.5326290812217196
9.666469882855933, -1.956565122748488, 2.990978057503707
9.61544677823356, -1.6316026346257664, 3.52640607886616
9.53284394188098, -1.3066947525309012, 4.146658713150576
9.415602945542282, -0.9830041806268315, 4.857709656231626
9.260549196207823, -0.6618115798446256, 5.6624125110823424
9.064533744364885, -0.34451816086358333, 6.558827879034662
8.8246380431223, -0.032642233174231594, 7.5383240751182985
8.538445872813158, 0.27219263997712795, 8.583672245477596
8.204376274793441, 0.5682802288555961, 9.66750511551185
7.822055816070618, 0.8538668569942097, 10.751646442575037
7.392689279428867, 1.127200064773404, 11.78788076034458
6.919369017298653, 1.3865897566594572, 12.720638616816624
6.407251476286789, 1.6304785279003384, 13.491757456852552
5.863532687984796, 1.857515335339181, 14.046950117615348
5.297178484840663, 2.066624596955634, 14.342995599670488
4.718409083036545, 2.2570619159593073, 14.354193290377983
4.137991836306676, 2.428448492773784, 14.076544122017214
3.56644370491083, 2.580779005625226, 13.528566442022477
3.013269128713752, 2.7144017827432787, 12.748512712193106
2.486350177342758, 2.8299744395309956, 11.788705787494589
1.991567428006029, 2.9284016773378174, 10.708389742557397
1.5326763062431523, 3.0107638187008097, 9.566664406997642
1.111412651787232, 3.0782446566456216, 8.41676250268043
0.7277667707293325, 3.1320656685540738, 7.302342672856885
0.38035226932532257, 3.1734312717389295, 6.25587062404158
0.06680152098068329, 3.203487294892622, 5.298728978606063
-0.21586367883078603, 3.2232927336407773, 4.442496444005661
-0.47091757368808984, 3.233803404870099, 3.6908327091840016
-0.7016852851618449, 3.2358653333210428, 3.0415177820901573
-0.9113670530219425, 3.2302154707950024, 2.4883466894435937
-1.1029238948033526, 3.2174874855420863, 2.0227212404575514
-1.2790130538063487, 3.198220698721507, 1.6348862215863795
-1.4419609777867257, 3.172870657692774, 1.3148241539969736
-1.5937627029203614, 3.1418202395065937, 1.0528572151807694
-2.101993274633046, 2.973898766660752, 0.43473201862887473
-2.5847810587134488, 2.6976998774296685, 0.15751879947904163
-2.9922994335070254, 2.3429048597644946, 0.06496206401319657
-3.330759433172893, 1.9181764558962058, 0.0358259346748069
-3.5958830026527364, 1.4324511357554033, 0.026842444524512437
-3.7808117289834917, 0.8958832654085912, 0.02395993409664841
-3.879094038380487, 0.32001719539312873, 0.022948452153386145
-3.8857944085147933, -0.28233369457373164, 0.02260953028177438
-3.797942906949124, -0.8973256865416054, 0.02262095594544851
-3.614749318231863, -1.5103803447064017, 0.022904985845180904
-3.3377133939902524, -2.1065141771427762, 0.023454955802998674
-2.9706673764613685, -2.6706905197148485, 0.024290828231074365
-2.519759241666018, -3.1881845090928103, 0.025449651451115808
-1.9933783513761, -3.6449523887178614, 0.026985842011006955
-1.402024457740835, -4.027996285751815, 0.028975119439600665
-0.7581219020613374, -4.325715526817441, 0.031520951831475415
-0.07578218002167188, -4.528235670683263, 0.0347637614190443
0.6294805305888491, -4.627706768970493, 0.03889358380592709
1.3410748905816168, -4.618562932289972, 0.0441669566516942
2.0416917106858463, -4.497736074326341, 0.05092843998819987
2.7136887905904636, -4.2648177080108765, 0.05963581740319053
3.339492536353866, -3.9221638231602576, 0.07088462246215012
3.902008102813728, -3.474939074113672, 0.08542038050110319
4.385032053143094, -2.931097539961895, 0.10411353479736482
4.77366655659884, -2.301297829232222, 0.12785123367156662
5.054742731741975, -1.598749759384982, 0.15727688939648132
5.217271726082073, -0.8389877036183063, 0.19230364240806544
5.252949055601218, -0.03956194347898131, 0.23138920020006068
5.156725206570757, 0.7803643398873721, 0.27075062838751673
4.927401732470295, 1.6005236962698655, 0.3040238056894974
4.5681094114996545, 2.400156334627911, 0.32313209997470504
4.08642003346071, 3.1586915428744136, 0.32085409664722125
3.4938552045651723, 3.85642307638721, 0.2944172416116539
2.8047930883993217, 4.475078864418605, 0.24797520140879956
2.035158163394719, 4.998212233635217, 0.19170451470456978
1.2014911005469464, 5.411421482549435, 0.13745835146538982
0.3207643012834232, 5.7024868102082795, 0.093765781759793
-0.5892213537916692, 5.8615335075263335, 0.0634523891313674
-1.5093806674510137, 5.881274687164368, 0.04472715433085047
-2.4192604593651406, 5.757307783540075, 0.033934121077673546
-3.297407191621176, 5.488397018788945, 0.0277537372422779
-4.122007543817511, 5.07668250066827, 0.024018844302520725
-4.871596642526892, 4.527787604766935, 0.021575384626606506
-5.525725119205101, 3.850820720933396, 0.01987564499473809
-6.0655606117591745, 3.058277173200811, 0.018665224236140972
-6.474430162203531, 2.1658481028149694, 0.0178183893367252
-6.738308740549882, 1.192142177853216, 0.017267501186505628
-6.846251134397938, 0.1583259896609392, 0.01697544581004047
-6.790759476974828, -0.9123099418886927, 0.016924936730755655
-6.568077210921361, -1.9948510467672929, 0.017114494585318413
-6.178400766600547, -3.063314263222376, 0.017557710692583973
-5.626001640417001, -4.091247480383695, 0.018284903709690423
-4.919253449813759, -5.0523596894206015, 0.019347338524572804
-4.070560759420412, -5.9211675688611285, 0.020824959267892528
-3.09618897094418, -6.673643467584464, 0.022839625640272938
-2.0159973671696005, -7.287849378925841, 0.025577628900753647
-0.8530806067698964, -7.744541587763247, 0.029328647324738187
0.36667213111382246, -8.02773131734615, 0.034555018767795334
1.6150858833262367, -8.125188076474256, 0.04201901687999983
2.8622538906492307, -8.02887485190668, 0.053025034152899976
4.0770997333103365, -7.735308526410829, 0.06989680817971138
5.227882045487002, -7.245846501106422, 0.09694840842876788
6.282516848871196, -6.566914925448461, 0.14250972076899507
7.208466697162102, -5.710222731436574, 0.22319936228590626
7.971688582343952, -4.6930644163752655, 0.3728132314695582
8.533670662305425, -3.5389310473445112, 0.6596810992783806
8.792131331877295, -2.6120979670690505, 1.0353870384931827
8.839573646579051, -2.320764699318184, 1.1908713629857248
8.87090257978366, -2.0259984600256704, 1.3704663272013125
8.885109914401811, -1.7283585053857884, 1.5772735606251977
8.881097244698829, -1.4284450082791116, 1.814485823966621
8.857676470066595, -1.12690251875592, 2.085253163094494
8.813576067259218, -0.8244233360033532, 2.3924916148919917
8.747455325706813, -0.5217505509037185, 2.738624452728525
8.657929020806591, -0.21968043442758103, 3.125249309712768
8.543605113618995, 0.08093624479195093, 3.5527312692259185
8.403137864997262, 0.37919448183198623, 4.019733159804797
8.235298072199091, 0.6741374611037374, 4.522710410494463
8.039060806729712, 0.9647617744429308, 5.055418494113888
7.8137089270703575, 1.2500260430441696, 5.608504079115894
7.558947747932823, 1.5288634497039384, 6.169272034614508
7.275022760901465, 1.800198469238202, 6.721732355518512
6.962828693048845, 2.0629677383684886, 7.247025158543007
6.623995237931523, 2.316144539460267, 7.724290020767319
6.260933494331983, 2.558765824549344, 8.1319843419636
5.876828489402942, 2.789960152414162, 8.44956895026618
5.475567766354555, 3.0089744573011714, 8.659383848476475
5.061603759178425, 3.2151973290505755, 8.748458455059811
4.639757459946976, 3.4081765562191655, 8.709967175545291
4.214980731856892, 3.5876291076031235, 8.544072459209525
3.792102169321868, 3.7534424667056783, 8.257994651697016
3.3755846577409225, 3.905667172348013, 7.865289303919939
2.969320826577103, 4.044501387296867, 7.384459978932202
2.576485937968906, 4.1702691375708065, 6.837147440922006
2.1994581655187235, 4.283394400083416, 6.24618676906171
1.8398060496747326, 4.384373401081114, 5.633806716423572
1.4983343389660104, 4.473747339258949, 5.02017580632379
1.1751738331607897, 4.55207734628084, 4.422404448833488
0.8698986325393181, 4.619922956552616, 3.8540194272411386
0.5816549175960387, 4.677824789407899, 3.324855934686289
0.30928809812589264, 4.726291637548075, 2.8412710676237256
0.051458835632222436, 4.765791757613755, 2.4065696778141796
-0.19325781671728326, 4.796747890208006, 2.0215413772617996
-0.42629267470022375, 4.819535390028161, 1.6850272666076986
-0.6490221440199537, 4.834482799536002, 1.3944587876822787
-0.8627258255433654, 4.841874223821472, 1.1463336076973218
-1.2954600223281432, 4.8333523089052575, 0.7384824684946624
-2.110903216184799, 4.7165768459672535, 0.2800581822396751
\end{filecontents}



\usepackage{ccfonts}
\usepackage[euler-digits]{eulervm}
\usepackage[T1]{fontenc}
\usepackage[latin1]{inputenc}
\usepackage[scaled=0.85]{luximono}
\usepackage{amsmath}
\usepackage{graphicx}
\usepackage{geometry}
\usepackage{pstricks}
\usepackage{pst-grad}
\usepackage{showexpl}
\usepackage{tabularx}
\lstset{wide=true}
\usepackage{pst-3dplot}
\let\myFV\fileversion
\usepackage{url}
\usepackage[bottom]{footmisc}
\usepackage{fancyhdr}
\usepackage{abstract}
\usepackage{calc}
\usepackage{comment}
\usepackage{prettyref}
%\usepackage[scaled=0.8]{luximono}
\pagestyle{fancy}
\usepackage{url}
\usepackage{longtable}
\def\UrlFont{\small\ttfamily}
\makeatletter
\def\verbatim@font{\small\normalfont\ttfamily}
\makeatother
\def\Lcs#1{{\ttfamily\textbackslash #1}}
\lfoot{\small\ttfamily\jobname.tex}
\cfoot{Documentation}
\rfoot{\thepage}
\lhead{PSTricks}% package \texttt{pst-3dplot}}
\renewcommand{\headrulewidth}{0pt}
\renewcommand{\footrulewidth}{0pt}
\let\myPart\part
\renewcommand{\part}[1]{\clearpage\myPart{#1}}
\newcommand{\PS}{PostScript}
\newcommand\CMD[1]{\texttt{\textbackslash#1}}
\newcommand\verbI[1]{\small\texttt{#1}}
\psset{subgriddiv=0,gridlabels=7pt,gridcolor=black!15}
\usepackage[pdfauthor={Herbert Voss},pdftitle={3D Plots},linktocpage,colorlinks]{hyperref}
%

%
\begin{document}
\author{Herbert Vo�\thanks{hv@pstricks.de}\\[3pt]
\small with contributions from Darrell Lamm\thanks{darrell.lamm@gtri.gatech.edu}}
\date{\today}
\title{3D plots: PST-3dplot v\myFV\\Documentation}

\maketitle%
\begin{abstract}
The well known \verbI{pstricks} package offers excellent macros to insert more or less complex 
graphics into a document. \verbI{pstricks} itself is the base for several other additional packages, 
which are mostly named \verbI{pst-xxxx}, like \verbI{pst-3dplot}.

There exist several packages for plotting three dimensional graphical objects. \verbI{pst-3dplot} 
is similiar to the \verbI{pst-plot} package for two dimensional objects and mathematical functions.

This version uses the extended keyval package \verb+xkeyval+, so be sure that you have installed
this package together with the spcecial one \verb+pst-xkey+ for PSTricks. The \verb+xkeyval+
package is available at \href{ftp://ftp.dante.de/pub/tex/macros/latex/contrib/xkeyval/}{CTAN:/macros/latex/contrib/xkeyval/}.
It is also important that after \verb+pst-3dplot+ no package is loaded, which uses the old keyval interface.
\vspace*{2cm}
\end{abstract}%

\clearpage
\tableofcontents

\clearpage
\section{The Parallel projection}

%\psset{coorType=1}
Figure \ref{Abb0-1} shows a point $P(x,y,z)$ in a three dimensional coordinate system ($x,y,z$) 
with a transformation into $P^*(x^*,y^*)$, the Point in the two dimensional system ($x_E,y_E$).

\begin{figure}[htb]
\centering
\unitlength1cm
\begin{picture}(10,9)
\thicklines
\put(0,8){\makebox(0,0)[l]{$\alpha$: horizontal rotating angle}}
\put(0,7.5){\makebox(0,0)[l]{$\beta$: vertikal rotating angle}}
%
\put(5.25,9){\makebox(0,0){z}}
\put(5,0){\vector(0,1){9}}
\put(6,5.5){\vector(-2,-1){5.5}}
\put(9.5,3){\makebox(0,0){y}}
\put(4,5.5){\vector(+2,-1){5.5}}
\put(0.5,3){\makebox(0,0){x}}
\thinlines
\put(0,5){\vector(+1,0){10}}
\put(5,5.02){\line(+1,0){1}}
\put(10,4.5){\makebox(0,0){$x_E$}}
\put(5.25,8.5){\makebox(0,0){$y_E$}} 
\put(5.3,4.5){\makebox(0,0){$\alpha$}}
\put(2.6,3.6){\makebox(0,0){$\alpha$}}
%
\put(2,3.5){\line(+2,-1){5}}
\put(9,3){\line(-2,-1){3.5}}
\put(6,1.5){\line(0,1){5}}
\put(6,6.5){\circle*{0.2}}
\put(6.2,6.7){\makebox(0,0)[l]{$P(x,y,z)$}}
\put(6.2,6.3){\makebox(0,0)[l]{$P^*(x^*,y^*)$}}
\put(6.2,5.3){\makebox(0,0)[l]{$x^*$}}
%
\put(5,3){\line(+1,0){4}}
\put(7,2.7){\makebox(0,0){$y\cdot\sin\alpha$}}
\put(2,3.5){\line(+1,0){3}}
\put(3.5,3.2){\makebox(0,0){$x\cdot\cos\alpha$}}
\thicklines
\put(5,1.5){\line(+1,0){1}}
\put(5,1.52){\line(+1,0){1}}
\thinlines
\put(4.2,4.8){\makebox(0,0){$\alpha$}}
\thicklines
\put(5.03,5){\line(0,-1){2}}
\put(5.1,4){\makebox(0,0)[l]{$y\cdot\cos\alpha$}}
\put(4.97,5){\line(0,-1){1.5}}
\put(4.9,4.2){\makebox(0,0)[r]{$x\cdot\sin\alpha$}} 
\thinlines
\put(5.5,0.5){\vector(0,+1){1}}
\put(5.5,0.5){\line(1,0){0.5}}
\put(6.2,0.5){\makebox(0,0)[l]{$y\cdot\sin\alpha-x\cdot\cos\alpha$}}
%\put(4.5,3.25){\vector(1,0){0.5}}
%\put(4.5,3.25){\line(-1,-2){1}}
%\put(3.5,1.25){\line(-1,0){0.5}}
\put(4.9,1.5){\makebox(0,0)[r]{$y\cdot\cos\alpha+x\cdot\sin\alpha$}}
\end{picture}%
\caption{Lengths in a three dimensional System}
\label{Abb0-1}
\end{figure}


The angle  $\alpha$  is the horizontal rotation with positive values for anti clockwise rotations 
of the 3D coordinates. The angle $\beta$  is the vertical rotation (orthogonal to the paper plane). 
In figure \ref{Abb0-2} we have $\alpha=\beta=0$. The y-axis comes perpendicular out of the paper 
plane. Figure \ref{Abb0-3} shows the same for another  angle with a view from the side, where the 
x-axis shows into the paper plane and the angle $\beta$ is greater than $0$ degrees.

\begin{figure}[htb]
\centering
\unitlength1cm
\begin{picture}(2,2.5)
\thicklines
\put(2.2,2){\makebox(0,0){$z$}}
\put(2,0.5){\vector(0,1){2}}
\put(2,0.5){\vector(-1,0){2}}
\put(0.5,0.7){\makebox(0,0){$x$}}
\put(2,0.5){\circle*{0.2}}
\put(2,0.5){\circle{0.5}}
\put(2.3,0.7){\makebox(0,0)[l]{$y$}}
\end{picture}
\caption{Coordinate System for $\alpha=\beta=0$ ($y$-axis comes out of the paper plane)}\label{Abb0-2}
\end{figure}


The two dimensional x coordinate  $x^*$ is the difference of the two horizontal lengths
$y\cdot\sin\alpha$ und $x\cdot\cos\alpha$ (figure \ref{Abb0-1}):
\begin{equation}
x^{*}=-x\cdot\cos\alpha+y\cdot\sin\alpha
\end{equation}


The z-coordinate is unimportant, because the rotation comes out of the paper plane, so we have 
only a different $y^*$ value for the two dimensional coordinate but no other $x^*$ value. 
The $\beta$ angle is well seen in figure  \ref{Abb0-3} which derives from figure \ref{Abb0-2}, 
if the coordinate system is rotated by $90�$ horizontally to the left and vertically by $\beta$ 
also to the left.

\begin{figure}[htbp]
\unitlength1cm
\centering
\begin{picture}(1,3)
\thicklines
\put(1.5,2.9){\makebox(0,0){z}}
\put(2,1){\vector(-1,2){1}}
\put(2,1){\vector(-2,-1){2}}
\put(0,0.3){\makebox(0,0){y}}
\put(2,1){\circle{0.5}}
\put(1.8,0.8){\line(1,1){0.4}}
\put(1.8,1.2){\line(1,-1){0.4}}
\put(2.3,1.2){\makebox(0,0)[l]{x}}
\thinlines
\put(2,1){\line(-1,0){2}}
\put(2,1){\line(0,1){2}}
\put(1.7,2){\makebox(0,0){$\beta$}}
\put(1.3,0.8){\makebox(0,0){$\beta$}}
\put(2,2.5){\line(-1,0){0.75}}
\put(2.2,2.5){\makebox(0,0)[l]{$z*_1=z\cdot\cos\beta$}}
\put(1,1){\line(0,-1){0.5}}
\put(1.3,0.5){\makebox(0,0)[l]{$y\cdot\cos\alpha+x\cdot\sin\alpha$}}
\put(0.9,0.7){\makebox(0,0)[r]{$-(y\cdot\cos\alpha+x\cdot\sin\alpha)\cdot\sin\beta$}}
\end{picture} 
\caption{Coordinate System for $\alpha=0$ and $\beta>0$ ($x$-axis goes into the paper plane)}\label{Abb0-3}
\end{figure}


The value of the perpendicular projected z coordinate is $z^{*}=z\cdot cos\beta$. With figure  \ref{Abb0-3} we see, that the point $P(x,y,z)$ runs on an elliptical curve when $\beta$ is constant and $\alpha$ changes continues. The vertical alteration of $P$ id the difefrence of the two {}``perpendicular'' lines  $y\cdot\cos\alpha$ and $x\cdot\sin\alpha$.
These lines are rotated by the angle  $\beta$, so we have them to multiply with  $\sin\beta$ to 
get the vertical part. We get the following transformation equations:

\begin{equation}
\begin{array}{lll}
  x_{E} & = & -x\cos\alpha+y\sin\alpha\\
  y_{E} & = & -(x\sin\alpha+y\cos\alpha)\cdot\sin\beta+z\cos\beta
\end{array}
\end{equation}

\noindent or written in matrix form:
{\footnotesize\addtolength{\arraycolsep}{-2pt}
\begin{equation}
\begin{pmatrix}x_E\\y_E\end{pmatrix}=
  \begin{pmatrix}
    -\cos\alpha & \sin\alpha & 0\\
    -\sin\alpha\sin\beta & -\cos\alpha\sin\beta & \cos\beta
  \end{pmatrix}\cdot
\begin{pmatrix}x\\y\\z\end{pmatrix}
\end{equation}%
\addtolength{\arraycolsep}{2pt}%
}

All following figures show a grid, which has only the sense to make things clearer.

\section{Options}
All options which are set with \verb|psset| are global and all which are passed with the optional 
argument of a macro are local for this macro. This is an important fact for setting the angles 
\verb|Alpha| and \verb|Beta|. Mostly all macro need these values, this is the reason why they 
should be set with \verb|psset| and not part of an optional argument.

\section{Coordinates}
\verb+pst-3dplot+ accepts cartesian or spherical coordinates. In both cases there
must be three parameters: \verb+(x,y,z)+ or alternatively ($r$,$\phi$,$\theta$),
where $r$ is the radius, $phi$ the longitude angle and $\theta$ the lattitude angle.
For the spherical coordinates set the option \verb+SphericalCoor=true+. Spherical coordinates
are possible for all macros where three dimensional coordinates are expected, except
for the plotting functions (math functions and data records). Maybe that this is also interesting
for someone, then let me know.


\section{Coordinate axes}

The syntax for drawing the coordinate axes is
{\footnotesize\begin{verbatim}
\pstThreeDCoor[<options>]
\end{verbatim}}

The only special option is \verb/drawing=true|false/, which enables the drawing of the
coordinate axes. The default is true. In nearly all cases the \verb|\pstThreeDCoor| macro
must be part of any drawing to initialize the 3d-system. If \verb+drawing+ is set to \verb+false+, then all ticklines options are also disabled.

Without any options we get the default view with the in table~\ref{tab:coor}
listed options with the predefined values.

{
\ttfamily
\begin{longtable}{lll}
\caption{All new parameters for \texttt{pst-plot}\label{tab:coor}}\\[-5pt]
\textrm{Name}         & \textrm{Type}  & \textrm{Default}\\\hline
\endfirsthead
\textrm{Name}         & \textrm{Type}  & \textrm{Default}\\\hline
\endhead
Alpha            & <angle>  & 45\\
Beta             & <angle>  & 30\\
xMin             & <value> & -1\\ %ok
xMax             & <value> & 4\\ %ok
yMin             & <value> & -1\\ %ok
yMax             & <value> & 4\\ %ok
zMin             & <value> & -1\\ %ok
zMax             & <value> & 4\\ %ok
nameX            & <string> &  \$x\$\\
spotX            & <angle>  & 180\\
nameY            & <string> &  \$y\$\\
spotY            & <angle>  & 0\\
nameZ            & <string> &  \$z\$\\
spotZ            & <angle>  & 90\\
IIIDticks        & false|true & false\\
Dx               & <value>  & 1\\
Dy               & <value>  & 1\\
Dz               & <value>  & 1\\
IIIDxTicksPlane  & xy|xz|yz & xy\\
IIIDyTicksPlane  & xy|xz|yz & yz\\
IIIDzTicksPlane  & xy|xz|yz & yz\\
IIIDticksize     & <value>  & 0.1\\
IIIDxticksep     & <value>  & -0.4\\
IIIDyticksep     & <value>  & -0.2\\
IIIDzticksep     & <value>  & 0.2\\
RotX             & <angle>  & 0\\
RotY             & <angle>  & 0\\
RotZ             & <angle>  & 0\\
RotAngle         & <angle>  & 0\\
xRotVec          & <angle>  & 0\\
yRotVec          & <angle>  & 0\\
zRotVec          & <angle>  & 0\\
RotSequence	 & xyz|xzy|yxz|yzx|zxy|zyx|quaternion & xyz\\
RotSet     	 & set|concat|keep & set\\
eulerRotation    & true|false & false\\
\end{longtable}
}


\begin{LTXexample}[width=6cm]
\begin{pspicture}(-3,-2.5)(3,4.25)
  \pstThreeDCoor
\end{pspicture}
\end{LTXexample}

There are no restrictions for the angles and the max and min values for the axes; all \verb|pstricks| 
options are possible as well. The following example changes the color and the width of the axes.

\medskip
\noindent\fbox{\parbox{\columnwidth-2\fboxsep}{The angles \verbI{Alpha} and \verbI{Beta} are 
important to all macros and should always be set with  \verbI{psset} to make them global to all 
other macros. Otherwise they are only local inside the macro to which they are passed.}}

\medskip
\verb+Alpha+ ist the horizontal and \verb+Beta+ the vertical rotation angle of the
Cartesian coordinate system. 

\lstset{wide=false}



\begin{LTXexample}[width=4.25cm]
\begin{pspicture}(-2,-1.25)(1,2.25)
  \pstThreeDCoor[linewidth=1.5pt,linecolor=blue,
    xMax=2,yMax=2,zMax=2,
    Alpha=-60,Beta=30]
\end{pspicture}
\end{LTXexample}



\begin{LTXexample}[width=4.25cm]
\begin{pspicture}(-2,-2)(2,2)
  \pstThreeDCoor[xMax=2,yMax=2,zMax=2]
\end{pspicture}
\end{LTXexample}


\begin{LTXexample}[width=4.25cm]
\begin{pspicture}(-2,-2)(2,2)
  \pstThreeDCoor[xMax=2,yMax=2,zMax=2,
     Alpha=30,Beta=60]
\end{pspicture}
\end{LTXexample}



\begin{LTXexample}[width=4.25cm]
\begin{pspicture}(-2,-2)(2,2)
  \pstThreeDCoor[xMax=2,yMax=2,zMax=2,
    Alpha=30,Beta=-60]
\end{pspicture}
\end{LTXexample}



\begin{LTXexample}[width=4.25cm]
\begin{pspicture}(-2,-2)(2,2)
  \pstThreeDCoor[
    xMax=2,yMax=2,zMax=2,
    Alpha=90,Beta=60]
\end{pspicture}
\end{LTXexample}



\begin{LTXexample}[width=4.25cm]
\begin{pspicture}(-2,-2)(2,2)
  \pstThreeDCoor[linewidth=1.5pt,
     xMax=2,yMax=2,zMax=2,
     Alpha=40,Beta=0]
\end{pspicture}
\end{LTXexample}


\lstset{wide=true}

\subsection{Ticks}
With the option \verb+IIIDticks+ the axes get ticks and labels. There are several 
options to place the labels in right plane to get an optimal view.  The view of the 
ticklabels can be changed by redefining the macro

\begin{verbatim}
\def\psxyzlabel#1{\bgroup\footnotesize\textsf{#1}\egroup}
\end{verbatim}



\psset{unit=1.25,gridlabels=0pt}
\begin{LTXexample}[width=7.25cm]
\begin{pspicture}(-3,-2.5)(3,4)
  \pstThreeDCoor[IIIDticks]%
  \pstThreeDPut(3,0,3){\Huge default}
\end{pspicture}
\end{LTXexample}

\begin{LTXexample}[width=7.25cm]
\begin{pspicture}(-3,-2.5)(3,4)
  \pstThreeDCoor[linecolor=black,%
     IIIDticks,xMin=-2,yMin=-2,zMin=-2]%
\end{pspicture}
\end{LTXexample}

\begin{LTXexample}[width=7.25cm]
\begin{pspicture}(-3,-2.5)(3,4)
  \pstThreeDCoor[linecolor=black,%
     IIIDticks,IIIDzTicksPlane=yz,IIIDzticksep=-0.2,%
     IIIDxTicksPlane=yz,,IIIDxticksep=-0.2,%
     IIIDyTicksPlane=xy,,IIIDyticksep=0.2,%
     Dx=2,Dy=1,Dz=0.25,Alpha=-135,Beta=-30]%
\end{pspicture}
\end{LTXexample}

The following example shows a wrong placing of the labels, the planes should be changed. 

\begin{LTXexample}[width=7.25cm]
\psset{Alpha=-60,Beta=60}
\begin{pspicture}(-4,-2.25)(1,3)
  \pstThreeDCoor[linecolor=black,%
      IIIDticks,Dx=2,Dy=1,Dz=0.25]%
\end{pspicture}
\end{LTXexample}

\begin{LTXexample}[width=7.25cm]
\psset{Alpha=-60,Beta=60}
\begin{pspicture}(-4,-2.25)(1,3)
  \pstThreeDCoor[linecolor=black,%
      IIIDticks,planecorr=normal,
      Dx=2,Dy=1,Dz=0.25]%
\end{pspicture}
\end{LTXexample}

\begin{LTXexample}[width=7.25cm]
\psset{Alpha=-60,Beta=60}
\begin{pspicture}(-4,-2.25)(1,3)
  \pstThreeDCoor[linecolor=black,%
      IIIDticks,planecorr=xyrot,
      Dx=2,Dy=1,Dz=0.25]%
\end{pspicture}
\end{LTXexample}


\subsection{Experimental features}
All features are as long as they are not really tested called experimental. With the optional
argument \texttt{coorType}, which is by default 0, one can change the the viewing of the axes
and all other three dimensional objects.

With \texttt{coorType=1} the y--z-axes are orthogonal and the angle between x- and y-axis
is \texttt{Alpha}. The angle \texttt{Beta} is not valid.
 
\begin{LTXexample}[width=9.75cm]
\psset{coorType=1,Alpha=135}
\begin{pspicture}(-2,-3)(3,3)
\pstThreeDCoor[IIIDticks,zMax=3]%
\end{pspicture}
\end{LTXexample}

With \texttt{coorType=2} the y--z-axes are orthogonal and the angle between x- and y-axis
is always 135 degrees and the x-axis is shortened by a factor of $1/\sqrt{2}$. 
The angle \texttt{Alpha} is only valid for placing the ticks, if any.  The angle \texttt{Beta} is not valid.

\begin{LTXexample}[width=9.75cm]
\psset{coorType=2,Alpha=90,
    IIIDxTicksPlane=yz}
\begin{pspicture}(-2,-2)(3,3)
\pstThreeDCoor[IIIDticks,zMax=3]%
\end{pspicture}
\end{LTXexample}

With \texttt{coorType=3} the y--z-axes are orthogonal and the angle between x- and y-axis
is always 45 degrees and the x-axis is shortened by a factor of $1/\sqrt{2}$. 
The angle \texttt{Alpha} is only valid for placing the ticks, if any.  The angle \texttt{Beta} is not valid.

\begin{LTXexample}[width=9.75cm]
\psset{coorType=3,Alpha=90,
    IIIDxTicksPlane=yz}
\begin{pspicture}(-2,-2)(3,3)
\pstThreeDCoor[IIIDticks,zMax=3]%
\end{pspicture}
\end{LTXexample}


\clearpage
\section{Rotation}

The coordinate system can be rotated independent from the given Alpha and Beta values. This makes it possible to place
the axes in any direction and any order. There are the three options \verb+RotX+, \verb+RotY+, \verb+RotZ+ and an
additional one for the rotating sequence, which can be any combination of the three letters \verb+xyz+.

\begin{LTXexample}[pos=t]
\begin{pspicture}(-6,-3)(6,3)
  \multido{\iA=0+10}{18}{%
    \pstThreeDCoor[RotZ=\iA,xMin=0,xMax=5,yMin=0,yMax=5,zMin=-1,zMax=3]%
  }
\end{pspicture}
\end{LTXexample}


\begin{LTXexample}[pos=t]
\psset{unit=2,linewidth=1.5pt,drawCoor=false}
\begin{pspicture}(-2,-1.5)(2,2.5)%
  \pstThreeDCoor[xMin=0,xMax=2,yMin=0,yMax=2,zMin=0,zMax=2]%
  \pstThreeDBox[RotX=90,RotY=90,RotZ=90,%
    linecolor=red](0,0,0)(.5,0,0)(0,1,0)(0,0,1.5)
  \pstThreeDBox[RotSequence=xzy,RotX=90,RotY=90,RotZ=90,%
    linecolor=yellow](0,0,0)(.5,0,0)(0,1,0)(0,0,1.5)
  \pstThreeDBox[RotSequence=zyx,RotX=90,RotY=90,RotZ=90,%
    linecolor=green](0,0,0)(.5,0,0)(0,1,0)(0,0,1.5)
  \pstThreeDBox[RotSequence=zxy,RotX=90,RotY=90,RotZ=90,%
    linecolor=blue](0,0,0)(.5,0,0)(0,1,0)(0,0,1.5)
  \pstThreeDBox[RotSequence=yxz,RotX=90,RotY=90,RotZ=90,%
    linecolor=cyan](0,0,0)(.5,0,0)(0,1,0)(0,0,1.5)
  \pstThreeDBox[RotSequence=yzx,RotX=90,RotY=90,RotZ=90,%
    linecolor=magenta](0,0,0)(.5,0,0)(0,1,0)(0,0,1.5)
  \pstThreeDBox[fillstyle=gradient,RotX=0](0,0,0)(.5,0,0)(0,1,0)(0,0,1.5)
  \pstThreeDCoor[xMin=0,xMax=2,yMin=0,yMax=2,zMin=0,zMax=2]%
\end{pspicture}%
\end{LTXexample}


\begin{LTXexample}[pos=t]
\begin{pspicture}(-2,-1.5)(2,2.5)%
  \pstThreeDCoor[xMin=0,xMax=2,yMin=0,yMax=2,zMin=0,zMax=2]%
  \pstThreeDBox(0,0,0)(.5,0,0)(0,1,0)(0,0,1.5)
  \pstThreeDBox[RotX=90,linecolor=red](0,0,0)(.5,0,0)(0,1,0)(0,0,1.5)
  \pstThreeDBox[RotX=90,RotY=90,linecolor=green](0,0,0)(.5,0,0)(0,1,0)(0,0,1.5)
  \pstThreeDBox[RotX=90,RotY=90,RotZ=90,linecolor=blue](0,0,0)(.5,0,0)(0,1,0)(0,0,1.5)
\end{pspicture}%
\end{LTXexample}


It is sometimes more convenient to rotate the coordinate system by 
specifying a \emph{single} angle of rotation \verb+RotAngle+ (in degrees) 
about a vector whose coordinates are \verb+xRotVec+, \verb+yRotVec+, 
and \verb+zRotVec+ using the \verb+quaternion+ option for \verb+RotSequence+.

\begin{LTXexample}[pos=t]
\begin{pspicture}(-3,-1.8)(3,3)
\multido{\iA=0+10}{18}{%
 \pstThreeDCoor[linecolor=red, RotSequence=quaternion, RotAngle=\iA, xRotVec=3,yRotVec=0,zRotVec=3,
  xMin=0,xMax=3, yMin=0,yMax=3,  zMin=0,zMax=3]}
\pstThreeDCoor[linecolor=blue, RotSequence=quaternion, RotAngle=0, xRotVec=0, yRotVec=0, zRotVec=1,
 xMin=0,xMax=3, yMin=0,yMax=3, zMin=0,zMax=3]
\pstThreeDLine[linecolor=blue, linewidth=2pt, arrows=->](0,0,0)(3,0,3)
\uput[0](-2.28,1.2){$\vec{R}_\Phi$}
\end{pspicture}
\end{LTXexample}

Rotations of the coordinate system may be ``accumulated'' by applying 
successive rotation sequences using the \verb+RotSet+ variable, 
which is set either as a \verb+pst-3dplot+ object's optional argument, or
with a \verb+\psset[pst-3dplot]{RotSet=value}+
command. The usual \TeX{} scoping rules for the value of \verb+RotSet+
hold. The following are valid values of \verb+RotSet+:
\begin{itemize}
\item \verb+set+: Sets the rotation matrix using the rotation
parameters. This is the default value for \verb+RotSet+ and is
what is used if \verb+RotSet+ is not set as an option for the
\verb+pst-3dplot+ object, or if not previously
set within the object's scope by a \verb+\psset[pst-3dplot]{RotSet=val}+ 
command.
\item \verb+concat+: Concatenates the current rotation matrix with
a the new rotation that is defined by the rotation parameters. This option
is most useful when multiple \verb+\\pstThreeDCoor+ calls are made,
with or without actual plotting of the axes,
to accumulate rotations. A previous value of \verb+RotSet=set+
must have been made!
\item \verb+keep+: Keeps the current rotation matrix, ignoring the
rotation parameters. Mostly used internally to eliminate redundant
calculations.
\end{itemize}

\begin{LTXexample}[pos=t]
\begin{pspicture}(-3,-3)(3.6,3)
\pstThreeDCoor[linecolor=blue, RotSequence=quaternion, RotAngle=0, RotSet=set, xRotVec=0,yRotVec=0,zRotVec=1,
 xMin=0,xMax=3, yMin=0,yMax=3, zMin=0,zMax=3]
 \pstThreeDCoor[linecolor=green, RotSequence=quaternion, RotSet=concat, RotAngle=22.5, xRotVec=0,yRotVec=0,zRotVec=1,
  xMin=0,xMax=3, yMin=0,yMax=3, zMin=-0.6,zMax=3]
 \pstThreeDCoor[linecolor=yellow, RotSequence=quaternion, RotSet=concat, RotAngle=30, xRotVec=0,yRotVec=1,zRotVec=0,
  xMin=0,xMax=3,yMin=-0.6,yMax=3, zMin=0,zMax=3]
 \pstThreeDCoor[linecolor=red, RotSequence=quaternion, RotSet=concat, RotAngle=60, xRotVec=1,yRotVec=0,zRotVec=0,
  xMin=-0.6,xMax=3, yMin=0,yMax=3, zMin=0,zMax=3]%
\end{pspicture}
\end{LTXexample}

By default, the rotations defined by \verb+RotX+, \verb+RotY+, and
\verb+RotZ+ are rotations about the \emph{original} coordinate system's, 
$x$, $y$, or $z$
axes, respectively. More traditionally, however, these rotation angles
are defined as rotations about the rotated coordinate system's \emph{current},
$x$, $y$, or $z$ axis. The \verb+pst-3dplot+ variable option
\verb+eulerRotation+ can be set to \verb+true+ to activate Euler angle
definitions; i.e., \verb+eulerRotation=true+. The default is
\verb+eulerRotation=false+.

\begin{LTXexample}[pos=t]
\begin{pspicture}(-4,-5)(6,5)
 \pstThreeDCoor[linecolor=red, RotSequence=zyx, RotZ=90,RotY=90,RotX=0, 
  xMin=0,xMax=5, yMin=0,yMax=5, zMin=0,zMax=5]
 \pstThreeDCoor[linecolor=blue, RotSequence=zyx, RotZ=0,RotY=0,RotX=0,
  xMin=0,xMax=2.5, yMin=0,yMax=2.5, zMin=0,zMax=2.5]
\end{pspicture}
\end{LTXexample}

\begin{LTXexample}[pos=t]
\begin{pspicture}(-3,-5)(7,5)
 \pstThreeDCoor[eulerRotation=true, linecolor=red, RotSequence=zyx, RotZ=90, RotY=90, RotX=0,
  xMin=0,xMax=5, yMin=0,yMax=5, zMin=0,zMax=5] 
 \pstThreeDCoor[linecolor=blue, RotSequence=zyx, RotZ=0,RotY=0,RotX=0,
  xMin=0,xMax=2.5, yMin=0,yMax=2.5, zMin=0,zMax=2.5]
\end{pspicture}
\end{LTXexample}

\clearpage
\psset{unit=1cm,gridlabels=7pt}


\section{Plane Grids}
\begin{verbatim}
\pstThreeDPlaneGrid[<options>](xMin,yMin)(xMax,yMax)
\end{verbatim}

There are three additional options

\noindent
\begin{tabularx}{\linewidth}{@{}>{\bfseries\ttfamily}lX@{}}
planeGrid & can be one of the following values: \verb+xy, xz, yz+. Default is \verb+xy+.\\
subticks  & Number of ticks. Default is \verb+10+.\footnotemark \\
planeGridOffset & a length for the shift of the grid. Default is \verb+0+.
\end{tabularx}

\footnotetext{This options is also defined
in the package \texttt{pstricks-add}, so it is nessecary to to set this option
locally or with the family option of \texttt{pst-xkey}, eg \CMD{psset[pst-3dplot]\{subticks=...\}}}
This macro is a special one for the coordinate system to show the units, but can
be used in any way. \verb+subticks+ defines the number of ticklines for both axes and
\verb+xsubticks+ and \verb+ysubticks+ for each one.

\iffalse
\newpsstyle{xyPlane}{fillstyle=solid,fillcolor=black!20}
\newpsstyle{xzPlane}{fillstyle=solid,fillcolor=black!35,planeGrid=xz}
\newpsstyle{yzPlane}{fillstyle=solid,fillcolor=black!50,planeGrid=yz}
\fi

\noindent
\begin{minipage}{0.49\linewidth}
\begin{LTXexample}[pos=t,wide=false]
\begin{pspicture}(-4,-3.5)(5,4)
  \pstThreeDCoor[xMin=0,yMin=0,zMin=0,linewidth=2pt]
  \psset{linewidth=0.1pt,linecolor=lightgray}
  \pstThreeDPlaneGrid(0,0)(4,4)
  \pstThreeDPlaneGrid[planeGrid=xz](0,0)(4,4)
  \pstThreeDPlaneGrid[planeGrid=yz](0,0)(4,4)
\end{pspicture}
\end{LTXexample}
\end{minipage}\hfill
\begin{minipage}{0.49\linewidth}
\begin{LTXexample}[pos=t,wide=false]
\begin{pspicture}(-3,-3.5)(5,4)
  \psset{coorType=2}% set it globally!
  \pstThreeDCoor[xMin=0,yMin=0,zMin=0,linewidth=2pt]
  \psset{linewidth=0.1pt,linecolor=lightgray}
  \pstThreeDPlaneGrid(0,0)(4,4)
  \pstThreeDPlaneGrid[planeGrid=xz](0,0)(4,4)
  \pstThreeDPlaneGrid[planeGrid=yz](0,0)(4,4)
\end{pspicture}
\end{LTXexample}
\end{minipage}


\begin{LTXexample}[pos=t]
\begin{pspicture}(-1,-2)(10,10)
  \psset{Beta=20,Alpha=160,subticks=7}
  \pstThreeDCoor[xMin=0,yMin=0,zMin=0,xMax=7,yMax=7,zMax=7,linewidth=1pt]
  \psset{linewidth=0.1pt,linecolor=gray}
  \pstThreeDPlaneGrid(0,0)(7,7)
  \pstThreeDPlaneGrid[planeGrid=xz,planeGridOffset=7](0,0)(7,7)
  \pstThreeDPlaneGrid[planeGrid=yz](0,0)(7,7)
  \pscustom[linewidth=0.1pt,fillstyle=gradient,gradbegin=gray,gradmidpoint=0.5,plotstyle=curve]{%
    \psset{xPlotpoints=200,yPlotpoints=1}
    \psplotThreeD(0,7)(0,0){ x dup mul y dup mul 2 mul add x 6 mul sub y 4 mul sub 3 add 10 div }
    \psset{xPlotpoints=1,yPlotpoints=200,drawStyle=yLines}
    \psplotThreeD(7,7)(0,7){ x dup mul y dup mul 2 mul add x 6 mul sub y 4 mul sub 3 add 10 div }
    \psset{xPlotpoints=200,yPlotpoints=1,drawStyle=xLines}
    \psplotThreeD(7,0)(7,7){ x dup mul y dup mul 2 mul add x 6 mul sub y 4 mul sub 3 add 10 div }
    \psset{xPlotpoints=1,yPlotpoints=200,drawStyle=yLines}
    \psplotThreeD(0,0)(7,0){ x dup mul y dup mul 2 mul add x 6 mul sub y 4 mul sub 3 add 10 div }}
  \pstThreeDPlaneGrid[planeGrid=yz,planeGridOffset=7](0,0)(7,7)
\end{pspicture}
\end{LTXexample}


\begin{LTXexample}[pos=t]
\begin{pspicture}(-6,-2)(4,7)
  \psset{Beta=10,Alpha=30,subticks=7}
  \pstThreeDCoor[xMin=0,yMin=0,zMin=0,xMax=7,yMax=7,zMax=7,linewidth=1.5pt]
  \psset{linewidth=0.1pt,linecolor=gray}
  \pstThreeDPlaneGrid(0,0)(7,7)
  \pstThreeDPlaneGrid[planeGrid=xz](0,0)(7,7)
  \pstThreeDPlaneGrid[planeGrid=yz](0,0)(7,7)
  \pscustom[linewidth=0.1pt,fillstyle=gradient,gradbegin=gray,gradend=white,gradmidpoint=0.5,
    plotstyle=curve]{%
    \psset{xPlotpoints=200,yPlotpoints=1}
    \psplotThreeD(0,7)(0,0){ x dup mul y dup mul 2 mul add x 6 mul sub y 4 mul sub 3 add 10 div }
    \psset{xPlotpoints=1,yPlotpoints=200,drawStyle=yLines}
    \psplotThreeD(7,7)(0,7){ x dup mul y dup mul 2 mul add x 6 mul sub y 4 mul sub 3 add 10 div }
    \psset{xPlotpoints=200,yPlotpoints=1,drawStyle=xLines}
    \psplotThreeD(7,0)(7,7){ x dup mul y dup mul 2 mul add x 6 mul sub y 4 mul sub 3 add 10 div }
    \psset{xPlotpoints=1,yPlotpoints=200,drawStyle=yLines}
    \psplotThreeD(0,0)(7,0){ x dup mul y dup mul 2 mul add x 6 mul sub y 4 mul sub 3 add 10 div }}
  \pstThreeDPlaneGrid[planeGrid=xz,planeGridOffset=7](0,0)(7,7)
  \pstThreeDPlaneGrid[planeGrid=yz,planeGridOffset=7](0,0)(7,7)
\end{pspicture}
\end{LTXexample}

\medskip
The equation for the examples is 
\[ f(x,y)=\frac{x^2+2y^2-6x-4y+3}{10} \]

\section{Put}
There exists a special option for the put macros: 
\begin{verbatim}
origin=lt|lB|lb|t|c|B|b|rt|rB|rb
\end{verbatim}
for the placing of the text or other objects. 

\fboxsep0pt
\newcommand\Gobble[1]{}
\newsavebox{\FrBox}
\savebox{\FrBox}{\Huge\fbox{Rotating}}
\unitlength1pt
\begin{center}
\newcommand{\Hoehe}{\ht\FrBox\Gobble}
\newcommand{\Breite}{\wd\FrBox\Gobble}
\newcommand{\Tiefe}{\dp\FrBox\Gobble}
\newlength{\totalHeight}
\setlength{\totalHeight}{\ht\FrBox}
\addtolength{\totalHeight}{\dp\FrBox}
\newcommand{\tHoehe}{\totalHeight\Gobble}
\begin{picture}(1.7\Breite,\Hoehe)
	\put(,\Hoehe){\textcolor{red}{\circle*{3}}}
%	\put(0.5\Breite ,\Hoehe){\textcolor{red}{\circle*{3}}}
	\put(\Breite,\Hoehe){\textcolor{red}{\circle*{3}}}
	% center
%	\put(0,0.5\Hoehe){\textcolor{red}{\circle*{3}}}%
	\put(0.5\Breite,0.5\Hoehe){\textcolor{red}{\circle*{3}}}%
%	\put(\Breite,0.5\Hoehe){\textcolor{red}{\circle*{3}}}%
	% Baseline
	\put(0,0){\textcolor{red}{\circle*{3}}}%
%	\put(0.5\Breite,0){\textcolor{red}{\circle*{3}}}%
	\put(\Breite,0){\textcolor{red}{\circle*{3}}}%
	% bottom
	\put(0,-\Tiefe){\textcolor{red}{\circle*{3}}}%
%	\put(0.5\Breite,-\Tiefe){\textcolor{red}{\circle*{3}}}%
	\put(\Breite,-\Tiefe){\textcolor{red}{\circle*{3}}}%
	% labels
	\put(0,1.2\Hoehe){l}
	\put(0.5\Breite,1.2\Hoehe){c}
	\put(\Breite,1.2\Hoehe){r}
	\put(1.05\Breite,0.9\Hoehe){t}
	\put(1.05\Breite,0.4\Hoehe){c}
	\put(1.15\Breite,-1){\textbf{B}aseline}
	\put(1.05\Breite,-1.2\Tiefe){b}
	\put(0,0){\usebox{\FrBox}}%
\end{picture}
\end{center}


This works only well for the \verb|\pstThreeDPut| macro. The default is \verb|c| and for the \verb|pstPlanePut| the left baseline \verb|lB|.
 
\subsection{\texttt{pstThreeDPut}}
The syntax is similiar to the \verb|\rput| macro:
{\footnotesize\begin{verbatim}
\pstThreeDPut[options](x,y,z){<any stuff>}
\end{verbatim}} 

\begin{LTXexample}[width=3.25cm]
\begin{pspicture}(-2,-1.25)(1,2.25)
  \psset{Alpha=-60,Beta=30}
  \pstThreeDCoor[linecolor=blue,%
     xMin=-1,xMax=2,yMin=-1,yMax=2,zMin=-1,zMax=2]
  \pstThreeDPut(1,0.5,1.25){pst-3dplot}
  \pstThreeDDot[drawCoor=true](1,0.5,1.25)
\end{pspicture}
\end{LTXexample}


\medskip
Internally the \verb|\pstThreeDPut| macro defines the two dimensional node \verb|temp@pstNode| and then uses the default \verb|\rput| macro from \verb|pstricks|. In fact of the perspective view od the coordinate system, the 3D dot must not be seen as the center of the printed stuff.

\subsection[\texttt{pstPlanePut}]{\texttt{pstPlanePut}\protect\footnote{Thanks to Torsten Suhling}}
The syntax of the \verb|pstPlanePut| is
% -------------------------------------------
% ### Aenderung 
% - raus: -----------------------------------
% \begin{verbatim}
% \pstPlanePut[plane=<2D plane>](x,y,z){Object}
% \end{verbatim}
% 
% Possible values for the two dimensional plane are \verb| xy xz yz |. If this optional parameter is missing then \verb|plane=xy| is set. The first letter marks the positive direction for the width and the second for the height.
% - rein: -----------------------------------
\begin{verbatim}
\pstPlanePut[plane=<2D plane>,planecorr=<Correction of plane's alignment>](x,y,z){Object}
\end{verbatim}

We have two parameters, \verb|plane| and \verb|planecorr|; both are optional.  Let's start with the first parameter, \verb|plane|. 
Possible values for the two dimensional plane are \verb| xy xz yz |. If this parameter is missing then \verb|plane=xy| is set. The first letter marks the positive direction for the width and the second for the height.
% - ende ------------------------------------------------

The object can be of any type, in most cases it will be some kind of text. The reference point for the object is the left side and vertically centered, often abbreviated as \verb|lB|. The following examples show for all three planes the same textbox.
 
\begin{LTXexample}[width=7.25cm]
\begin{pspicture}(-4,-4)(3,4)
  \psset{Alpha=30}
  \pstThreeDCoor[xMin=-4,yMin=-4,zMin=-4]
  \pstPlanePut[plane=xy](0,0,-3){\fbox{\Huge\red xy plane}}
  \pstPlanePut[plane=xy](0,0,0){\fbox{\Huge\red xy plane}}
  \pstPlanePut[plane=xy](0,0,3){\fbox{\Huge\red xy plane}}
\end{pspicture}
\end{LTXexample}

\begin{LTXexample}[width=7.25cm]
\begin{pspicture}(-5,-3)(2,3)
  \pstThreeDCoor[xMin=2,yMin=-4,zMin=-3,zMax=2]
  \pstPlanePut[plane=xz](0,-3,0){\fbox{\Huge\green\textbf{xz plane}}}
  \pstPlanePut[plane=xz](0,0,0){\fbox{\Huge\green\textbf{xz plane}}}
  \pstPlanePut[plane=xz](0,3,0){\fbox{\Huge\green\textbf{xz plane}}}
\end{pspicture}
\end{LTXexample}


\begin{LTXexample}[width=7.25cm]
\begin{pspicture}(-2,-4)(6,2)
  \pstThreeDCoor[xMin=-4,yMin=-4,zMin=-4,xMax=2,zMax=2]
  \pstPlanePut[plane=yz](-3,0,0){\fbox{\Huge\blue\textbf{yz plane}}}
  \pstPlanePut[plane=yz](0,0,0){\fbox{\Huge\blue\textbf{yz plane}}}
  \pstPlanePut[plane=yz](3,0,0){\fbox{\Huge\blue\textbf{yz plane}}}
\end{pspicture}
\end{LTXexample}


\bigskip
The following examples use the \verb|origin| option to show that there are 
still some problems with the xy-plane. 
% ----------------------------------------------------------------------
% Einfuegung Erlaeuterug examplee und Abbildungen 
% - raus: --------------------------------------------------------------
% 
% \begin{LTXexample}[width=6.25cm]
% \begin{pspicture}(-3,-2)(3,4)\psgrid
%   \psset{origin=lb}
%   \pstThreeDCoor
%   \pstThreeDDot[drawCoor=true,linecolor=red](-1,-1,2)
%   \pstPlanePut[plane=xy](-1,-1,2){\fbox{\Huge\red\textbf{XY}}}
%   \pstThreeDDot[drawCoor=true,linecolor=green](1,3.5,1)
%   \pstPlanePut[plane=xz](1,3.5,1){\fbox{\Huge\green\textbf{XZ}}}
%   \pstThreeDDot[drawCoor=true,linecolor=blue](-2,1,3)
%   \pstPlanePut[plane=yz](-2,1,3){\fbox{\Huge\blue\textbf{YZ}}}
% \end{pspicture}
% \end{LTXexample}
% % 
% - rein: --------------------------------------------------------------
The second parameter is \verb|planecorr|.  As first the values:

\begin{description}
  \item[\texttt{off}]~Former and default behaviour; nothing will be changed.
  This value is set, when parameter is missing.
  \item[\texttt{normal}]~Default correction, planes will be rotated to be readable.
  \item[\texttt{xyrot}]~Additionaly correction for $xy$ plane; bottom line of
  letters will be set parallel to the $y$-axis. 
\end{description}

What kind off correction is ment?  In the plots above labels for the $xy$ 
plane and the $xz$ plane are mirrored.  This is not a bug, it's \dots mathematics. 

\verb|\pstPlanePut| puts the labels on the plane of it's value.  That means, 
\verb|plane=xy| puts the label on the $xy$ plane, so that the $x$ marks the 
positive direction for the width, the $y$ for the height and the label 
{\small{XY plane}} on the top side of plane. If you see the label mirrored, 
you just look from the bottom side of plane \dots{}

If you want to keep the labels readable for every view, i.\,e.\ for every 
value of \verb|Alpha| and \verb|Beta|, you should set the value of the 
parameter \verb|planecorr| to \verb|normal|; just like in next example:

\medskip
\begin{LTXexample}[width=6cm]
\begin{pspicture}(-3,-2)(3,4)
  \psset{origin=lb}
  \pstThreeDCoor[xMax=3.2,yMax=3.2,zMax=4]
  \pstThreeDDot[drawCoor=true,linecolor=red](1,-1,2)
  \pstPlanePut[plane=xy,planecorr=normal](1,-1,2)
	{\fbox{\Huge\red\textbf{XY}}}
  \pstThreeDDot[drawCoor=true,linecolor=green](1,3,1)
  \pstPlanePut[plane=xz,planecorr=normal](1,3,1)
	{\fbox{\Huge\green\textbf{XZ}}}
  \pstThreeDDot[drawCoor=true,linecolor=blue](-1.5,0.5,3)
  \pstPlanePut[plane=yz,planecorr=normal](-1.5,0.5,3)
	{\fbox{\Huge\blue\textbf{YZ}}}
\end{pspicture}
\end{LTXexample}

\medskip
But, why we have a third value \verb|xyrot| of \verb|planecorr|?
If there isn't an symmetrical view, -- just like in this example -- it
could be usefull to rotate the label for $xy$-plane, so that body line of
letters is parallel to the $y$ axis.  It's done by setting
\verb|planecorr=xyrot|\,:


\medskip
\begin{LTXexample}[width=6cm]
\begin{pspicture}(-2,-2)(4,4)
  \psset{origin=lb}
  \psset{Alpha=69.3,Beta=19.43}
  \pstThreeDCoor[xMax=4,yMax=4,zMax=4]
  \pstThreeDDot[drawCoor=true,linecolor=red](1,-1,2)
  \pstPlanePut[plane=xy,planecorr=xyrot](1,-1,2)
	{\fbox{\Huge\red\textbf{XY}}}
  \pstThreeDDot[drawCoor=true,linecolor=green](1,3.5,1)
  \pstPlanePut[plane=xz,planecorr=xyrot](1,3.5,1)
	{\fbox{\Huge\green\textbf{XZ}}}
  \pstThreeDDot[drawCoor=true,linecolor=blue](-2,1,3)
  \pstPlanePut[plane=yz,planecorr=xyrot](-2,1,3)
	{\fbox{\Huge\blue\textbf{YZ}}}
\end{pspicture}
\end{LTXexample}
% --- ende ------------------------------------------------------------



\psset{Alpha=45,xunit=1cm,yunit=1cm,xMin=-1,yMin=-1,zMin=-1}

\section{Nodes}
The syntax is
\begin{verbatim}
\pstThreeDNode(x,y,z){<node name>}
\end{verbatim}

This node is internally a two dimensional node, so it cannot be used as a replacement for the parameters \verb|(x,y,z)| of a 3D dot, which is possible with the \verb|\psline| macro from \verb|pst-plot|: \verb|\psline{A}{B}|, where \verb|A| and \verb|B| are two nodes. It is still on the to do list, that it may also be possible with \verb|pst-3dplot|. On the other hand it is no problem to define two 3D nodes \verb|C| and \verb|D| and then drawing a two dimensional line from \verb|C| to \verb|D|.

\section{Dots}
The syntax for a dot is
\begin{verbatim}
\pstThreeDDot[<options>](x,y,z)
\end{verbatim}

Dots can be drawn with dashed lines for the three coordinates, when the option \verb|drawCoor| is set to \verb|true|. It is also possible to draw an unseen dot
with the option \verb|dotstyle=none|.\index{dotstyle} In this case the macro draws only the coordinates\index{coordinates}
when the \verb|drawCoor| option is set to true.

\begin{LTXexample}[width=4.25cm]
\begin{pspicture}(-2,-2)(2,2)
  \pstThreeDCoor[xMin=-2,xMax=2,yMin=-2,yMax=2,zMin=-2,zMax=2]
  \psset{dotstyle=*,dotscale=2,linecolor=red,drawCoor=true}
  \pstThreeDDot(-1,1,1)
  \pstThreeDDot(1.5,-1,-1)
\end{pspicture}
\end{LTXexample}

In the following figure the coordinates of the dots are $(a,a,a)$ where a is $-2,-1,0,1,2$.


\begin{LTXexample}[width=5.25cm]
\begin{pspicture}(-3,-3.25)(2,3.25)
  \psset{Alpha=30,Beta=60,dotstyle=square*,dotsize=3pt,%
    linecolor=blue,drawCoor=true}
  \pstThreeDCoor[xMin=-3,xMax=3,yMin=-3,yMax=3,zMin=-3,zMax=3]
  \multido{\n=-2+1}{5}{\pstThreeDDot(\n,\n,\n)}  
\end{pspicture}
\end{LTXexample}


\section{Lines}
The syntax for a three dimensional line is just like the same from \verb+\psline+
{\footnotesize\begin{verbatim}
\pstThreeDLine[<options>]{<arrow>}(x1,y1,z1)(...)(xn,yn,zn)
\end{verbatim}}

The option and arrow part are both optional and the number of points is only limited
to the memory.
All options for lines from \verb|pstricks| are possible, there are no special ones for a 3D line. There is no difference in drawing a line or a vector; the first one has an arrow of type "'\verb|-|"` and the second of "'\verb|->|"`.

There is no special \verb+polygon+ macro, because you can get nearlx the same with
\verb+\pstThreeDLine+.

\begin{LTXexample}[width=4.25cm]
\begin{pspicture}(-2,-2.25)(2,2.25)
  \pstThreeDCoor[xMin=-2,xMax=2,yMin=-2,yMax=2,zMin=-2,zMax=2]
  \psset{dotstyle=*,linecolor=red,drawCoor=true}
  \pstThreeDDot(-1,1,0.5)
  \pstThreeDDot(1.5,-1,-1)
  \pstThreeDLine[linewidth=3pt,linecolor=blue,arrows=->]%
    (-1,1,0.5)(1.5,-1,-1)
\end{pspicture}
\end{LTXexample}




\begin{LTXexample}[width=4.25cm]
\begin{pspicture}(-2,-2.25)(2,2.25)
  \pstThreeDCoor[xMin=-2,xMax=2,yMin=-2,yMax=2,zMin=-2,zMax=2]
  \psset{dotstyle=*,linecolor=red,drawCoor=true}
  \pstThreeDDot(-1,1,1)
  \pstThreeDDot(1.5,-1,-1)
  \pstThreeDLine[linewidth=3pt,linecolor=blue](-1,1,1)(1.5,-1,-1)
\end{pspicture}
\end{LTXexample}


\begin{LTXexample}[width=4.25cm]
\begin{pspicture}(-2,-2.25)(2,2.25)
  \psset{Alpha=30,Beta=60,dotstyle=pentagon*,dotsize=5pt,%
    linecolor=red,drawCoor=true}
  \pstThreeDCoor[xMin=-2,xMax=2,yMin=-2,yMax=2,zMin=-2,zMax=2]
  \pstThreeDDot(-1,1,1)
  \pstThreeDDot(1.5,-1,-1)
  \pstThreeDLine[linewidth=3pt,linecolor=blue](-1,1,1)(1.5,-1,-1)
\end{pspicture}
\end{LTXexample}

\begin{LTXexample}[width=4.25cm]
\begin{pspicture}(-2,-2.25)(2,2.25)
  \psset{Alpha=30,Beta=-60}
  \pstThreeDCoor[xMin=-2,xMax=2,yMin=-2,yMax=2,zMin=-2,zMax=2]
  \pstThreeDDot[dotstyle=square,linecolor=blue,drawCoor=true](-1,1,1)
  \pstThreeDDot[drawCoor=true](1.5,-1,-1)
  \pstThreeDLine[linewidth=3pt,linecolor=blue](-1,1,1)(1.5,-1,-1)
\end{pspicture}
\end{LTXexample}


\begin{LTXexample}[width=4.25cm]
\begin{pspicture}(-2,-2.25)(2,2.25)
  \psset{Alpha=30,Beta=-60}
  \pstThreeDCoor[xMin=-2,xMax=2,yMin=-2,yMax=2,zMin=-2,zMax=2]
  \pstThreeDDot[dotstyle=square,linecolor=blue,drawCoor=true](-1,1,1)
  \pstThreeDDot[drawCoor=true](1.5,-1,-1)
  \pstThreeDLine[linewidth=3pt,arrowscale=1.5,%
    linecolor=magenta,linearc=0.5]{<->}(-1,1,1)(1.5,2,-1)(1.5,-1,-1)
\end{pspicture}
\end{LTXexample}

\begin{LTXexample}[pos=t]
\begin{pspicture}(-3,-2)(4,5)\label{lines}
  \pstThreeDCoor[xMin=-3,xMax=3,yMin=-1,yMax=4,zMin=-1,zMax=3]
  \multido{\iA=1+1,\iB=60+-10}{5}{%
    \ifcase\iA\or\psset{linecolor=red}\or\psset{linecolor=green}
      \or\psset{linecolor=blue}\or\psset{linecolor=cyan}
      \or\psset{linecolor=magenta}
    \fi
    \pstThreeDLine[SphericalCoor=true,linewidth=3pt]%
      (\iA,0,\iB)(\iA,30,\iB)(\iA,60,\iB)(\iA,90,\iB)(\iA,120,\iB)(\iA,150,\iB)%
      (\iA,180,\iB)(\iA,210,\iB)(\iA,240,\iB)(\iA,270,\iB)(\iA,300,\iB)%
      (\iA,330,\iB)(\iA,360,\iB)%
  }
  \multido{\iA=0+30}{12}{%
    \pstThreeDLine[SphericalCoor=true,linestyle=dashed]%
       (0,0,0)(1,\iA,60)(2,\iA,50)(3,\iA,40)(4,\iA,30)(5,\iA,20)}
\end{pspicture}
\end{LTXexample}


\section{Triangles}
A triangle is given with its three points:
\begin{verbatim}
\pstThreeDTriangle[<options>](P1)(P2)(P3)
\end{verbatim}

When the option \verb|fillstyle| is set to another value than \verb|none| the triangle is filled with the active color or with the one which is set with the option \verb|fillcolor|.

\begin{LTXexample}[width=6.25cm]
\begin{pspicture}(-3,-4.25)(3,3.25)
  \pstThreeDCoor[xMin=-4,xMax=4,yMin=-3,yMax=5,zMin=-4,zMax=3]
  \pstThreeDTriangle[drawCoor=true,linecolor=black,%
    linewidth=2pt](3,1,-2)(1,4,-1)(-2,2,0)
  \pstThreeDTriangle[fillcolor=yellow,fillstyle=solid,%
    linecolor=blue,linewidth=1.5pt](5,1,2)(3,4,-1)(-1,-2,2)
\end{pspicture}
\end{LTXexample}

Especially for triangles the option \verb|linejoin| is important. The default value is $1$, which gives rounded edges. 

\begin{figure}[htb]
\centering
\psset{linewidth=0.2}
\begin{pspicture}(3,2.25)
  \psline[linejoin=0](0,0)(1,2)(2,0)(3,2)
\end{pspicture}%
\hspace{0.4cm}%
\begin{pspicture}(4,2.25)
  \psline[linejoin=1](0,0)(1,2)(2,0)(3,2)(4,0)
\end{pspicture}%
\hspace{0.4cm}%
\begin{pspicture}(3,2.25)
  \psline[linejoin=2](0,0)(1,2)(2,0)(3,2)
\end{pspicture}
\caption{The meaning of the option \texttt{linejoin=0|1|2} for drawing lines}
\end{figure} 


\section{Squares}
The syntax for a 3D square is:
\begin{verbatim}
\pstThreeDSquare(<vector o>)(<vector u>)(<vector v>)
\end{verbatim}

\begin{LTXexample}[width=5cm]
\begin{pspicture}(-1,-1)(4,3)
  \pstThreeDCoor[xMin=-3,xMax=1,yMin=-1,yMax=2,zMin=-1,zMax=3]
  \psset{arrows=->,arrowsize=0.2,linecolor=blue,linewidth=1.5pt}
  \pstThreeDLine[linecolor=green](0,0,0)(-2,2,3)\uput[45](1.5,1){$\vec{o}$}
  \pstThreeDLine(-2,2,3)(2,2,3)\uput[0](3,2){$\vec{u}$}
  \pstThreeDLine(-2,2,3)(-2,3,3)\uput[180](1,2){$\vec{v}$}
\end{pspicture}
\end{LTXexample}

\medskip
Squares are nothing else than a polygon with the starting point $P_o$ given with the origin vector $\vec{o}$ and the two direction vectors $\vec{u}$ and $\vec{v}$, which build the sides of the square.


\begin{LTXexample}[width=7.25cm]
\begin{pspicture}(-3,-2)(4,3)
  \pstThreeDCoor[xMin=-3,xMax=3,yMin=-1,yMax=4,zMin=-1,zMax=3]
  {\psset{fillcolor=blue,fillstyle=solid,drawCoor=true,dotstyle=*}
   \pstThreeDSquare(-2,2,3)(4,0,0)(0,1,0)}
\end{pspicture}
\end{LTXexample}


\section{Boxes}
A box is a special case of a square and has the syntax

\begin{verbatim}
\pstThreeDBox[<options>](<vector o>(<vector u>)(<vector v>)(<vector w>)
\end{verbatim}


These are the origin vector $\vec{o}$ and three direction vectors $\vec{u}$, $\vec{v}$ and $\vec{w}$, which are for example shown in the following figure.


\begin{LTXexample}[width=5.25cm]
\begin{pspicture}(-2,-1.25)(3,4.25)
  \psset{Alpha=30,Beta=30}
  \pstThreeDCoor[xMin=-3,xMax=1,yMin=-1,yMax=2,zMin=-1,zMax=4]
  \pstThreeDDot[drawCoor=true](-1,1,2)
  \psset{arrows=->,arrowsize=0.2}
  \pstThreeDLine[linecolor=green](0,0,0)(-1,1,2)
  \uput[0](0.5,0.5){$\vec{o}$}
  \uput[0](0.9,2.25){$\vec{u}$}
  \uput[90](0.5,1.25){$\vec{v}$}
  \uput[45](2,1.){$\vec{w}$}
  \pstThreeDLine[linecolor=blue](-1,1,2)(-1,1,4)
  \pstThreeDLine[linecolor=blue](-1,1,2)(1,1,2)
  \pstThreeDLine[linecolor=blue](-1,1,2)(-1,2,2)
\end{pspicture}
\end{LTXexample}

\begin{LTXexample}[width=5.25cm]
\begin{pspicture}(-2,-1.25)(3,4.25)
  \psset{Alpha=30,Beta=30}
  \pstThreeDCoor[xMin=-3,xMax=1,yMin=-1,yMax=2,zMin=-1,zMax=4]
  \pstThreeDBox[hiddenLine](-1,1,2)(0,0,2)(2,0,0)(0,1,0)
  \pstThreeDDot[drawCoor=true](-1,1,2)
\end{pspicture}
\end{LTXexample}

\begin{LTXexample}[width=5.25cm]
\begin{pspicture}(-2,-1.25)(3,4.25)
  \psset{Alpha=210,Beta=30}
  \pstThreeDCoor[xMin=-3,xMax=1,yMin=-1,yMax=2,zMin=-1,zMax=4]
  \pstThreeDBox[hiddenLine](-1,1,2)(0,0,2)(2,0,0)(0,1,0)
  \pstThreeDDot[drawCoor=true](-1,1,2)
\end{pspicture}
\end{LTXexample}

\begin{LTXexample}[width=5.25cm]
\begin{pspicture}(-2,-1.25)(3,4.25)
  \psset{Alpha=30,Beta=30}
  \pstThreeDCoor[xMin=-3,xMax=1,yMin=-1,yMax=2,zMin=-1,zMax=4]
  \pstThreeDBox[hiddenLine](-1,1,2)(0,0,2)(2,0,0)(0,1,0)
  \pstThreeDDot[drawCoor=true](-1,1,2)
\end{pspicture}
\end{LTXexample}

\begin{LTXexample}[width=5.25cm]
\begin{pspicture}(-2,-1.25)(3,4.25)
  \psset{Alpha=130,Beta=30}
  \pstThreeDCoor[xMin=-3,xMax=1,yMin=-1,yMax=2,zMin=-1,zMax=4]
  \pstThreeDBox[hiddenLine](-1,1,2)(0,0,2)(2,0,0)(0,1,0)
  \pstThreeDDot[drawCoor=true](-1,1,2)
\end{pspicture}
\end{LTXexample}

\begin{LTXexample}[width=5.25cm]
\begin{pspicture}(-2,-1.25)(3,4.25)
  \psset{Alpha=130,Beta=100}
  \pstThreeDCoor[xMin=-3,xMax=1,yMin=-1,yMax=2,zMin=-1,zMax=4]
  \pstThreeDBox[hiddenLine](-1,1,2)(0,0,2)(2,0,0)(0,1,0)
  \pstThreeDDot[drawCoor=true](-1,1,2)
\end{pspicture}
\end{LTXexample}


\begin{verbatim}
\psBox[<options>](<vector o>){width}{depth}{height}
\end{verbatim}

The origin vector $\vec{o}$ determines the left corner of the box.


\begin{LTXexample}[width=6.25cm]
\begin{pspicture}(-3,-2)(3,5)
\psset{Alpha=2,Beta=10}
\pstThreeDCoor[zMax=5,yMax=7]
  \psBox(0,0,0){2}{4}{3}
\end{pspicture}
\end{LTXexample}

\begin{LTXexample}[width=6.25cm]
\begin{pspicture}(-3,-3)(3,3)
\psset{Beta=50}
\pstThreeDCoor[xMax=3,zMax=6,yMax=6]
  \psBox[showInside=false](0,0,0){2}{5}{3}
\end{pspicture}
\end{LTXexample}


\begin{LTXexample}[width=6.25cm]
\begin{pspicture}(-3,-4)(3,2)
\psset{Beta=40}
\pstThreeDCoor[zMax=3]
  \psBox[RotY=20,showInside=false](0,0,0){2}{5}{3}
\end{pspicture}
\end{LTXexample}



\begin{LTXexample}[width=6.25cm]
\psset{Beta=10,xyzLight=-7 3 4}
\begin{pspicture}(-3,-2)(3,4)
\pstThreeDCoor[zMax=5]
  \psBox(0,0,0){2}{5}{3}
\end{pspicture}
\end{LTXexample}

\begin{LTXexample}[width=6.25cm]
\psset{Beta=10,xyzLight=-7 3 4}
\begin{pspicture}(-3,-2)(3,4)
\psset{Alpha=110}
\pstThreeDCoor[zMax=5]
  \psBox(0,0,0){2}{5}{3}
\end{pspicture}
\end{LTXexample}

\begin{LTXexample}[width=6.25cm]
\psset{Beta=10,xyzLight=-7 3 4}
\begin{pspicture}(-3,-2)(3,4)
\psset{Alpha=200}
\pstThreeDCoor[zMax=5]
  \psBox(0,0,0){2}{5}{3}
\end{pspicture}
\end{LTXexample}

\begin{LTXexample}[width=6.25cm]
\psset{Beta=10,xyzLight=-7 3 4}
\begin{pspicture}(-3,-2)(3,4)
\psset{Alpha=290}
\pstThreeDCoor[zMax=5]
  \psBox(0,0,0){2}{5}{3}
\end{pspicture}
\end{LTXexample}




\section{Ellipses and circles}

The equation for a two dimensional ellipse (figure \ref{fig:ellipse})is:
\begin{equation}
e:\frac{\left(x-x_{M}\right)^{2}}{a^{2}}+\frac{\left(y-y_{M}\right)^{2}}{b^{2}}=1
\label{gl.600}
\end{equation}

\begin{figure}[htb]
\centering
\begin{pspicture*}(-3,-1.5)(3,1.5)
  \psset{unit=0.75cm}
  \psline{->}(-3.5,0)(3.5,0)
  \rput(3.3,-0.3){x}\psline{->}(0,-2.5)(0,2.5)\rput(-0.3,2.3){y}
  \pscircle(-2,0){0.1}\pscircle(2,0){0.1}
  \psline[linestyle=dotted,linewidth=0.5pt](-2,0)(0,2)(2,0)
  \rput(-1.4,1){a}\rput(1.4,1){a}
  \psline[linewidth=0.5pt,linearc=.25]%
 	(-3,0)(-3,-0.25)(-1.5,-0.25)%
	(-1.5,-0.4)(-1.5,-0.25)(0,-0.25)(0,0)
  \rput(-1.5,-0.7){a}
  \psline[linewidth=0.5pt,linearc=.25]%
		(0,2)(0.2,2)(0.2,1)(0.4,1)(0.2,1)(0.2,0)(0,0)
	\rput(0.7,1){b}
	\psellipse[linewidth=2pt](3,2)
	\rput(-.4,-.4){M}\rput(2,-.4){$F_2$}
	\rput(-2.2,0.4){$F_1$}
	\rput(1,.2){e}\rput(-1,.2){e}
	\rput(-1,-1){$r_1$}\rput(1.7,-1){$r_2$}
	\psline[linestyle=dotted,linewidth=1pt]{->}(-2,0)(1,-1.8)
	\psline[linestyle=dotted,linewidth=1pt]{<-}(1,-1.8)(2,0)
\end{pspicture*}
\caption{Definition of an Ellipse}\label{fig:ellipse}
\end{figure}



$\left(x_m;y_m\right)$ is the center,  $a$
and $b$ the semi major and semi minor axes 
respectively and $e$ the excentricity. For $a=b=1$ in equation~\ref{gl.600} we get the one for the circle, 
which is nothing else than a special ellipse. 
The equation written in the parameterform is

\begin{equation}\label{gl601}
\begin{split}
	x = a\cdot\cos\alpha\\
	y = b\cdot\sin\alpha
\end{split}
\end{equation}

or the same with vectors to get an ellipse in a 3D system:
\begin{align}\label{gl.6}
	e:\vec{x} &=\vec{m}+\cos\alpha\cdot\vec{u}+\sin\alpha\cdot\vec{v}\qquad 0\leq\alpha\leq360	
\end{align}

where $\vec{m}$ is the center,  $\vec{u}$ and $\vec{v}$ the directions vectors which are 
perpendicular to each other. 


\subsection{Options}
In addition to all possible options from \verb|pst-plot| there are two special 
options to allow drawing of an arc (with predefined values for a full ellipse/circle):
%
\begin{verbatim}
beginAngle=0
endAngle=360
\end{verbatim}

Ellipses and circles are drawn with the in section~\ref{subsec:parametricplotThreeD} described 
\verb|parametricplotThreeD| macro with a default setting of $50$ points for a full ellipse/circle.

\subsection{Ellipse}
It is very difficult to see in a 3D coordinate system the difference of an ellipse and a circle. Depending to the view point an ellipse maybe seen as a circle and vice versa. The syntax of the ellipse macro is:
\begin{verbatim}
\pstThreeDEllipse[<option>](cx,cy,cz)(ux,uy,uz)(vx,vy,vz)
\end{verbatim}

where \verb|c| is for center and \verb|u| and \verb|v| for the two direction vectors. 
The order of these two vectors is important for the drawing if it
is a left or right turn. It follows the right hand rule: flap the first vector $\vec{u}$ on the 
shortest way into the second one  $\vec{u}$, then you'll get the positive rotating.


\begin{LTXexample}[pos=t]
\begin{pspicture}(-3,-2)(3,3)
  \pstThreeDCoor[IIIDticks]
  \psset{arrowscale=2,arrows=->}
  \pstThreeDLine(0,0,0)(3,0,0)\pstThreeDLine(0,0,0)(0,3,0)\pstThreeDLine(0,0,0)(0,0,3)
  \psset{linecolor=blue,linewidth=1.5pt,beginAngle=0,endAngle=90}
  \pstThreeDEllipse(0,0,0)(3,0,0)(0,3,0) \pstThreeDEllipse(0,0,0)(0,0,3)(3,0,0)
  \pstThreeDEllipse(0,0,0)(0,3,0)(0,0,3)
\end{pspicture}\hspace{2em}
\begin{pspicture}(-3,-2)(3,3)
  \pstThreeDCoor[IIIDticks]
  \psset{arrowscale=2,arrows=->}
  \pstThreeDLine(0,0,0)(3,0,0)\pstThreeDLine(0,0,0)(0,3,0)\pstThreeDLine(0,0,0)(0,0,3)
  \psset{linecolor=blue,linewidth=1.5pt,beginAngle=0,endAngle=90}
  \pstThreeDEllipse(0,0,0)(0,3,0)(3,0,0) \pstThreeDEllipse(0,0,0)(3,0,0)(0,0,3)
  \pstThreeDEllipse(0,0,0)(0,0,3)(0,3,0)
\end{pspicture}
\end{LTXexample}



\begin{LTXexample}[width=4.25cm]
\begin{pspicture}(-2,-2.25)(2,2.25)
  \pstThreeDCoor[xMax=2,yMax=2,zMax=2]
  \pstThreeDDot[linecolor=red,drawCoor=true](1,0.5,0.5)
  \psset{linecolor=blue, linewidth=1.5pt}
  \pstThreeDEllipse(1,0.5,0.5)(-0.5,1,0.5)(1,-0.5,-1)
  \psset{beginAngle=0,endAngle=270,linecolor=green}
  \pstThreeDEllipse(1,0.5,0.5)(-0.5,0.5,0.5)(0.5,0.5,-1)
  \pstThreeDEllipse[RotZ=45,linecolor=red](1,0.5,0.5)(-0.5,0.5,0.5)(0.5,0.5,-1)
\end{pspicture}
\end{LTXexample}


\subsection{Circle}

The circle is a special case of an ellipse (equ.~\ref{gl.6}) with the vectors
$\vec{u}$ and $\vec{v}$  which are perpendicular 
to each other: $\left|\vec{u}\right|=\left|\vec{v}\right|=r$.
with
$\vec{u}\cdot\vec{v}=\vec{0}$

The macro \verb|\pstThreeDCircle| is nothing else than a synonym for \verb|\pstThreeDEllipse|. 
In the following example the circle is drawn with only $20$  plotpoints and the option \verb|showpoints=true|. 

\begin{LTXexample}[width=4.25cm]
\begin{pspicture}(-2,-1.25)(2,2.25)
  \pstThreeDCoor[xMax=2,yMax=2,zMax=2,linecolor=black]
  \psset{linecolor=red,linewidth=2pt,plotpoints=20,showpoints=true}
  \pstThreeDCircle(1.6,0.6,1.7)(0.8,0.4,0.8)(0.8,-0.8,-0.4)
  \pstThreeDDot[drawCoor=true,linecolor=red](1.6,0.6,1.7)
  \pstThreeDCircle[linecolor=blue](1.6,1.6,1.7)(0.8,0.4,0.8)(0.8,-0.8,-0.4)
  \pstThreeDDot[drawCoor=true,linecolor=blue](1.6,1.6,1.7)
\end{pspicture}
\end{LTXexample}


\begin{center}
\bgroup
\makebox[\linewidth]{%
\def\radius{4 }\def\PhiI{20 }\def\PhiII{50 }
%
\def\RadIs{\radius \PhiI sin mul}
\def\RadIc{\radius \PhiI cos mul}
\def\RadIIs{\radius \PhiII sin mul}
\def\RadIIc{\radius \PhiII cos mul}
\begin{pspicture}(-4,-4)(4,5)
  \psset{Alpha=45,Beta=30,linestyle=dashed}
  \pstThreeDCoor[linestyle=solid,xMin=-5,xMax=5,yMax=5,zMax=5,IIIDticks]
  \pstThreeDEllipse[linecolor=red](0,0,0)(0,\radius,0)(0,0,\radius)
  \pstThreeDEllipse(\RadIs,0,0)(0,\RadIc,0)(0,0,\RadIc)
  \pstThreeDEllipse(\RadIIs,0,0)(0,\RadIIc,0)(0,0,\RadIIc)
%
  \pstThreeDEllipse[linestyle=dotted,SphericalCoor](0,0,0)(\radius,90,\PhiI)(\radius,0,0)
  \pstThreeDEllipse[SphericalCoor,
	beginAngle=-90,endAngle=90](0,0,0)(\radius,90,\PhiI)(\radius,0,0)
  \pstThreeDEllipse[linestyle=dotted,SphericalCoor](0,0,0)(\radius,90,\PhiII)(\radius,0,0)
  \pstThreeDEllipse[SphericalCoor,
	beginAngle=-90,endAngle=90](0,0,0)(\radius,90,\PhiII)(\radius,0,0)
%
  \psset{linecolor=blue,arrows=->,arrowscale=2,linewidth=1.5pt,linestyle=solid}
    \pstThreeDEllipse[SphericalCoor,beginAngle=\PhiI,endAngle=\PhiII]%
	(0,0,0)(\radius,90,\PhiII)(\radius,0,0)
    \pstThreeDEllipse[beginAngle=\PhiII,endAngle=\PhiI](\RadIIs,0,0)(0,\RadIIc,0)(0,0,\RadIIc)
    \pstThreeDEllipse[SphericalCoor,beginAngle=\PhiII,endAngle=\PhiI]%
	(0,0,0)(\radius,90,\PhiI)(\radius,0,0)
    \pstThreeDEllipse[beginAngle=\PhiI,endAngle=\PhiII](\RadIs,0,0)(0,\RadIc,0)(0,0,\RadIc)
\end{pspicture}
\begin{pspicture}(-4,-4)(4,5)
  \psset{Alpha=45,Beta=30,linestyle=dashed}
  \pstThreeDCoor[linestyle=solid,xMin=-5,xMax=5,yMax=5,zMax=5,IIIDticks]
  \pstThreeDEllipse[linecolor=red](0,0,0)(0,\radius,0)(0,0,\radius)
  \pstThreeDEllipse(\RadIs,0,0)(0,\RadIc,0)(0,0,\RadIc)
  \pstThreeDEllipse(\RadIIs,0,0)(0,\RadIIc,0)(0,0,\RadIIc)
%
  \pstThreeDEllipse[linestyle=dotted,SphericalCoor](0,0,0)(\radius,90,\PhiI)(\radius,0,0)
  \pstThreeDEllipse[SphericalCoor,
	beginAngle=-90,endAngle=90](0,0,0)(\radius,90,\PhiI)(\radius,0,0)
  \pstThreeDEllipse[linestyle=dotted,SphericalCoor](0,0,0)(\radius,90,\PhiII)(\radius,0,0)
  \pstThreeDEllipse[SphericalCoor,
	beginAngle=-90,endAngle=90](0,0,0)(\radius,90,\PhiII)(\radius,0,0)
%
  \pscustom[fillstyle=solid,fillcolor=blue]{
    \pstThreeDEllipse[SphericalCoor,beginAngle=\PhiI,endAngle=\PhiII]%
	(0,0,0)(\radius,90,\PhiII)(\radius,0,0)
    \pstThreeDEllipse[beginAngle=\PhiII,endAngle=\PhiI](\RadIIs,0,0)(0,\RadIIc,0)(0,0,\RadIIc)
    \pstThreeDEllipse[SphericalCoor,beginAngle=\PhiII,endAngle=\PhiI]%
	(0,0,0)(\radius,90,\PhiI)(\radius,0,0)
    \pstThreeDEllipse[beginAngle=\PhiI,endAngle=\PhiII](\RadIs,0,0)(0,\RadIc,0)(0,0,\RadIc)
  }
\end{pspicture}
}
\egroup
\end{center}

\begin{lstlisting}
\def\radius{4 }\def\PhiI{20 }\def\PhiII{50 }
%
\def\RadIs{\radius \PhiI sin mul}
\def\RadIc{\radius \PhiI cos mul}
\def\RadIIs{\radius \PhiII sin mul}
\def\RadIIc{\radius \PhiII cos mul}
\begin{pspicture}(-4,-4)(4,5)
  \psset{Alpha=45,Beta=30,linestyle=dashed}
  \pstThreeDCoor[linestyle=solid,xMin=-5,xMax=5,yMax=5,zMax=5,IIIDticks]
  \pstThreeDEllipse[linecolor=red](0,0,0)(0,\radius,0)(0,0,\radius)
  \pstThreeDEllipse(\RadIs,0,0)(0,\RadIc,0)(0,0,\RadIc)
  \pstThreeDEllipse(\RadIIs,0,0)(0,\RadIIc,0)(0,0,\RadIIc)
%
  \pstThreeDEllipse[linestyle=dotted,SphericalCoor](0,0,0)(\radius,90,\PhiI)(\radius,0,0)
  \pstThreeDEllipse[SphericalCoor,
	beginAngle=-90,endAngle=90](0,0,0)(\radius,90,\PhiI)(\radius,0,0)
  \pstThreeDEllipse[linestyle=dotted,SphericalCoor](0,0,0)(\radius,90,\PhiII)(\radius,0,0)
  \pstThreeDEllipse[SphericalCoor,
	beginAngle=-90,endAngle=90](0,0,0)(\radius,90,\PhiII)(\radius,0,0)
%
  \psset{linecolor=blue,arrows=->,arrowscale=2,linewidth=1.5pt,linestyle=solid}
    \pstThreeDEllipse[SphericalCoor,beginAngle=\PhiI,endAngle=\PhiII]%
	(0,0,0)(\radius,90,\PhiII)(\radius,0,0)
    \pstThreeDEllipse[beginAngle=\PhiII,endAngle=\PhiI](\RadIIs,0,0)(0,\RadIIc,0)(0,0,\RadIIc)
    \pstThreeDEllipse[SphericalCoor,beginAngle=\PhiII,endAngle=\PhiI]%
	(0,0,0)(\radius,90,\PhiI)(\radius,0,0)
    \pstThreeDEllipse[beginAngle=\PhiI,endAngle=\PhiII](\RadIs,0,0)(0,\RadIc,0)(0,0,\RadIc)
\end{pspicture}
\begin{pspicture}(-4,-4)(4,5)

[ ... ]

  \pstThreeDEllipse[linestyle=dotted,SphericalCoor](0,0,0)(\radius,90,\PhiI)(\radius,0,0)
  \pstThreeDEllipse[SphericalCoor,
	beginAngle=-90,endAngle=90](0,0,0)(\radius,90,\PhiI)(\radius,0,0)
  \pstThreeDEllipse[linestyle=dotted,SphericalCoor](0,0,0)(\radius,90,\PhiII)(\radius,0,0)
  \pstThreeDEllipse[SphericalCoor,
	beginAngle=-90,endAngle=90](0,0,0)(\radius,90,\PhiII)(\radius,0,0)
%
  \pscustom[fillstyle=solid,fillcolor=blue]{
    \pstThreeDEllipse[SphericalCoor,beginAngle=\PhiI,endAngle=\PhiII]%
	(0,0,0)(\radius,90,\PhiII)(\radius,0,0)
    \pstThreeDEllipse[beginAngle=\PhiII,endAngle=\PhiI](\RadIIs,0,0)(0,\RadIIc,0)(0,0,\RadIIc)
    \pstThreeDEllipse[SphericalCoor,beginAngle=\PhiII,endAngle=\PhiI]%
	(0,0,0)(\radius,90,\PhiI)(\radius,0,0)
    \pstThreeDEllipse[beginAngle=\PhiI,endAngle=\PhiII](\RadIs,0,0)(0,\RadIc,0)(0,0,\RadIc)
  }
\end{pspicture}
\end{lstlisting}

% ---------------------------------------------------------------------------------------
\section{\CMD{pstIIIDCylinder}}
% ---------------------------------------------------------------------------------------
The syntax is 
 
\begin{verbatim}
\pstIIIDCylinder[Parameter](x,y,z){radius}{height}
\end{verbatim}

\verb+(x,y,z)+ defines the center of the lower part of the cylinder. If it is
missing, then \verb+(0,0,0)+ are taken into account.

\begin{LTXexample}[width=6.5cm]
\psframebox{%
\begin{pspicture}(-3.5,-2)(3,6)
\pstThreeDCoor[zMax=6]
\pstIIIDCylinder{2}{5}
\end{pspicture}
}
\end{LTXexample}

\begin{LTXexample}[width=6.5cm]
\psframebox{%
\begin{pspicture}(-3.5,-2)(3,6.75)
\pstThreeDCoor[zMax=7]
\pstIIIDCylinder[RotY=30,fillstyle=solid,
    fillcolor=red!20,linecolor=black!60](0,0,0){2}{5}
\end{pspicture}
}
\end{LTXexample}

\begin{LTXexample}[width=6.5cm]
\psframebox{%
\begin{pspicture}(-3.2,-1.75)(3,6.25)
\pstThreeDCoor[zMax=7]
\pstIIIDCylinder[linecolor=black!20,
    increment=0.4,fillstyle=solid]{2}{5}
\psset{linecolor=red}
\pstThreeDLine{->}(0,0,5)(0,0,7)
\end{pspicture}
}
\end{LTXexample}

\begin{LTXexample}[width=7.5cm]
\psframebox{%
\begin{pspicture}(-4.5,-1.5)(3,6.8)
\psset{Beta=20}
\pstThreeDCoor[zMax=7]
\pstIIIDCylinder[fillcolor=blue!20,
    RotX=45](1,1,0){2}{5}
\end{pspicture}
}
\end{LTXexample}



% ---------------------------------------------------------------------------------------
\section{\CMD{psCylinder}}
% ---------------------------------------------------------------------------------------
The syntax is 
 
\begin{verbatim}
\psCylinder[Parameter](x,y,z){radius}{height}
\end{verbatim}

\verb+(x,y,z)+ defines the center of the lower part of the cylinder. If it is
missing, then \verb+(0,0,0)+ are taken into account.

\begin{LTXexample}[width=7cm]
\begin{pspicture}(-3,-2)(3,7)
\psset{Beta=10}
\pstThreeDCoor[zMax=7]
  \psCylinder[increment=5]{2}{5}
\end{pspicture}
\end{LTXexample}

\begin{LTXexample}[width=7cm]
\begin{pspicture}(-3,-2)(3,6)
\psset{Beta=60}
\pstThreeDCoor[zMax=9]
  \psCylinder[RotX=10,increment=5]{3}{5}
  \pstThreeDLine[linecolor=red](0,0,0)(0,0,8.5)
\end{pspicture}
\end{LTXexample}

\begin{LTXexample}[width=7cm]
\begin{pspicture}(-3,-2)(3,6)
\psset{Beta=60}
\pstThreeDCoor[zMax=9]
  \psCylinder[RotX=10,RotY=45,showInside=false]{2}{5}
  \pstThreeDLine[linecolor=red](0,0,0)(0,0,8.5)
\end{pspicture}
\end{LTXexample}

\begin{LTXexample}[width=7cm]
\begin{pspicture}(-3,-2)(3,6)
\psset{Beta=60}
\pstThreeDCoor[zMax=9]
  \psCylinder[RotY=-45](0,1,0){2}{5}
\end{pspicture}
\end{LTXexample}


\clearpage

% ---------------------------------------------------------------------------------------
\section{\CMD{pstParaboloid}}
% ---------------------------------------------------------------------------------------
The syntax is 
 
\begin{verbatim}
\pstParaboloid[Parameter]{height}{radius}
\end{verbatim}
 
\verb+height+ and \verb+radius+ depend to each other, it is the radius of the circle
at the height. By default the paraboloid is placed in the origin of coordinate system, but
with \verb+\pstThreeDput+ it can be placed anywhere. The possible options are listed in 
table~\ref{tab:paraboloid}.
The segment color must be set as a cmyk color \verb|SegmentColor={[cmyk]{c,m,y,k}}| in parenthesis,
otherwise \verb|xcolor| cannot read the values. A white color is given by \verb|SegmentColor={[cmyk]{0,0,0,0}}|.

\begin{table}[htb]
\centering
\caption{Options for the \Lcs{pstParaboloid}  macro}\label{tab:paraboloid}
\smallskip
\begin{tabular}{l|l}
\textbf{Option name} & \textbf{value}\\\hline
\verb|SegmentColor| & cmyk color for the segments (0.2,0.6,1,0)\\
\verb|showInside|  & show inside  (true)\\
\verb|increment|  & number for the segments (10)
\end{tabular}
\end{table}



% x=radius/sqrt(h)*V*cos(V)
% y=radius/sqrt(h)*V*sin(V)
% z=radius/sqrt(h)*V*V
\begin{LTXexample}[width=4cm]
\begin{pspicture}(-2,-1)(2,5)
\pstThreeDCoor[xMax=2,yMax=2,zMin=0,zMax=6,IIIDticks]%
\pstParaboloid{5}{1}% Hoehe 5 und Radius 1
\end{pspicture}
\end{LTXexample}

\begin{LTXexample}[width=.65\linewidth,wide]
\begin{pspicture}(-.25\linewidth,-1)%
    (.25\linewidth,7.5)
\pstParaboloid[showInside=false,
    SegmentColor={[cmyk]{0.8,0.1,.11,0}}]{4}{5}%
\pstThreeDCoor[xMax=3,yMax=3,
    zMax=7.5,IIIDticks]
\end{pspicture}
\end{LTXexample}


\begin{LTXexample}[width=9cm,wide]
\begin{pspicture}(0,-3)(7,5)
\pstThreeDCoor[xMax=2,yMax=13,zMin=0,zMax=6,IIIDticks]%
\multido{\rA=2.0+2.5,
  \rB=0.15+0.20}{5}{%
    \pstParaboloid[%
      SegmentColor={[cmyk]%
       {\rB,0.1,0.11,0.1}}]%
    (0,\rA,0){5}{1}}% height 5 and radius 1
\pstThreeDLine[linestyle=dashed]{->}(0,0,5)(0,13,5)
\end{pspicture}
\end{LTXexample}


\clearpage
\section{Spheres}\label{sec:spheres}

\begin{LTXexample}[width=6.25cm]
\begin{pspicture}(-4,-2.25)(2,4.25)
  \pstThreeDCoor[xMin=-3,yMax=2]
  \pstThreeDSphere(1,-1,2){2}
  \pstThreeDDot[dotstyle=x,linecolor=red,drawCoor=true](1,-1,2)
\end{pspicture}
\end{LTXexample}

\begin{verbatim}
\pstThreeDSphere[<options>](x,y,z){Radius}
\end{verbatim}

\verb|(x,y,z)| is the center of the sphere and possible options are listed in table~\ref{tab:sphereOptions}.
The segment color must be set as a cmyk color \verb|SegmentColor={[cmyk]{c,m,y,k}}| in parenthesis,
otherwise \verb|xcolor| cannot read the values. A white color is given by \verb|SegmentColor={[cmyk]{0,0,0,0}}|.

\begin{table}[htb]
\centering
\caption{Options for the sphere macro}\label{tab:sphereOptions}
\smallskip
\begin{tabular}{l|l}
\textbf{Option name} & \textbf{value}\\\hline
\verb|SegmentColor| & cmyk color for the segments (0.2,0.6,1,0)\\
\verb|increment|  & number for the segments (10)
\end{tabular}
\end{table}


\begin{LTXexample}[width=6.25cm]
\begin{pspicture}(-4,-2.25)(2,4.25)
  \pstThreeDCoor[xMin=-3,yMax=2]
  \pstThreeDSphere[SegmentColor={[cmyk]{0,0,0,0}}](1,-1,2){2}
  \pstThreeDDot[dotstyle=x,linecolor=red,drawCoor=true](1,-1,2)
\end{pspicture}
\end{LTXexample}



\section{Mathematical functions}
There are two macros for plotting mathematical functions, which work similiar to the one from \verb|pst-plot|.

\subsection{Function $f(x,y)$}

The macro for plotting functions does not have the same syntax as the one from 
\verb|pst-plot|\cite{dtk02.1:voss:mathematischen}, but it is used in the same way:

\begin{verbatim}
\psplotThreeD[<options>](xMin,xMax)(yMin,yMax){<the function>}
\end{verbatim}

The function has to be written in \PS{} code and the only valid variable names are  \verb|x| 
and \verb|y|, f.ex: \verb|{x dup mul y dup mul add sqrt}| for the math expression $\sqrt{x^2 + y^2}$. 
The macro has the same plotstyle options as \verb|psplot|, except the \verb|plotpoints|-option which is 
split into one for \verb|x| and one for \verb|y| (table \ref{tab:lineOptions}).

\begin{table}[htb]
\centering
\caption{Options for the plot Macros}\label{tab:lineOptions}
\smallskip
\begin{tabular}{l|l}
\textbf{Option name} & \textbf{value}\\\hline
\verb|plotstyle| &
	\verb|dots|\\
&	\verb|line|\\
&	\verb|polygon|\\
&	\verb|curve|\\
&	\verb|ecurve|\\
&	\verb|ccurve|\\
&	\verb|none| (default)\\
\verb|showpoints|  & default is false\\
\verb|xPlotpoints| & default is $25$\\
\verb|yPlotpoints| & default is $25$\\
\verb|drawStyle| & default is \verb|xLines|\\
& \verb|yLines|\\
& \verb|xyLines|\\
& \verb|yxLines|\\
\verb|hiddenLine|  & default is false\\
\verb|algebraic| & default is false
\end{tabular}
\end{table}

The equation \ref{eq:3dfunc} is plotted with the following parameters and seen in figure \ref{fig:3dfunc}.

\begin{align}\label{eq:3dfunc}
z&=10\left(x^3+xy^4-\frac{x}{5}\right)e^{-\left(x^2+y^2\right)}+
	e^{-\left((x-1.225)^2+y^2\right)}
\end{align}


The function is calculated within two loops:
{\small\begin{verbatim}
for (float y=yMin; y<yMax; y+=dy)
    for (float x=xMin; x<xMax; x+=dx)
         z=f(x,y);
\end{verbatim}}

It depends to the inner loop in which direction the curves are drawn. There are four possible
values for the option \verb|drawStyle| :
\begin{itemize}
\item \verb|xLines| (default) Curves are drawn in x direction
\item \verb|yLines| Curves are drawn in y direction
\item \verb|xyLines| Curves are first drawn in x and then in y direction
\item \verb|yxLines| Curves are first drawn in y and then in x direction
\end{itemize}

In fact of the inner loop it is only possible to get a closed curve in the defined direction. 
For lines in x direction less \verb|yPlotpoints| are no problem, in difference to 
\verb|xPlotpoints|, especially for the plotstyle options \verb|line| and \verb|dots|.

Drawing three dimensional functions with curves which are transparent makes it difficult 
to see if a point is before or behind another one. \verb|\psplotThreeD| has an option 
\verb|hiddenLine| for a primitive hidden line mode, which only works when the y-intervall 
is defined in a way that $y_2>y_1$. Then every new curve is plotted over the forgoing one 
and filled with the color white. Figure~\ref{fig:3dfunc-hidden} is the same as 
figure~\ref{fig:3dfunc}, only with the option \verb|hiddenLine=true|.

\begin{lstlisting}
\begin{pspicture}(-6,-4)(6,5)
  \psset{Beta=15}
  \psplotThreeD[plotstyle=line,drawStyle=xLines,% is the default anyway
    yPlotpoints=50,xPlotpoints=50,linewidth=1pt](-4,4)(-4,4){%
    x 3 exp x y 4 exp mul add x 5 div sub 10 mul
    2.729 x dup mul y dup mul add neg exp mul
    2.729 x 1.225 sub dup mul y dup mul add neg exp add}
  \pstThreeDCoor[xMin=-1,xMax=5,yMin=-1,yMax=5,zMin=-1,zMax=5]
\end{pspicture}
\end{lstlisting}


\begin{figure*}
\centering
\begin{pspicture}(-6,-4)(6,5)
 \psset{Beta=15}
  \pstThreeDCoor[xMin=-1,xMax=5,yMin=-1,yMax=5,zMin=-1,zMax=5]
  \psplotThreeD[
      algebraic,   
      plotstyle=curve,
      yPlotpoints=50,xPlotpoints=50,
      linewidth=0.5pt](-4,4)(-4,4){10*(x^3+x*y^4-x/5)*Euler^(-x^2-y^2)+Euler^(-((x-1.225)^2+y^2))}
\end{pspicture}
\caption{Plot of the equation \ref{eq:3dfunc}}\label{fig:3dfunc}
\end{figure*}


\begin{figure*}
\centering
\begin{pspicture}(-6,-4)(6,5)
	\psset{Alpha=45,Beta=15}
	\pstThreeDCoor[xMin=-1,xMax=5,yMin=-1,yMax=5,zMin=-1,zMax=5]
	\psplotThreeD[%
		plotstyle=curve,%
		yPlotpoints=50,xPlotpoints=50,%
		linewidth=0.5pt,hiddenLine=true](-4,4)(-4,4){%
		x 3 exp x y 4 exp mul add x 5 div sub 10 mul
           2.729 x dup mul y dup mul add neg exp mul
           2.729 x 1.225 sub dup mul y dup mul add neg exp add}
\end{pspicture}
\caption{Plot of the equation \ref{eq:3dfunc} with the \texttt{hiddenLine=true} option}\label{fig:3dfunc-hidden}
\end{figure*}



\begin{figure*}
\centering
\begin{pspicture}(-6,-4)(6,5)
	\psset{Alpha=45,Beta=15}
	\pstThreeDCoor[xMin=-1,xMax=5,yMin=-1,yMax=5,zMin=-1,zMax=5]
	\psplotThreeD[%
		plotstyle=line,%
		drawStyle=yLines,%
		yPlotpoints=50,xPlotpoints=50,%
		linewidth=0.2pt](-4,4)(-4,4){%
		x 3 exp x y 4 exp mul add x 5 div sub 10 mul
           2.729 x dup mul y dup mul add neg exp mul
           2.729 x 1.225 sub dup mul y dup mul add neg exp add}
\end{pspicture}
\caption{Plot of the equation \ref{eq:3dfunc} with the \texttt{drawStyle=yLines} option}
\end{figure*}

\begin{figure*}
\centering
\begin{pspicture}(-6,-4)(6,5)
	\psset{Alpha=45,Beta=15}
	\pstThreeDCoor[xMin=-1,xMax=5,yMin=-1,yMax=5,zMin=-1,zMax=5]
	\psplotThreeD[%
		plotstyle=curve,%
		drawStyle=yLines,%
		hiddenLine=true,%
		yPlotpoints=50,xPlotpoints=50,%
		linewidth=0.2pt](-4,4)(-4,4){%
		x 3 exp x y 4 exp mul add x 5 div sub 10 mul
           2.729 x dup mul y dup mul add neg exp mul
           2.729 x 1.225 sub dup mul y dup mul add neg exp add}
\end{pspicture}
\caption{Plot of the equation \ref{eq:3dfunc} with the \texttt{drawStyle=yLines} and
\texttt{hiddenLine=true} option}
\end{figure*}




\begin{figure*}
\centering
\begin{pspicture}(-6,-4)(6,5)
	\psset{Alpha=45,Beta=15}
	\pstThreeDCoor[xMin=-1,xMax=5,yMin=-1,yMax=5,zMin=-1,zMax=5]
	\psplotThreeD[%
		plotstyle=line,%
		drawStyle=xyLines,%
		yPlotpoints=50,xPlotpoints=50,%
		linewidth=0.2pt](-4,4)(-4,4){%
		x 3 exp x y 4 exp mul add x 5 div sub 10 mul
           2.729 x dup mul y dup mul add neg exp mul
           2.729 x 1.225 sub dup mul y dup mul add neg exp add}
\end{pspicture}
\caption{Plot of the equation \ref{eq:3dfunc} with the \texttt{drawStyle=xyLines} option}
\end{figure*}

\begin{figure*}[htbp]
\centering
\begin{pspicture}(-6,-4)(6,5)
	\psset{Alpha=45,Beta=15}
	\pstThreeDCoor[xMin=-1,xMax=5,yMin=-1,yMax=5,zMin=-1,zMax=5]
	\psplotThreeD[%
		plotstyle=curve,%
		drawStyle=xLines,%
		hiddenLine=true,%
		yPlotpoints=50,xPlotpoints=50,%
		linewidth=0.2pt](-4,4)(-4,4){%
		x 3 exp x y 4 exp mul add x 5 div sub 10 mul
           2.729 x dup mul y dup mul add neg exp mul
           2.729 x 1.225 sub dup mul y dup mul add neg exp add}
\end{pspicture}
\caption{Plot of the equation \ref{eq:3dfunc} with the \texttt{drawStyle=xLines} and
\texttt{hiddenLine=true} option}
\end{figure*}

\begin{figure*}[htbp]
\centering
\begin{pspicture}(-6,-4)(6,5)
  \psset{Alpha=45,Beta=15}
  \pstThreeDCoor[xMin=-1,xMax=5,yMin=-1,yMax=5,zMin=-1,zMax=5]
  \psplotThreeD[%
    plotstyle=curve,%
    drawStyle=yLines,%
    fillstyle=gradient,%
    yPlotpoints=50,xPlotpoints=50,%
    linewidth=0.2pt](-4,4)(-4,4){%
    x 3 exp x y 4 exp mul add x 5 div sub 10 mul
    2.729 x dup mul y dup mul add neg exp mul
    2.729 x 1.225 sub dup mul y dup mul add neg exp add}
\end{pspicture}
\caption{Plot of the equation \ref{eq:3dfunc} with the \texttt{drawStyle=yLines} and
\texttt{hiddenLine=true} option}
\end{figure*}



\subsection{Parametric Plots}\label{subsec:parametricplotThreeD}
Parametric plots are only possible for  drawing curves or areas.  The syntax for this plot macro is:
\begin{verbatim}
\parametricplotThreeD(t1,t2)(u1,u2){<three parametric functions x y z}
\end{verbatim}

The only possible variables are \verb|t| and \verb|u| with \verb|t1,t2| and  \verb|u1,u2| as the 
range for the parameters.  The order for the functions is not important and \verb|u| may be 
optional when having only a three dimensional curve and not an area.
\begin{align}
\begin{array}{rl}
	x & =f(t,u)\\
	y & =f(t,u)\\
	z & =f(t,u)
\end{array}
\end{align}
To draw a spiral we have the parametric functions:
\begin{align}
\begin{array}{rl}
	x & = r \cos t\\
	y & = r \sin t\\
	z & = t/600
\end{array}
\end{align}

In the example the $t$ value is divided by $600$ for the \verb|z| coordinate, because we have the 
values for $t$ in degrees, here with a range of $0\mbox{�}\ldots 2160\mbox{�}$. Drawing a curve in 
a three dimensional coordinate system does only require one parameter, which has to be by default 
\verb|t|. In this case we do not need all parameters, so that one can write

\begin{verbatim}
\parametricplotThreeD(t1,t2){<three parametric functions x y z}
\end{verbatim}

which is the same as \verb|(0,0)| for the parameter \verb|u|.

\begin{LTXexample}[width=6.75cm]
\begin{pspicture}(-3.25,-2.25)(3.25,5.25)
  \pstThreeDCoor[zMax=5]
  \parametricplotThreeD[xPlotpoints=200,linecolor=blue,%
    linewidth=1.5pt,plotstyle=curve](0,2160){%
    2.5 t cos mul 2.5 t sin mul t 600 div}
\end{pspicture}
\end{LTXexample}


Instead of using the \verb|\pstThreeDSphere| macro (see section \ref{sec:spheres}) it is  also 
possible to use parametric functions for a sphere. The macro plots continous lines only for 
the \verb|t| parameter, so a sphere plotted with the longitudes need the parameter equations as
\begin{align}
\begin{array}{l}
x = \cos t \cdot \sin u\\
y = \cos t \cdot \cos u\\
z = \sin t
\end{array}
\end{align}

The same is possible for a sphere drawn with the latitudes:
\begin{align}
\begin{array}{l}
x = \cos u \cdot \sin t\\
y = \cos u \cdot \cos t\\
z = \sin u
\end{array}
\end{align}

and at last both together is also not a problem when having these parametric functions together in one \verb|pspicture| environment (see figure \ref{fig:paraSpheres}). 

\begin{lstlisting}
\begin{pspicture}(-1,-1)(1,1)
\parametricplotThreeD[plotstyle=curve,yPlotpoints=40](0,360)(0,360){%
  t cos u sin mul t cos u cos mul t sin
}
\parametricplotThreeD[plotstyle=curve,yPlotpoints=40](0,360)(0,360){%
  u cos t sin mul  u cos t cos mul u sin
}
\end{pspicture}
\end{lstlisting}


\begin{figure}[htbp]
{\psset{xunit=1.75cm,yunit=1.75cm}
\begin{pspicture}(-1,-1)(1,1)
\setIIIDplotDefaults
%\pstThreeDCoor[xMin=-1,xMax=1,yMin=-1,yMax=1,zMin=-1,zMax=1]
\parametricplotThreeD[plotstyle=curve](0,360)(0,360){%
	t cos u sin mul
	t cos u cos mul
	t sin
}
\end{pspicture}\hfill%
\begin{pspicture}(-1,-1)(1,1)

%\pstThreeDCoor[xMin=-1,xMax=1,yMin=-1,yMax=1,zMin=-1,zMax=1]
\parametricplotThreeD[plotstyle=curve](0,360)(0,360){%
	u cos t sin mul
	u cos t cos mul
	u sin
}
\end{pspicture}}\hfill
{\psset{xunit=2.5cm,yunit=2.5cm}
\begin{pspicture}(-1,-1)(1,1)

\pstThreeDCoor[xMin=-1,xMax=1,yMin=-1,yMax=1,zMin=-1,zMax=1]
\parametricplotThreeD[plotstyle=curve](0,360)(0,360){%
	t cos u sin mul
	t cos u cos mul
	t sin
}
\parametricplotThreeD[plotstyle=curve](0,360)(0,360){%
	u cos t sin mul
	u cos t cos mul
	u sin
}
\end{pspicture}}
\caption{Different Views of the same Parametric Functions}\label{fig:paraSpheres} 
\end{figure}



\section{Plotting data files}
There are the same conventions for data files which holds 3D coordinates, than for the 2D one. For example:
\begin{verbatim}
 0.0000  1.0000  0.0000
-0.4207  0.9972  0.0191
....

0.0000, 1.0000, 0.0000
-0.4207, 0.9972, 0.0191
....

(0.0000,1.0000,0.0000)
(-0.4207,0.9972,0.0191)
....

{0.0000,1.0000,0.0000}
{-0.4207,0.9972,0.0191}
....
\end{verbatim}

There are the same three plot functions:
\begin{verbatim}
\fileplotThreeD[<options>]{<datafile>}
\dataplotThreeD[<options>]{<data object>}
\listplotThreeD[<options>]{<data object>}
\end{verbatim}

The in the following examples used data file has 446 entries like 

\begin{verbatim}
6.26093349..., 2.55876582..., 8.131984...
\end{verbatim}

This may take some time on slow machines when using the \CMD{listplotThreeD} macro. 
The possible options for the lines are the ones from table \ref{tab:lineOptions}. 

\subsection{\textbackslash\texttt{fileplotThreeD}}
The syntax is very easy

{\footnotesize\begin{verbatim}
\fileplotThreeD[<options>]{<datafile>}
\end{verbatim}}

If the data file is not in the same directory than the document, insert the file name 
with the full path. Figure \ref{fig:fileplot} shows a file plot with the option \texttt{linestyle=line}.


\begin{figure}[!htbp]
\begin{LTXexample}[pos=t]
\begin{pspicture}(-6,-3)(6,10)
  \psset{xunit=0.5cm,yunit=0.75cm,Alpha=30,Beta=30}% the global parameters
  \pstThreeDCoor[xMin=-10,xMax=10,yMin=-10,yMax=10,zMin=-2,zMax=10]
  \fileplotThreeD[plotstyle=line]{data3D.Roessler}
\end{pspicture}%
\end{LTXexample}%
\caption{Demonstration of \CMD{fileplotThreeD} with \texttt{Alpha=30} and \texttt{Beta=15}}\label{fig:fileplot}
\end{figure}


\subsection{\CMD{dataplotThreeD}}
The syntax is 

\begin{verbatim}
\dataplotThreeD[<options>]{<data object>}
\end{verbatim}

In difference to the macro \CMD{fileplotThreeD} the \CMD{dataplotThreeD} cannot plot any external data 
without reading this with the macro \CMD{readdata} which reads external data and save it in a macro, 
f.ex.: \CMD{dataThreeD}.\cite{dtk02.2:jackson.voss:plot-funktionen}

\begin{verbatim}
\readdata{<data object>}{<datafile>}
\end{verbatim}

\readdata{\dataThreeD}{data3D.Roessler}
\begin{figure}[htbp]
\begin{LTXexample}[width=8.5cm]
\begin{pspicture}(-4.5,-3.5)(4,11)
  \psset{xunit=0.5cm,yunit=0.75cm,Alpha=-30}
  \pstThreeDCoor[xMin=-10,xMax=10,yMin=-10,yMax=10,zMin=-2,zMax=10]
  \dataplotThreeD[plotstyle=line]{\dataThreeD}
\end{pspicture}%
\end{LTXexample}
\caption{Demonstration of \texttt{\textbackslash dataplotThreeD} with \texttt{Alpha=-30} and 
\texttt{Beta=30}}\label{fig:fileplot}
\end{figure}


\subsection{\CMD{listplotThreeD}}
The syntax is 
\begin{verbatim}
\listplotThreeD[<options>]{<data object>}
\end{verbatim}

\CMD{listplotThreeD} ist similiar to \CMD{dataplotThreeD}, so it cannot plot any external data in a direct way, too. But \CMD{readdata} reads external data and saves it in a macro, f.ex.: \CMD{dataThreeD}.\cite{dtk02.2:jackson.voss:plot-funktionen} \CMD{listplot} can handle some additional PostScript code, which can be appended to the data object, f.ex.:

\readdata{\data}{data3D.Roessler}
\newcommand{\dataThreeDDraft}{%
  \data\space
  gsave          % save grafic status
  /Helvetica findfont 40 scalefont setfont
  45 rotate      % rotate 45 degrees
  0.9 setgray    % 1 ist white
  -60 30 moveto (DRAFT) show
  grestore
}

\begin{lstlisting}
\dataread{\data}{data3D.Roessler}
\newcommand{\dataThreeDDraft}{%
  \data\space
  gsave          % save grafic status
  /Helvetica findfont 40 scalefont setfont
  45 rotate      % rotate 45 degrees
  0.9 setgray    % 1 ist white
  -60 30 moveto (DRAFT) show
  grestore
}
\end{lstlisting}

\begin{figure}[htb]
\begin{LTXexample}[pos=t]
\begin{pspicture}(-5,-4)(5,4)
  \psset{xunit=0.5cm,yunit=0.5cm,Alpha=0,Beta=90}
  \pstThreeDCoor[xMin=-10,xMax=10,yMin=-10,yMax=7.5,zMin=-2,zMax=10]
  \listplotThreeD[plotstyle=line]{\dataThreeDDraft}
\end{pspicture}%
\end{LTXexample}%
\caption{Demonstration of \texttt{\textbackslash listplotThreeD} with a view from above (\texttt{Alpha=0} and \texttt{Beta=90}) and some additional PostScript code}\label{fig:listplot}
\end{figure}

Figure \ref{fig:listplot} shows what happens with this code. For another example see \cite{dtk02.1:voss:mathematischen}, where the macro \texttt{ScalePoints} is modified. This macro is in \texttt{pst-3dplot} called \texttt{ScalePointsThreeD}.

\section{Utility macros}

\subsection{Rotation of three dimensional coordinates}

With the three optional arguments \verb+RotX RotY RotZ+ one can rotate a three dimensional
point. This makes only sense when one wants to save the coordinates. In general it is more
powerful to use directly the optional parameters \verb+RotX+, \verb+RotY+, \verb+RotZ+ for
the plot macros. However, the macro syntax is
%
\begin{verbatim}
\pstRotPOintIIID[RotX=...,RotY=...,RotZ=...](<x,y,z>)<\xVal><\yVal><\zVal>
\end{verbatim}

the \verb+\xVal \yVal \zVal+ hold the new rotated coordinates and must be defined by the user like \verb+\def\xVal{}+,
where the name of the macro is not important.

The rotation angles are all predefined to $0$ degrees.

\begin{center}
\def\xVal{}\def\yVal{}\def\zVal{}
\begin{pspicture}(-6,-4)(6,5)
  \pstThreeDCoor[xMin=-1,xMax=5,yMin=-1,yMax=5,zMin=-1,zMax=5]
  \multido{\iA=0+10}{36}{\pstRotPointIIID[RotX=\iA](2,0,3){\xVal}{\yVal}{\zVal}
    \pstThreeDDot[drawCoor=true](\xVal,\yVal,\zVal)
  }
\end{pspicture}

\begin{pspicture}(-6,-4)(6,5)
  \pstThreeDCoor[xMin=-1,xMax=5,yMin=-1,yMax=5,zMin=-1,zMax=5]
  \multido{\iA=0+10}{36}{\pstRotPointIIID[RotY=\iA](2,0,3){\xVal}{\yVal}{\zVal}
    \pstThreeDDot[drawCoor=true](\xVal,\yVal,\zVal)
  }
\end{pspicture}

\begin{pspicture}(-6,-4)(6,5)
  \pstThreeDCoor[xMin=-1,xMax=5,yMin=-1,yMax=5,zMin=-1,zMax=5]
  \multido{\iA=0+10}{36}{\pstRotPointIIID[RotZ=\iA](2,0,3){\xVal}{\yVal}{\zVal}
    \pstThreeDDot[drawCoor=true](\xVal,\yVal,\zVal)
  }
\end{pspicture}
\end{center}

\begin{lstlisting}
\def\xVal{}\def\yVal{}\def\zVal{}
\begin{pspicture}(-6,-4)(6,5)
  \pstThreeDCoor[xMin=-1,xMax=5,yMin=-1,yMax=5,zMin=-1,zMax=5]
  \multido{\iA=0+10}{36}{\pstRotPointIIID[RotX=\iA](2,0,3){\xVal}{\yVal}{\zVal}
    \pstThreeDDot[drawCoor=true](\xVal,\yVal,\zVal)
  }
\end{pspicture}

\begin{pspicture}(-6,-4)(6,5)
  \pstThreeDCoor[xMin=-1,xMax=5,yMin=-1,yMax=5,zMin=-1,zMax=5]
  \multido{\iA=0+10}{36}{\pstRotPointIIID[RotY=\iA](2,0,3){\xVal}{\yVal}{\zVal}
    \pstThreeDDot[drawCoor=true](\xVal,\yVal,\zVal)
  }
\end{pspicture}

\begin{pspicture}(-6,-4)(6,5)
  \pstThreeDCoor[xMin=-1,xMax=5,yMin=-1,yMax=5,zMin=-1,zMax=5]
  \multido{\iA=0+10}{36}{\pstRotPointIIID[RotZ=\iA](2,0,3){\xVal}{\yVal}{\zVal}
    \pstThreeDDot[drawCoor=true](\xVal,\yVal,\zVal)
  }
\end{pspicture}
\end{lstlisting}


\subsection{Transformation of coordinates}
To run the macros with more than 9 parameters \verb|pst-3dplot| uses
the syntax \verb|(#1)| for a collection of three coordinates \verb|(#1,#2,#3)|.
To handle these triple in PostScript the following macro is used, which converts
the parameter \verb|#1| into a sequence of the three coordinates, dived by a space.
The syntax is:
%
\begin{verbatim}
\getThreeDCoor(<vector>)<\macro>
\end{verbatim}

\verb|\macro| holds the sequence of the three coordinates \verb|x y z|, divided by a space.

\subsection{Adding two vectors}
The syntax is
\begin{verbatim}
\pstaddThreeDVec(<vector A>)(<vector B>)\tempa\tempb\tempc
\end{verbatim}

\verb|\tempa\tempb\tempc| must be user or system defined macros, which holds the three
coordinates of the vector $\vec{C}=\vec{A}+\vec{B}$.

\subsection{Substract two vectors}
The syntax is
\begin{verbatim}
\pstsubThreeDVec(<vector A>)(<vector B>)\tempa\tempb\tempc
\end{verbatim}

\verb|\tempa\tempb\tempc| must be user or system defined macros, which holds the three
coordinates of the vector $\vec{C}=\vec{A}-\vec{B}$.



\section{PDF output}

\verb|pst-3dplot| is based on the popular \verb|pstricks| package and writes pure \PS code\cite{PostScript}, so it is not possible to run \TeX{} files  with pdf\LaTeX{} when there are pstricks macros in the document. If you still need a PDF output use one of the following possibilities:
\begin{itemize}
\item package \verb|pdftricks.sty|\cite{pdftricks}
\item the for Linux free available program VTeX/Lnx\footnote{\url{http://www.micropress-inc.com/linux/}}
\item build the PDF with \verb|ps2pdf| (\verb|dvi|$\rightarrow$\verb|ps|$\rightarrow$\verb|pdf|)
\item use the \verb|ps4pdf| package.\footnote{\url{http://www.perce.de/LaTeX/ps4pdf/}}
\end{itemize}

If you need package \verb|graphicx.sty| load it before any \verb|pstricks| package. 
You do not need to load 
\verb|pstricks.sty|, it will be done by \verb|pst-3dplot| by default.

\section{FAQ}
\begin{itemize}
\item The labels for the axis are not right placed in the preview.

Be sure that you view your output with a dvi viewer which can show PostScript code, like
kdvi but not xdvi. It is better to run \verb|dvips| and then view the ps-file with \verb|gv|.
\item The three axes have a wrong intersection point.

Be sure that you have the ''newest`` \verb|pst-node.tex| file
\begin{verbatim}
\def\fileversion{97 patch 11}
\def\filedate{2000/11/09}
\end{verbatim}
and the ''newest`` \verb|pst-plot.tex|
\begin{verbatim}
\def\fileversion{97 patch 2}
\def\filedate{1999/12/12}
\end{verbatim}

\item Using \verb+amsmath+ and \verb+\hat+ or other accents as label for the axes gives an
error. In this case save prevent expanding with e.g.: \verb+\psset{nameX=$\noexpand\hat{x}$}+.
\end{itemize}



\section{Credits}
Bruce Burlton | Christophe Jorssen | Chris Kuklewicz | Thorsten Suhling 

\bgroup
\nocite{*}
\raggedright
\bibliographystyle{plain}
\bibliography{pst-3dplot-doc}
\egroup


\end{document}
