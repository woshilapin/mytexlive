\section{Un tore coup� par l'�quateur}
\bgroup\begin{LTXexample}[pos=t]
\psset{unit=0.75}
\begin{pspicture}(-5,-5)(6,6)
\psframe(-5,-5)(6,6)
\psset[pst-solides3d]{viewpoint=20 20 20,SphericalCoor,Decran=20,lightsrc=10 15 0}
% Parametric Surfaces
\psSolid[object=grille,base=-3 3 -3 3,action=draw,linecolor=red](0,0,-2)
\defFunction[algebraic]{torus}(u,v){(1+ 0.5*cos(u))*cos(v)}{(1+ 0.5*cos(u))*sin(v)}{0.5*sin(u)}
\psSolid[object=surfaceparametree,linecolor={[cmyk]{1,0,1,0.5}},
   base=pi neg 0 0 2 pi mul ,fillcolor=yellow!50,incolor=green!50,
   function=torus,linewidth=0.5\pslinewidth,unit=2,
   tracelignedeniveau=true,
   hauteurlignedeniveau=-.01,
   linewidthlignedeniveau=1,
   couleurlignedeniveau=blue,
   ngrid=20]%
\gridIIID[Zmin=-2,Zmax=2](-3,3)(-3,3)
\end{pspicture}
\end{LTXexample}

%% \defFunction[algebraic]{cercleMilieu}(t){0.5*cos(t)}{0.5*sin(t)}{0}
%% \psSolid[object=courbe,
%%     range=0 2 pi mul,unit=2,
%%     linecolor=blue,
%%     resolution=360,
%%     function=cercleMilieu]%

\egroup
