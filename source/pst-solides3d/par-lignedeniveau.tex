\section {Ligne de niveau}

Pour chaque objet de type \textsl {solid}, il est possible de tracer
une ligne de niveau, autrement dit l'intersection du solide consid�r�
avec un plan d'�quation $z=z_0$. Quatres param�tres pour cette option~:
\begin{itemize}

\item \verb+tracelignedeniveau+~: bool�en. Positionn� � \verb+true+,
  il y aura trac� de la ligne de niveau

\item \verb+hauteurlignedeniveau+~: la valeur $z_0$ d�finissant le
  plan d'intersection

\item \verb+linewidthlignedeniveau+~: l'�paisseur en picas du trac� de
  la ligne de niveau

\item \verb+couleurlignedeniveau+~: la couleur du trac�.

\end{itemize}

\begin{multicols}{2}

%\begin{center}
\bgroup
\psset{unit=0.5}
\psset{lightsrc=20 -20 10,SphericalCoor=true,viewpoint=50 -20 10,Decran=50}
\begin{pspicture*}(-5,-4)(5,5)
\psframe(-5,-4)(5,5)
\psSolid[object=cube,
   tracelignedeniveau=true,
   hauteurlignedeniveau=1,
   linewidthlignedeniveau=3,
   couleurlignedeniveau=blue,
   RotX=20,
   RotY=90,
   RotZ=30,
   a=6,
   ngrid=4,
   base=-4 4 -4 4,
](0,0,0)
\end{pspicture*}
\egroup
%\end{center}

\columnbreak

\begin{verbatim}
\psSolid[object=cube,
   tracelignedeniveau=true,
   hauteurlignedeniveau=1,
   linewidthlignedeniveau=3,
   couleurlignedeniveau=blue,
   RotX=20,
   RotY=90,
   RotZ=30,
   a=6,
   ngrid=4,
](0,0,0)
\end{verbatim}

\end {multicols}

\llap {\dbend} La syntaxe pr�sent�e dans ce paragraphe n'est que
provisoire. Cette option a �t� introduite quelques jours avant la
publication de la version $3.0$, et la syntaxe est vou�e � changer d�s
la prochaine version de \verb+pst-solides3d+, afin de permettre le
trac� de plusieurs lignes de niveau sur un m�me solide.

